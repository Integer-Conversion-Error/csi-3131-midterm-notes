\section{Summary}
\begin{itemize}
    \item Virtual memory: abstracts physical memory into extremely large uniform array of storage.
    \item Benefits of virtual memory:
    \begin{itemize}
        \item Program can be larger than physical memory.
        \item Program does not need to be entirely in memory.
        \item Processes can share memory.
        \item Processes can be created more efficiently.
    \end{itemize}
    \item \textbf{Demand paging}: pages loaded only when demanded during program execution.
    \item Pages never demanded are never loaded.
    \item \textbf{Page fault}: occurs when page not in memory is accessed.
    \item Page must be brought from backing store into available page frame.
    \item \textbf{Copy-on-write}: child process shares same address space as parent.
    \item If child or parent modifies page, copy of page is made.
    \item When available memory low: page-replacement algorithm selects existing page to replace.
    \item Page-replacement algorithms: FIFO, optimal, LRU.
    \item Pure LRU: impractical to implement; most systems use LRU-approximation algorithms.
    \item \textbf{Global page-replacement algorithms}: select page from any process for replacement.
    \item \textbf{Local page-replacement algorithms}: select page from faulting process.
    \item \textbf{Thrashing}: system spends more time paging than executing.
    \item \textbf{Locality}: set of pages actively used together.
    \item Process execution: moves from locality to locality.
    \item \textbf{Working set}: based on locality, set of pages currently in use by a process.
    \item \textbf{Memory compression}: compresses number of pages into single page.
    \item Alternative to paging, used on mobile systems without paging support.
    \item \textbf{Kernel memory}: allocated differently than user-mode processes.
    \item Allocated in contiguous chunks of varying sizes.
    \item Two common techniques for kernel memory allocation:
    \begin{itemize}
        \item Buddy system.
        \item Slab allocation.
    \end{itemize}
    \item \textbf{TLB reach}: amount of memory accessible from TLB.
    \item Equal to number of entries in TLB $\times$ page size.
    \item Technique to increase TLB reach: increase page size.
    \item Linux, Windows, Solaris: manage virtual memory similarly.
    \item Use demand paging, copy-on-write, and variations of LRU approximation (clock algorithm).
\end{itemize}
