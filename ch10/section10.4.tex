\section{10.4 Page replacement}

\subsection{Over-allocating Memory}
\begin{itemize}
    \item Demand paging saves I/O by loading only used pages.
    \item Can increase degree of multiprogramming by \textbf{over-allocating} memory.
    \item Example: 6 processes (10 pages each, use 5) on 40 frames $\rightarrow$ higher CPU utilization/throughput.
    \item Problem: Processes may suddenly need all pages (e.g., 60 frames needed, only 40 available).
    \item System memory also used for I/O buffers, increasing strain on memory-placement.
    \item Over-allocation manifests as page fault with no free frames.
\end{itemize}

\subsection{Basic Page Replacement}
\begin{itemize}
    \item If no free frame, find one not in use and free it.
    \item Freeing a frame: write contents to swap space, change page table (page no longer in memory).
    \item Use freed frame for faulted page.
    \item Modified page-fault service routine:
    \begin{enumerate}
        \item Find desired page on secondary storage.
        \item Find a free frame:
        \begin{enumerate}
            \item If free frame, use it.
            \item If no free frame, use page-replacement algorithm to select \textbf{victim frame}.
            \item Write victim frame to secondary storage (if modified); update page/frame tables.
        \end{enumerate}
        \item Read desired page into newly freed frame; update page/frame tables.
        \item Continue process from page fault.
    \end{enumerate}
    \item No free frames $\rightarrow$ two page transfers (page-out, page-in) $\rightarrow$ doubles page-fault service time.
    \item Reduce overhead with \textbf{modify bit} (or \textbf{dirty bit}).
    \begin{itemize}
        \item Hardware sets modify bit if page written to.
        \item If modify bit set: write page to storage before replacement.
        \item If modify bit not set: no need to write to storage (page unchanged).
        \item Applies to read-only pages (discardable).
        \item Significantly reduces page-fault service time if page not modified.
    \end{itemize}
    \item Page replacement separates logical and physical memory.
    \item Enormous virtual memory on smaller physical memory.
    \item Two major problems for demand paging:
    \begin{itemize}
        \item \textbf{Frame-allocation algorithm}: how many frames to allocate to each process.
        \item \textbf{Page-replacement algorithm}: which frames to replace.
    \end{itemize}
    \item Goal: lowest page-fault rate.
    \item Evaluate algorithms using \textbf{reference string} (trace of memory accesses).
    \item Reference string simplification: only page number, ignore immediate repeated references.
    \item More frames $\rightarrow$ fewer page faults (generally).
\end{itemize}

\subsection{FIFO Page Replacement}
\begin{itemize}
    \item First-in, first-out (FIFO) algorithm.
    \item Replaces the oldest page (first one brought into memory).
    \item Can use a FIFO queue: replace head, insert new page at tail.
    \item Easy to understand and program.
    \item Performance not always good: may replace actively used pages.
    \item Bad replacement choice $\rightarrow$ increased page-fault rate, slowed execution (not incorrect).
    \item Suffers from \textbf{Belady's anomaly}: page-fault rate may \textbf{increase} as allocated frames increase.
\end{itemize}

\subsection{Optimal Page Replacement}
\begin{itemize}
    \item \textbf{Optimal page-replacement algorithm} (OPT or MIN).
    \item Lowest page-fault rate, never suffers from Belady's anomaly.
    \item Rule: Replace the page that will not be used for the longest period of time.
    \item Guarantees lowest possible page-fault rate.
    \item Difficult to implement: requires future knowledge of reference string.
    \item Used mainly for comparison studies.
\end{itemize}

\subsection{LRU Page Replacement}
\begin{itemize}
    \item \textbf{Least recently used (LRU) algorithm}: approximation of optimal.
    \item Replaces the page that \textbf{has not been used} for the longest period of time.
    \item Optimal algorithm looking backward in time.
    \item Does not suffer from Belady's anomaly (is a \textbf{stack algorithm}).
    \item Implementation requires substantial hardware assistance.
    \begin{itemize}
        \item \textbf{Counters}: associate time-of-use field with page-table entry; CPU logical clock increments; copy clock to field on reference. Replace page with smallest time value.
        \item \textbf{Stack}: keep stack of page numbers; on reference, remove page and put on top. Most recently used at top, least recently used at bottom. Best with doubly linked list.
    \end{itemize}
    \item True LRU implementation is expensive due to per-memory-reference updates.
\end{itemize}

\subsection{LRU-approximation Page Replacement}
\begin{itemize}
    \item Many systems lack hardware for true LRU.
    \item Use \textbf{reference bit}: hardware sets bit when page referenced.
    \item OS clears bits periodically. Can determine which pages used, but not order.
    \item Basis for LRU approximation algorithms.
    \item \textbf{Additional-reference-bits algorithm}:
    \begin{itemize}
        \item Keep 8-bit byte for each page.
        \item Timer interrupt (e.g., every 100ms): OS shifts reference bit into high-order bit of byte, shifts others right.
        \item 8-bit shift registers show history of page use.
        \item Page with lowest number (interpreted as unsigned int) is LRU.
        \item Can replace all with smallest value or use FIFO among them.
    \end{itemize}
    \item \textbf{Second-chance page-replacement algorithm} (\textbf{clock} algorithm):
    \begin{itemize}
        \item Basic FIFO.
        \item If selected page's reference bit is 0, replace it.
        \item If reference bit is 1, give second chance: clear bit, reset arrival time to current time.
        \item Page with second chance not replaced until others replaced/given second chances.
        \item Implemented as circular queue with pointer.
        \item Pointer advances, clearing reference bits, until 0-bit page found.
        \item Degenerates to FIFO if all bits set.
    \end{itemize}
    \item \textbf{Enhanced second-chance algorithm}:
    \begin{itemize}
        \item Considers (reference bit, modify bit) as ordered pair.
        \item Four classes:
        \begin{enumerate}
            \item (0, 0): neither recently used nor modified (best to replace).
            \item (0, 1): not recently used but modified (needs write-out).
            \item (1, 0): recently used but clean (likely used again soon).
            \item (1, 1): recently used and modified (likely used again soon, needs write-out).
        \end{enumerate}
        \item Replace first page in lowest nonempty class.
        \item Prefers clean pages to reduce I/Os.
    \end{itemize}
\end{itemize}

\subsection{Counting-based Page Replacement}
\begin{itemize}
    \item Keep counter of references for each page.
    \item \textbf{Least frequently used (LFU)}: replace page with smallest count.
    \begin{itemize}
        \item Problem: heavily used page initially, then unused, keeps high count.
        \item Solution: shift counts right periodically (exponentially decaying average).
    \end{itemize}
    \item \textbf{Most frequently used (MFU)}: replace page with smallest count (assumes just brought in).
    \item Neither LFU nor MFU common: expensive, don't approximate OPT well.
\end{itemize}

\subsection{Page-buffering Algorithms}
\begin{itemize}
    \item Used in addition to replacement algorithms.
    \item Pool of free frames:
    \begin{itemize}
        \item On page fault, desired page read into free frame from pool before victim written out.
        \item Process restarts faster. Victim frame added to pool after write-out.
    \end{itemize}
    \item List of modified pages:
    \begin{itemize}
        \item When paging device idle, modified page written to secondary storage; modify bit reset.
        \item Increases probability of clean page for replacement (no write-out needed).
    \end{itemize}
    \item Pool of free frames remembering old page:
    \begin{itemize}
        \item If old page needed before frame reused, can be reused directly from pool (no I/O).
        \item Check free-frame pool first on page fault.
    \end{itemize}
    \item UNIX uses this with second-chance algorithm.
\end{itemize}

\subsection{Applications and Page Replacement}
\begin{itemize}
    \item Some applications (e.g., databases) perform worse with OS virtual memory buffering.
    \item Applications understand their memory/storage use better than general-purpose OS algorithms.
    \item Double buffering if OS and application both buffer I/O.
    \item Data warehouses: sequential reads, then computations/writes. LRU removes old pages, but older pages might be read again. MFU could be more efficient.
    \item Some OS allow special programs to use secondary storage as large sequential array of logical blocks $\rightarrow$ \textbf{raw disk}.
    \item \textbf{Raw I/O}: bypasses file-system services (demand paging, locking, prefetching, allocation, names, directories).
    \item Raw partitions efficient for specific apps, but most apps better with regular file-system services.
\end{itemize}

\subsection*{Section glossary}
\begin{tabular}{p{0.3\textwidth}p{0.7\textwidth}}
    \toprule
    \textbf{Term} & \textbf{Definition} \\
    \midrule
    \textbf{over-allocating} & Providing access to more resources than physically available; allocating more virtual memory than physical memory. \\
    \textbf{page replacement} & Selection of a physical memory frame to be replaced when a new page is allocated. \\
    \textbf{victim frame} & Frame selected by page-replacement algorithm to be replaced. \\
    \textbf{modify bit} & MMU bit indicating a frame has been modified (must be saved before replacement). \\
    \textbf{dirty bit} & MMU bit indicating a frame has been modified (must be saved before replacement). \\
    \textbf{frame-allocation algorithm} & OS algorithm for allocating frames among all demands. \\
    \textbf{page-replacement algorithm} & Algorithm choosing which victim frame will be replaced by a new data frame. \\
    \textbf{reference string} & Trace of accesses to a resource; list of pages accessed over time. \\
    \textbf{Belady's anomaly} & Page-fault rate may increase as allocated frames increase for some algorithms. \\
    \textbf{optimal page-replacement algorithm} & Algorithm with lowest page-fault rate, never suffers Belady's anomaly. \\
    \textbf{least recently used (LRU)} & Selects item used least recently; in memory, page not accessed in longest time. \\
    \textbf{stack algorithm} & Class of page-replacement algorithms that do not suffer from Belady's anomaly. \\
    \textbf{reference bit} & MMU bit indicating a page has been referenced. \\
    \textbf{second-chance page-replacement algorithm} & FIFO algorithm; if reference bit set, clear bit and don't replace. \\
    \textbf{clock} & Circular queue in second-chance algorithm containing possible victim frames. \\
    \textbf{least frequently used (LFU)} & Selects item used least frequently; in virtual memory, page with lowest access count. \\
    \textbf{most frequently used (MFU)} & Selects item used most frequently; in virtual memory, page with highest access count. \\
    \textbf{raw disk} & Direct access to secondary storage as array of blocks, no file system. \\
    \bottomrule
\end{tabular}
