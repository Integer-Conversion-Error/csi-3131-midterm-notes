\section{Thrashing}
\begin{itemize}
    \item Process without "enough" frames (minimum needed for working set) $\rightarrow$ quickly page-faults.
    \item Replaces page needed immediately $\rightarrow$ faults again and again.
    \item High paging activity called \textbf{thrashing}.
    \item Thrashing: spending more time paging than executing.
    \item Results in severe performance problems.
\end{itemize}

\subsection{Cause of thrashing}
\begin{itemize}
    \item Scenario: OS monitors CPU utilization. Low CPU utilization $\rightarrow$ increase multiprogramming (new process).
    \item Global page-replacement algorithm used (replaces pages without regard to process).
    \item Process needs more frames $\rightarrow$ starts faulting, takes frames from other processes.
    \item Other processes fault $\rightarrow$ take frames from others.
    \item Faulting processes use paging device $\rightarrow$ ready queue empties.
    \item Processes wait for paging device $\rightarrow$ CPU utilization decreases.
    \item CPU scheduler sees decreasing CPU utilization $\rightarrow$ \textbf{increases} multiprogramming.
    \item New process takes frames $\rightarrow$ more page faults, longer paging device queue.
    \item CPU utilization drops further $\rightarrow$ CPU scheduler increases multiprogramming more.
    \item Thrashing occurs $\rightarrow$ system throughput plunges.
    \item Page-fault rate increases tremendously $\rightarrow$ effective memory-access time increases.
    \item No work done, processes spend all time paging.
    \item Illustrated in Figure 10.6.1: CPU utilization vs. degree of multiprogramming.
    \begin{itemize}
        \item Multiprogramming increases $\rightarrow$ CPU utilization increases (slower) until max.
        \item Further increase $\rightarrow$ thrashing, CPU utilization drops sharply.
        \item To stop thrashing: \textbf{decrease} degree of multiprogramming.
    \end{itemize}
    \item Limit thrashing effects: use \textbf{local replacement algorithm} (or \textbf{priority replacement algorithm}).
    \begin{itemize}
        \item Local replacement: process selects only from its own frames.
        \item Thrashing process cannot steal frames from others.
        \item Problem not entirely solved: thrashing processes queue for paging device $\rightarrow$ increased average service time for page fault $\rightarrow$ increased effective access time for all processes.
    \end{itemize}
    \item To prevent thrashing: provide process with enough frames.
    \item How many frames needed? Look at frames actually used $\rightarrow$ \textbf{locality model}.
    \item \textbf{Locality model}: process moves from locality to locality during execution.
    \begin{itemize}
        \item Locality: set of pages actively used together.
        \item Running program: several overlapping localities.
        \item Example: function call $\rightarrow$ new locality (function instructions, local variables, global variables subset).
        \item Exit function $\rightarrow$ leave locality.
        \item Localities defined by program structure and data structures.
        \item Principle behind caching: accesses are patterned, not random.
    \end{itemize}
    \item Allocate enough frames for current locality: faults until pages in memory, then no faults until locality changes.
    \item Not enough frames for locality $\rightarrow$ thrashing.
\end{itemize}

\subsection{Working-set model}
\begin{itemize}
    \item Based on locality assumption.
    \item Uses parameter $\Delta$ to define \textbf{working-set window}.
    \item Examine most recent $\Delta$ page references.
    \item Set of pages in most recent $\Delta$ references = \textbf{working set} (Figure 10.6.3).
    \item Page in active use $\rightarrow$ in working set.
    \item No longer used $\rightarrow$ drops from working set $\Delta$ time units after last reference.
    \item Working set: approximation of program's locality.
    \item Example (Figure 10.6.3): $\Delta = 10$ references.
    \begin{itemize}
        \item Working set at $t_1$: \{1, 2, 5, 6, 7\}.
        \item Working set at $t_2$: \{3, 4\}.
    \end{itemize}
    \item Accuracy of working set depends on $\Delta$ selection.
    \begin{itemize}
        \item $\Delta$ too small: won't encompass entire locality.
        \item $\Delta$ too large: may overlap several localities.
        \item Extreme: $\Delta$ infinite $\rightarrow$ working set is all pages touched during execution.
    \end{itemize}
    \item Most important property: working-set size.
    \item Compute working-set size, $WSS_i$, for each process.
    \item Total demand for frames: $D = \sum WSS_i$.
    \item If $D > m$ (total available frames) $\rightarrow$ thrashing (some processes lack frames).
    \item Working-set model usage:
    \begin{itemize}
        \item OS monitors working set of each process.
        \item Allocates enough frames for its working-set size.
        \item Enough extra frames $\rightarrow$ new process initiated.
        \item Sum of working-set sizes exceeds available frames $\rightarrow$ OS suspends a process.
        \item Suspended process's pages swapped out, frames reallocated. Restarted later.
    \end{itemize}
    \item Prevents thrashing, keeps multiprogramming high, optimizes CPU utilization.
    \item Difficulty: tracking moving working-set window.
    \item Approximation: fixed-interval timer interrupt + reference bit.
    \begin{itemize}
        \item Example: $\Delta = 10,000$ references, interrupt every 5,000 references.
        \item Timer interrupt: copy and clear reference bits.
        \item Page fault: examine current reference bit and two in-memory bits.
        \item Used within last 10,000-15,000 references $\rightarrow$ at least one bit on $\rightarrow$ in working set.
        \item Not entirely accurate (cannot tell exact reference time within interval).
        \item Reduce uncertainty: increase history bits, interrupt frequency (higher cost).
    \end{itemize}
\end{itemize}

\subsection{Page-fault frequency}
\begin{itemize}
    \item Working-set model successful, useful for prepaging, but clumsy for thrashing control.
    \item \textbf{Page-fault frequency} (\textbf{PFF}) strategy: more direct.
    \item Problem: prevent thrashing (high page-fault rate).
    \item Control page-fault rate:
    \begin{itemize}
        \item Too high $\rightarrow$ process needs more frames.
        \item Too low $\rightarrow$ process may have too many frames.
    \end{itemize}
    \item Establish upper and lower bounds on desired page-fault rate (Figure 10.6.4).
    \begin{itemize}
        \item Actual PFF exceeds upper limit $\rightarrow$ allocate another frame.
        \item Actual PFF falls below lower limit $\rightarrow$ remove a frame.
    \end{itemize}
    \item Directly measure and control PFF to prevent thrashing.
    \item If PFF increases and no free frames: select process, swap out to backing store. Freed frames distributed to high-PFF processes.
\end{itemize}

\subsection{Current practice}
\begin{itemize}
    \item Thrashing and swapping: high performance impact.
    \item Best practice: include enough physical memory to avoid thrashing/swapping.
    \item Provides best user experience (smartphones to large servers).
\end{itemize}

\vspace{1em}
\subsection*{Section glossary}
\begin{tabular}{p{0.3\textwidth}p{0.7\textwidth}}
    \toprule
    \rowcolor{gray!20} \textbf{Term} & \textbf{Definition} \\
    \midrule
    \textbf{thrashing} & High rate of paging memory; occurs when insufficient physical memory to meet virtual memory demand. \\
    \textbf{local replacement algorithm} & Page replacement algorithm that avoids thrashing by not allowing a process to steal frames from other processes. \\
    \textbf{priority replacement algorithm} & Page replacement algorithm that avoids thrashing by not allowing a process to steal frames from other processes. \\
    \textbf{locality model} & Model for page replacement based on the working-set strategy. \\
    \textbf{working-set model} & Memory access model based on tracking the set of most recently accessed pages. \\
    \textbf{working-set window} & Limited set of most recently accessed pages (a "window" view of the entire set of accessed pages). \\
    \textbf{working set} & The set of pages in the most recent page references. \\
    \textbf{page-fault frequency} & The frequency of page faults. \\
    \bottomrule
\end{tabular}
