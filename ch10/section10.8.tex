\section{Allocating kernel memory}
\begin{itemize}
    \item User-mode process requests memory $\rightarrow$ pages allocated from kernel's free page frame list.
    \begin{itemize}
        \item List populated by page-replacement algorithms (e.g., Section 10.4).
        \item Free pages scattered throughout physical memory.
        \item Single byte request $\rightarrow$ entire page frame granted $\rightarrow$ internal fragmentation.
    \end{itemize}
    \item Kernel memory often allocated from a different free-memory pool. Reasons:
    \begin{enumerate}
        \item Kernel requests varying data structure sizes, some less than a page.
        \begin{itemize}
            \item Must use memory conservatively, minimize fragmentation waste.
            \item Many OS do not subject kernel code/data to paging.
        \end{itemize}
        \item User-mode pages don't need contiguous physical memory.
        \begin{itemize}
            \item Hardware devices interact directly with physical memory (no virtual memory interface).
            \item May require physically contiguous pages.
        \end{itemize}
    \end{enumerate}
    \item Strategies for managing kernel free memory: "buddy system" and "slab allocation".
\end{itemize}

\subsection{Buddy system}
\begin{itemize}
    \item Allocates memory from fixed-size segment of physically contiguous pages.
    \item Uses a \textbf{power-of-2 allocator}: satisfies requests in units sized as a power of 2 (4 KB, 8 KB, 16 KB, etc.).
    \item Request not appropriately sized $\rightarrow$ rounded up to next highest power of 2.
    \begin{itemize}
        \item Example: 11 KB request $\rightarrow$ satisfied with 16-KB segment.
    \end{itemize}
    \item Example (Figure 10.8.1):
    \begin{itemize}
        \item Initial segment: 256 KB. Kernel requests 21 KB.
        \item Segment divided into two \textbf{buddies} ($A_L$ and $A_R$), each 128 KB.
        \item One buddy ($A_L$) divided into two 64-KB buddies ($B_L$ and $B_R$).
        \item Next-highest power of 2 for 21 KB is 32 KB.
        \item One 64-KB buddy ($B_L$) divided into two 32-KB buddies ($C_L$ and $C_R$).
        \item One 32-KB buddy ($C_L$) used for 21-KB request.
    \end{itemize}
    \item Advantage: quickly combine adjacent buddies to form larger segments using \textbf{coalescing}.
    \begin{itemize}
        \item Example: kernel releases $C_L$ unit $\rightarrow$ coalesce $C_L$ and $C_R$ into 64-KB segment ($B_L$).
        \item $B_L$ can coalesce with $B_R$ to form 128-KB segment ($A_L$).
        \item Can eventually form original 256-KB segment.
    \end{itemize}
    \item Drawback: rounding up to next highest power of 2 causes internal fragmentation.
    \begin{itemize}
        \item Example: 33-KB request $\rightarrow$ 64-KB segment allocated.
        \item Cannot guarantee less than 50\% of allocated unit wasted.
    \end{itemize}
\end{itemize}

\subsection{Slab allocation}
\begin{itemize}
    \item Second strategy for kernel memory allocation.
    \item A \textbf{slab}: one or more physically contiguous pages.
    \item A \textbf{cache}: one or more slabs.
    \item Single cache for each unique kernel data structure (e.g., process descriptors, file objects, semaphores).
    \item Each cache populated with \textbf{objects} (instantiations of kernel data structure).
    \item Relationship (Figure 10.8.2):
    \begin{itemize}
        \item Two 3 KB kernel objects, three 7 KB objects, each in separate cache, linked to slab.
    \end{itemize}
    \item Algorithm uses caches to store kernel objects.
    \begin{itemize}
        \item Cache created $\rightarrow$ objects (initially \texttt{free}) allocated to cache.
        \item Number of objects depends on slab size (e.g., 12-KB slab $\rightarrow$ six 2-KB objects).
        \item New object needed $\rightarrow$ allocator assigns any \texttt{free} object from cache. Object marked \texttt{used}.
    \end{itemize}
    \item Scenario: kernel requests memory for process descriptor (\texttt{struct task\_struct}, ~1.7 KB).
    \begin{itemize}
        \item Cache fulfills request with pre-allocated, free \texttt{struct task\_struct} object.
    \end{itemize}
    \item Slab states in Linux:
    \begin{enumerate}
        \item \textbf{Full}: All objects in slab \texttt{used}.
        \item \textbf{Empty}: All objects in slab \texttt{free}.
        \item \textbf{Partial}: Slab has both \texttt{used} and \texttt{free} objects.
    \end{enumerate}
    \item Slab allocator request satisfaction:
    \begin{itemize}
        \item First: free object in a partial slab.
        \item If none: free object from an empty slab.
        \item If no empty slabs: new slab allocated from contiguous physical pages, assigned to cache; object memory allocated from new slab.
    \end{itemize}
    \item Two main benefits of slab allocator:
    \begin{enumerate}
        \item No memory wasted due to fragmentation.
        \begin{itemize}
            \item Each kernel data structure has associated cache.
            \item Caches made of slabs divided into object-sized chunks.
            \item Kernel requests memory $\rightarrow$ exact amount returned.
        \end{itemize}
        \item Memory requests satisfied quickly.
        \begin{itemize}
            \item Effective for frequent object allocation/deallocation (common in kernel).
            \item Objects created in advance $\rightarrow$ quickly allocated from cache.
            \item Released objects marked free, returned to cache $\rightarrow$ immediately available.
        \end{itemize}
    \end{enumerate}
    \item History:
    \begin{itemize}
        \item First appeared in Solaris 2.4 kernel.
        \item Now used for some user-mode requests in Solaris.
        \item Linux (Version 2.2+): adopted slab allocator (referred to as SLAB).
    \end{itemize}
    \item Recent Linux kernel memory allocators: SLOB and SLUB.
    \begin{itemize}
        \item \textbf{SLOB allocator}: for systems with limited memory (e.g., embedded systems).
        \begin{itemize}
            \item Maintains three lists: \texttt{small} (<256 bytes), \texttt{medium} (<1,024 bytes), \texttt{large} (other objects < page size).
            \item Allocates from appropriate list using first-fit policy.
        \end{itemize}
        \item \textbf{SLUB allocator}: default for Linux kernel (Version 2.6.24+), replaced SLAB.
        \begin{itemize}
            \item Reduced SLAB overhead.
            \item Stores metadata in \texttt{page} structure (not with each slab).
            \item No per-CPU queues for objects (significant memory saving on multi-processor systems).
            \item Better performance with more processors.
        \end{itemize}
    \end{itemize}
\end{itemize}

\vspace{1em}
\subsection*{Section glossary}
\begin{tabular}{p{0.3\textwidth}p{0.7\textwidth}}
    \toprule
    \rowcolor{gray!20} \textbf{Term} & \textbf{Definition} \\
    \midrule
    \textbf{power-of-2 allocator} & Allocator in buddy system; satisfies memory requests in units sized as a power of 2. \\
    \textbf{buddies} & Pairs of equal size in buddy memory allocation. \\
    \textbf{coalescing} & Combining freed memory in adjacent buddies into larger segments. \\
    \textbf{slab allocation} & Memory allocation method; slab split into object-sized chunks, eliminating fragmentation. \\
    \textbf{slab} & Section of memory (one or more contiguous pages) used in slab allocation. \\
    \textbf{cache} & Temporary data copy for performance; in slab allocator, consists of one or more slabs. \\
    \textbf{object} & Instance of a class or data structure. \\
    \bottomrule
\end{tabular}
