\section{Overview of mass-storage structure}

\subsection{Introduction}
\begin{itemize}
    \item Computer systems must provide mass storage for permanently storing files and data.
    \item Modern computers use secondary storage: hard disks (HDDs) and nonvolatile memory (NVM) devices.
    \item Secondary storage devices vary:
    \begin{itemize}
        \item Transfer: character at a time vs. block of characters.
        \item Access: sequentially vs. randomly.
        \item Data transfer: synchronously vs. asynchronously.
        \item Usage: dedicated vs. shared.
        \item Read-only vs. read-write.
        \item Speed: slowest major component of computer.
    \end{itemize}
    \item OS needs to provide wide range of functionality for device control.
    \item Key OS I/O subsystem goals:
    \begin{itemize}
        \item Simplest interface possible to rest of system.
        \item Optimize I/O for maximum concurrency (devices are performance bottleneck).
    \end{itemize}
    \item Mass storage: nonvolatile storage system of a computer.
    \item Main mass-storage system: secondary storage (HDDs and NVM devices).
    \item Some systems: slower, larger tertiary storage (magnetic tape, optical disks, cloud storage).
    \item This chapter focuses on HDDs and NVM devices.
    \item Discuss physical structure, scheduling algorithms (maximize performance), device formatting, boot blocks, damaged blocks, swap space, RAID systems.
    \item General term: \textbf{non-volatile storage} (NVS) or "storage drives" for all types.
\item Things to learn:
    \begin{itemize}
        \item Describe physical structures of various secondary storage devices and effect of structure on uses.
        \item Explain performance characteristics of mass-storage devices.
        \item Evaluate I/O scheduling algorithms.
        \item Discuss operating-system services for mass storage, including RAID.
    \end{itemize}
\end{itemize}

\subsection{Hard disk drives}
\begin{itemize}
    \item Bulk of secondary storage: \textbf{hard disk drives} (\textit{HDDs}) and \textbf{nonvolatile memory} (\textit{NVM}) devices.
    \item This section: basic mechanisms, OS translation of physical properties to logical storage via address mapping.
    \item HDDs: conceptually simple.
    \item Each disk \textbf{platter}: flat circular shape (like a CD).
    \item Common platter diameters: 1.8 to 3.5 inches.
    \item Two surfaces of platter: covered with magnetic material.
    \item Information stored by recording magnetically, read by detecting magnetic pattern.
    \item Read-write head: "flies" just above each surface of every platter.
    \item Heads attached to a \textbf{disk arm}: moves all heads as a unit.
    \item Platter surface: logically divided into circular \textbf{tracks}.
    \item Tracks subdivided into \textbf{sectors}.
    \item Set of tracks at given arm position: \textbf{cylinder}.
    \item Thousands of concentric cylinders; hundreds of sectors per track.
    \item Each sector: fixed size, smallest unit of transfer.
    \item Sector size: commonly 512 bytes until ~2010, then migrating to 4KB.
    \item Storage capacity: gigabytes and terabytes.
    \item Disk drive motor: spins at high speed (60 to 250 times/sec, RPM).
    \item Common RPM: 5,400, 7,200, 10,000, 15,000.
    \item Some drives: power down when not in use, spin up on I/O request.
    \item Rotation speed relates to \textbf{transfer rate}: rate data flows between drive and computer.
    \item Another performance aspect: \textbf{positioning time} or \textbf{random-access time}.
    \item Positioning time parts:
    \begin{itemize}
        \item \textbf{seek time}: move disk arm to desired cylinder.
        \item \textbf{rotational latency}: desired sector rotates to disk head.
    \end{itemize}
    \item Typical disks: tens to hundreds of MB/sec transfer, several ms seek/rotational latency.
    \item Performance increase: DRAM buffers in drive controller.
    \item Disk head: flies on thin cushion (microns) of air/gas.
    \item Danger: head contact with disk surface (\textbf{head crash}).
    \item Head crash: normally irreparable, entire disk replaced, data lost (unless backed up/RAID protected).
    \item HDDs: sealed units.
    \item Some chassis: allow HDD removal without shutting down system (\textbf{removable}).
    \item Other removable media: CDs, DVDs, Blu-ray discs.
\end{itemize}

\subsection{Disk transfer rates}
\begin{itemize}
    \item Published performance numbers for disks: not same as real-world.
    \item Stated transfer rates: always higher than \textbf{effective transfer rates}.
    \item Transfer rate (bits read from magnetic media) vs. rate blocks delivered to OS.
\end{itemize}

\subsection{Nonvolatile memory devices}
\begin{itemize}
    \item NVM devices: growing importance, electrical rather than mechanical.
    \item Composed of: controller and flash NAND die semiconductor chips (store data).
    \item Other NVM technologies (DRAM with battery, 3D XPoint): less common, not discussed.
\end{itemize}

\subsubsection{Overview of nonvolatile memory devices}
\begin{itemize}
    \item Flash-memory-based NVM: often in disk-drive-like container, called \textbf{solid-state disk} (\textit{SSD}).
    \item Other forms: \textbf{USB drive} (thumb drive/flash drive), DRAM stick.
    \item Surface-mounted on motherboards (smartphones).
    \item All forms: act and treated similarly.
    \item NVM devices: more reliable (no moving parts), faster (no seek/rotational latency), consume less power than HDDs.
    \item Negative: more expensive per MB, less capacity than larger HDDs.
    \item Trend: NVM capacity increased faster, price dropped quicker; increasing use.
    \item SSDs: used in laptops for smaller, faster, energy-efficient systems.
    \item NVM devices faster than HDDs: standard bus interfaces can limit throughput.
    \item Some NVM devices: connect directly to system bus (PCIe).
    \item NVM technology: changing computer design (direct disk replacement, new cache tier).
    \item NAND semiconductors: storage/reliability challenges.
    \item Read/write in "page" increment (similar to sector).
    \item Data cannot be overwritten: NAND cells must be erased first.
    \item Erasure: "block" increment (several pages), much slower than read/write.
    \item NVM flash devices: many die, many datapaths; parallel operations possible.
    \item NAND semiconductors: deteriorate with every erase cycle (~100,000 program-erase cycles).
    \item NVM lifespan: measured in \textbf{Drive Writes Per Day} (\textbf{DWPD}).
    \item DWPD: how many times drive capacity can be written per day before failure.
    \item Example: 1 TB NAND drive, 5 DWPD rating $\implies$ 5 TB/day written for warranty period.
    \item Limitations: led to ameliorating algorithms (usually in NVM device controller, transparent to OS).
    \item OS reads/writes logical blocks; device manages.
    \item NVM performance variations: based on operating algorithms.
\end{itemize}

\subsubsection{NAND flash controller algorithms}
\begin{itemize}
    \item NAND semiconductors cannot be overwritten: pages contain invalid data.
    \item Example: file-system block written, then rewritten. Old data invalid, new data valid.
    \item Controller maintains \textbf{flash translation layer} (\textit{FTL}): maps physical pages to currently valid logical blocks.
    \item FTL tracks physical block state: which blocks contain only invalid pages (can be erased).
    \item Full SSD, pending write:
    \begin{itemize}
        \item If block contains no valid data: write waits for erase, then occurs.
        \item If no free blocks but individual pages hold invalid data: \textbf{garbage collection} occurs.
        \item Good data copied to other locations, freeing blocks for erase/writes.
    \end{itemize}
    \item To solve problem and improve write performance: \textbf{over-provisioning}.
    \item Over-provisioning: device sets aside pages (e.g., 20\% total) as always available write area.
    \item Blocks totally invalid by garbage collection/write operations: erased, placed in over-provisioning space or returned to free pool.
    \item Over-provisioning space: helps with \textbf{wear leveling}.
    \item \textbf{Wear leveling}: effort to select all NAND cells over time as write targets.
    \item Avoids premature media failure due to frequently erased blocks.
    \item Controller: uses algorithms to place data on less-erased blocks to level wear.
    \item Data protection: NVM devices provide error-correcting codes (calculated/stored with data, detect/correct errors).
    \item If page frequently has correctible errors: marked bad, not used in subsequent writes.
    \item Single NVM device: catastrophic failure possible (corrupts/fails read/write).
    \item Data recoverable: RAID protection used.
\end{itemize}

\subsection{Volatile memory}
\begin{itemize}
    \item DRAM: frequently used as mass-storage device.
    \item \textbf{RAM drives} (RAM disks): created by device drivers, carve out DRAM section, present as storage device.
    \item Can be used as raw block devices or with file systems.
    \item Purpose of RAM drives (despite volatility): allow user/programmer to place data in memory for temporary safekeeping using standard file operations.
    \item Useful for temporary files, sharing data.
    \item Found in all major OS: Linux (\texttt{/dev/ram}), MacOS (\texttt{diskutil}), Windows (third-party tools), Solaris/Linux (\texttt{/tmp} as "tmpfs" RAM drive at boot).
\end{itemize}

\subsection{Magnetic tapes}
\begin{itemize}
    \item Early secondary-storage medium.
    \item Nonvolatile, holds large data quantities.
    \item Access time: slow compared to main memory/drives.
    \item Random access: ~1000x slower than HDDs, ~100,000x slower than SSDs.
    \item Main uses: backup, infrequently used info, transferring info between systems.
    \item Tape: kept in spool, wound/rewound past read-write head.
    \item Moving to correct spot: can take minutes.
    \item Once positioned: tape drives read/write at speeds comparable to HDDs.
    \item Capacities: exceed several terabytes.
    \item Some tapes: built-in compression (double effective storage).
    \item Categorized by width (4, 8, 19mm; 1/4, 1/2 inch) or technology (LTO-6, SDLT).
\end{itemize}

\subsection{Secondary storage connection methods}
\begin{itemize}
    \item Secondary storage device: attached to computer by system bus or \textbf{I/O bus}.
    \item Bus types: \textbf{advanced technology attachment} (\textit{ATA}), \textbf{serial ATA} (\textbf{SATA}), \textbf{eSATA}, \textbf{serial attached SCSI} (\textit{SAS}), \textbf{universal serial bus} (\textit{USB}), \textbf{fibre channel} (\textit{FC}).
    \item Most common: SATA.
    \item NVM devices faster than HDDs: special fast interface \textbf{NVM express} (\textit{NVMe}).
    \item NVMe: directly connects device to system PCI bus (increases throughput, decreases latency).
    \item Data transfers on bus: by special electronic processors (\textbf{controllers} or \textbf{host-bus adapters} (\textit{HBA})).
    \item \textbf{Host controller}: controller at computer end of bus.
    \item \textbf{Device controller}: built into each storage device.
    \item Mass storage I/O operation: computer places command into host controller (memory-mapped I/O ports).
    \item Host controller sends command via messages to device controller.
    \item Device controller operates drive hardware.
    \item Device controllers: usually have built-in cache.
    \item Data transfer at drive: between cache and storage media.
    \item Data transfer to host: fast electronic speeds, between cache host DRAM via DMA.
\end{itemize}

\subsection{Address mapping}
\begin{itemize}
    \item Storage devices: addressed as large one-dimensional arrays of \textbf{logical blocks}.
    \item \textbf{Logical block}: smallest unit of transfer.
    \item Each logical block: maps to physical sector or semiconductor page.
    \item One-dimensional array of logical blocks: mapped onto sectors/pages of device.
    \item HDD mapping example: Sector 0 = first sector of first track on outermost cylinder. Proceeds through track, then rest of tracks on cylinder, then rest of cylinders (outermost to innermost).
    \item NVM mapping: tuple (chip, block, page) to array of logical blocks.
    \item \textbf{Logical block address} (\textbf{LBA}): easier for algorithms than sector, cylinder, head tuple or chip, block, page tuple.
    \item Converting LBA to old-style disk address (cylinder, track, sector): difficult in practice. Reasons:
    \begin{itemize}
        \item Defective sectors: mapping hides by substituting spare sectors (LBA sequential, physical changes).
        \item Number of sectors per track: not constant on some drives.
        \item Disk manufacturers manage LBA to physical address mapping internally (little relationship between LBA and physical sectors).
    \end{itemize}
    \item Algorithms dealing with HDDs: assume logical addresses relatively related to physical addresses (ascending logical $\implies$ ascending physical).
    \item \textbf{Constant linear velocity} (\textit{CLV}): bit density uniform per track.
    \item Farther track from center: greater length, more sectors.
    \item Outer zones to inner zones: sectors per track decreases.
    \item Outer zone tracks: typically 40\% more sectors than innermost.
    \item Drive increases rotation speed: head moves outer to inner tracks (keep same data rate under head).
    \item Used in CD-ROM and DVD-ROM drives.
    \item Alternatively: disk rotation speed constant.
    \item Bit density decreases inner to outer tracks (keep data rate constant, performance same).
    \item Used in hard disks: \textbf{constant angular velocity} (\textit{CAV}).
    \item Sectors per track and cylinders per disk: increasing with technology.
    \item Large disks: tens of thousands of cylinders.
    \item Other storage devices exist (shingled magnetic recording HDDs, combination NVM/HDD devices, volume managers).
    \item These devices: different characteristics, may need different caching/scheduling algorithms.
\end{itemize}

\vspace{1em}
\begin{tabular}{p{0.3\textwidth}p{0.7\textwidth}}
\toprule
\rowcolor{lightgray} \textbf{Term} & \textbf{Definition} \\
\midrule
\textbf{hard disk drive (HDD)} & Secondary storage device based on mechanical components (spinning magnetic media platters, moving read-write heads). \\
\textbf{nonvolatile memory (NVM)} & Persistent storage based on circuits and electric charges. \\
\textbf{platter} & HDD component with magnetic media layer for holding charges. \\
\textbf{disk arm} & HDD component holding read-write head, moves over cylinders of platters. \\
\textbf{track} & On HDD platter, medium under read-write head during platter rotation. \\
\textbf{sectors} & On HDD platter, fixed-size section of a track. \\
\textbf{cylinder} & On HDD, set of tracks under read-write heads on all platters in device. \\
\textbf{transfer rate} & Rate at which data flows. \\
\textbf{positioning time} & On HDD, time for read-write head to position over desired track. \\
\textbf{seek time} & On HDD, time for read-write head to position over desired cylinder. \\
\textbf{rotational latency} & On HDD, time for read-write head (once over desired cylinder) to access desired track. \\
\textbf{head crash} & On HDD, mechanical problem: read-write head touching platter. \\
\textbf{effective transfer rate} & Actual, measured transfer rate of data between two devices. \\
\textbf{solid-state disk} & Disk-drive-like storage device using flash-memory-based nonvolatile memory. \\
\textbf{USB drive} & Nonvolatile memory in form of device plugging into USB port. \\
\textbf{flash translation layer (FTL)} & For nonvolatile memory, table tracking currently valid blocks. \\
\textbf{garbage collection} & Recovery of space containing no-longer-valid data. \\
\textbf{over-provisioning} & In non-volatile memory, space set aside for data writes not counted in device free space. \\
\textbf{wear leveling} & In nonvolatile memory, effort to select all NAND cells over time as write targets to avoid premature media failure. \\
\textbf{RAM drives} & Sections of system's DRAM presented as secondary storage devices. \\
\textbf{magnetic tape} & Magnetic media storage device (tape spooled on reels, passing over read-write head); mostly for backups. \\
\textbf{I/O bus} & Physical connection of I/O device to computer system. \\
\textbf{advanced technology attachment (ATA)} & Older-generation I/O bus. \\
\textbf{eSATA} & Type of I/O bus. \\
\textbf{serial-attached SCSI (SAS)} & Common type of I/O bus. \\
\textbf{universal serial bus (USB)} & Type of I/O bus. \\
\textbf{fibre channel (FC)} & Type of storage I/O bus used in data centers to connect computers to storage arrays. \\
\textbf{NVM express (NVMe)} & High-speed I/O bus for NVM storage. \\
\textbf{controller} & Special processor managing I/O devices. \\
\textbf{host bus adapter (HBA)} & Device controller installed in host bus port to allow device connection to host. \\
\textbf{host controller} & I/O-managing processors within a computer (e.g., inside HBA). \\
\textbf{device controller} & I/O managing processor within a device. \\
\textbf{logical blocks} & Logical addresses used to access blocks on storage devices. \\
\textbf{constant linear velocity (CLV)} & Device-recording method: constant bit density per track by varying rotational speed. \\
\textbf{constant angular velocity (CAV)} & Device-recording method: medium spins at constant velocity, bit density decreases inner to outer tracks. \\
\bottomrule
\end{tabular}
