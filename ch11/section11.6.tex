\section{Swap-space management}

\begin{itemize}
    \item Swapping: moving entire processes between secondary storage and main memory (Section 9.5).
    \item Occurs when physical memory critically low, processes moved to swap space to free memory.
    \item Modern OS: very few implement swapping this way.
    \item Systems now combine swapping with virtual memory (Chapter Virtual Memory), swap pages, not entire processes.
    \item Terms "swapping" and "paging" used interchangeably.
    \item \textbf{Swap-space management}: low-level OS task.
    \item Virtual memory uses secondary storage as extension of main memory.
    \item Drive access much slower than memory access $\implies$ swap space significantly decreases system performance.
    \item Main goal for swap space design/implementation: provide best throughput for virtual memory system.
    \item Discuss: swap space use, location, management.
\end{itemize}

\subsection{Swap-space use}
\begin{itemize}
    \item Swap space used differently by OS, depending on memory-management algorithms.
    \item Swapping systems: may hold entire process image (code, data segments).
    \item Paging systems: store pages pushed out of main memory.
    \item Amount of swap space needed: varies (few MB to GB).
    \item Depends on: physical memory, virtual memory backing, virtual memory usage.
    \item Safer to overestimate than underestimate swap space.
    \item Running out of swap space: may abort processes or crash.
    \item Overestimation: wastes secondary storage space (no other harm).
    \item Some systems recommend swap space amount.
    \item Solaris: swap space = virtual memory exceeding pageable physical memory.
    \item Past Linux: swap space = double physical memory.
    \item Today's Linux: paging algorithms changed, considerably less swap space used.
    \item Some OS (Linux): allow multiple swap spaces (files, dedicated partitions).
    \item Multiple swap spaces: usually on separate storage devices.
    \item Purpose: spread I/O load from paging/swapping over system's I/O bandwidth.
\end{itemize}

\subsection{Swap-space location}
\begin{itemize}
    \item Swap space can reside in two places:
    \begin{itemize}
        \item Carved out of normal file system (large file).
        \item In a separate \textbf{raw partition}.
    \end{itemize}
    \item Swap space as file: normal file-system routines create, name, allocate space.
    \item Swap space in raw partition:
    \begin{itemize}
        \item No file system or directory structure.
        \item Separate swap-space storage manager allocates/deallocates blocks.
        \item Manager uses algorithms optimized for speed (not storage efficiency).
        \item Swap space accessed more frequently than file systems.
        \item Internal fragmentation may increase (acceptable trade-off, data life shorter).
        \item Fragmentation short-lived (reinitialized at boot time).
        \item Raw-partition approach: fixed amount of swap space during disk partitioning.
        \item Adding more swap space: repartitioning device (moving/destroying other partitions) or adding another swap space elsewhere.
    \end{itemize}
    \item Some OS: flexible, can swap in raw partitions and file-system space (e.g., Linux).
    \item Trade-off: convenience of file system allocation/management vs. performance of raw partitions.
\end{itemize}

\subsection{Swap-space management: an example}
\begin{itemize}
    \item Evolution of swapping/paging in UNIX systems:
    \item Traditional UNIX kernel: copied entire processes between contiguous disk regions and memory.
    \item Later UNIX: evolved to combination of swapping and paging (as paging hardware available).
    \item Solaris 1 (SunOS): changed standard UNIX methods for efficiency/tech developments.
    \item Process executes: text-segment pages (code) brought from file system, accessed in memory, thrown away if selected for pageout.
    \item More efficient to reread page from file system than write to swap and reread.
    \item Swap space used only as backing store for pages of \textbf{anonymous} memory.
    \item \textbf{Anonymous} memory: not backed by any file (stack, heap, uninitialized data of process).
    \item Later Solaris versions: biggest change $\implies$ allocates swap space only when page forced out of physical memory (not when virtual memory page first created).
    \item Scheme: better performance on modern computers (more physical memory, page less).
    \item Linux: similar to Solaris, swap space only for anonymous memory.
    \item Linux: allows one or more swap areas (swap file on regular file system or dedicated swap partition).
    \item Each swap area: series of 4-KB \textbf{page slots} (hold swapped pages).
    \item Associated with each swap area: \textbf{swap map} (array of integer counters).
    \item Each counter: corresponds to page slot.
    \item Counter value 0: page slot available.
    \item Counter value $>$ 0: page slot occupied by swapped page.
    \item Counter value: indicates number of mappings to swapped page (e.g., 3 $\implies$ mapped to 3 processes if shared memory).
\end{itemize}

\vspace{1em}
\begin{tabular}{p{0.3\textwidth}p{0.7\textwidth}}
\toprule
\rowcolor{lightgray} \textbf{Term} & \textbf{Definition} \\
\midrule
\textbf{swap-space management} & Low-level OS task of managing space on secondary storage for swapping and paging. \\
\textbf{raw partition} & Partition within a storage device not containing a file system. \\
\textbf{anonymous memory} & Memory not associated with a file; dirty pages paged out are stored in swap space. \\
\textbf{page slot} & In Linux swap-space management, part of data structure tracking swap-space use. \\
\textbf{swap map} & In Linux swap-space management, part of data structure tracking swap-space use. \\
\bottomrule
\end{tabular}
