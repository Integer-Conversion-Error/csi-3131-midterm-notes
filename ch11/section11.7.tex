\section{Storage attachment}

\begin{itemize}
    \item Computers access secondary storage in three ways: host-attached, network-attached, and cloud storage.
\end{itemize}

\subsection{Host-attached storage}
\begin{itemize}
    \item \textbf{Host-attached storage}: accessed through local I/O ports (most common: SATA).
    \item Typical system: one or few SATA ports.
    \item More storage access: individual storage device, device in chassis, or multiple drives in chassis connected via USB, FireWire, or Thunderbolt.
    \item High-end workstations/servers: need more/shared storage, use sophisticated I/O architectures.
    \item \textbf{Fibre channel} (\textit{FC}): high-speed serial architecture (optical fiber or copper cable).
    \item FC benefits: large address space, switched communication, multiple hosts/storage devices attach to fabric (flexibility in I/O communication).
    \item Wide variety of devices suitable for host-attached storage: HDDs, NVM devices, CD/DVD/Blu-ray/tape drives, storage-area networks (\textbf{SANs}).
    \item I/O commands for host-attached storage: reads/writes of logical data blocks directed to specifically identified storage units (bus ID or target logical unit).
\end{itemize}

\subsection{Network-attached storage}
\begin{itemize}
    \item \textbf{Network-attached storage} (\textit{NAS}): provides access to storage across a network.
    \item NAS device: special-purpose storage system or general computer system providing storage to other hosts.
    \item Clients access NAS via remote-procedure-call (RPC) interface: NFS (UNIX/Linux), CIFS (Windows).
    \item RPCs carried via TCP/UDP over IP network (usually same LAN).
    \item NAS unit: usually implemented as storage array with RPC interface software.
    \item CIFS and NFS: provide locking features, allowing file sharing between hosts accessing NAS.
    \item NAS: convenient way for LAN computers to share storage pool (ease of naming/access).
    \item Downside: less efficient, lower performance than some direct-attached storage.
    \item \textbf{iSCSI}: latest network-attached storage protocol.
    \item Uses IP network protocol to carry SCSI protocol.
    \item Networks (not SCSI cables) used as interconnects between hosts and storage.
    \item Hosts treat storage as directly attached, even if distant.
    \item NFS/CIFS: present file system, send parts of files.
    \item iSCSI: sends logical blocks across network, client uses blocks directly or creates file system.
\end{itemize}

\subsection{Cloud storage}
\begin{itemize}
    \item \textbf{Cloud storage}: similar to network-attached storage, access across network.
    \item Unlike NAS: accessed over Internet/WAN to remote data center (storage for fee/free).
    \item Difference from NAS: how storage accessed/presented.
    \item NAS: accessed as another file system (CIFS/NFS) or raw block device (iSCSI). OS integrates these protocols.
    \item Cloud storage: API based; programs use APIs to access.
    \item Examples: Amazon S3, Dropbox, Microsoft OneDrive, Apple iCloud.
    \item Reason for APIs vs. existing protocols: WAN latency and failure scenarios.
    \item NAS protocols: designed for LANs (lower latency, less connectivity loss).
    \item LAN connection failure (NFS/CIFS): system might hang.
    \item Cloud storage: failures more likely, application pauses access until connectivity restored.
\end{itemize}

\subsection{Storage-area networks and storage arrays}
\begin{itemize}
    \item Drawback of NAS: storage I/O consumes data network bandwidth, increases network communication latency.
    \item Acute in large client-server installations: server-client communication competes with server-storage communication.
    \item \textbf{Storage-area network} (\textit{SAN}): private network (storage protocols, not networking protocols) connecting servers and storage units.
    \item SAN power: flexibility.
    \item Multiple hosts/storage arrays: attach to same SAN.
    \item Storage: dynamically allocated to hosts.
    \item Storage arrays: RAID protected or unprotected drives (\textbf{Just a Bunch of Disks} (\textbf{JBOD})).
    \item SAN switch: allows/prohibits access between hosts and storage.
    \item Example: host low on disk space, SAN configured to allocate more storage.
    \item SANs: clusters of servers share same storage, storage arrays include multiple direct host connections.
    \item SANs: typically more ports, higher cost than storage arrays.
    \item SAN connectivity: short distances, typically no routing. NAS can have more connected hosts than SAN.
    \item Storage array: purpose-built device (includes SAN/network ports or both).
    \item Contains: drives to store data, controller(s) to manage storage/access.
    \item Controllers: CPUs, memory, software (implement array features: network protocols, UIs, RAID, snapshots, replication, compression, deduplication, encryption).
    \item Some storage arrays include SSDs.
    \item Array with only SSDs: maximum performance, smaller capacity.
    \item Mix of SSDs/HDDs: array software/administrator selects best medium, or SSDs as cache/HDDs as bulk storage.
    \item FC: most common SAN interconnect.
    \item iSCSI: increasing use due to simplicity.
    \item Another SAN interconnect: \textbf{InfiniBand} (\textit{IB}).
    \item IB: special-purpose bus architecture, hardware/software support for high-speed interconnection networks (servers, storage units).
\end{itemize}

\vspace{1em}
\begin{tabular}{p{0.3\textwidth}p{0.7\textwidth}}
\toprule
\rowcolor{lightgray} \textbf{Term} & \textbf{Definition} \\
\midrule
\textbf{host-attached storage} & Storage accessed through local I/O ports (directly attached to a computer). \\
\textbf{fibre channel (FC)} & Storage I/O bus used in data centers to connect computers to storage arrays. \\
\textbf{network-attached storage (NAS)} & Storage accessed from a computer over a network. \\
\textbf{iSCSI} & Protocol using IP network to carry SCSI protocol for distant storage access. \\
\textbf{cloud storage} & Storage accessed over Internet/WAN to a remote, shared data center. \\
\textbf{storage-area network (SAN)} & Local-area storage network allowing multiple computers to connect to storage devices. \\
\textbf{Just a Bunch of Disks (JBOD)} & Unprotected drives in a storage array. \\
\textbf{InfiniBand (IB)} & High-speed network communications link for servers and storage units. \\
\bottomrule
\end{tabular}
