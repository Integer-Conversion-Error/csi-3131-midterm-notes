\section{Summary}
\begin{itemize}
    \item Hard disk drives (HDDs) and nonvolatile memory (NVM) devices: major secondary storage I/O units.
    \item Modern secondary storage: structured as large one-dimensional arrays of logical blocks.
    \item Drives attached to computer:
    \begin{enumerate}
        \item Through local I/O ports on host.
        \item Directly connected to motherboards.
        \item Through communications network or storage network connection.
    \end{enumerate}
    \item Requests for secondary storage I/O: generated by file system and virtual memory system.
    \item Each request: specifies device address as logical block number.
    \item Disk-scheduling algorithms: improve HDD effective bandwidth, average response time, variance in response time.
    \item Algorithms (SCAN, C-SCAN): improve via disk-queue ordering strategies.
    \item HDD performance: varies greatly with scheduling algorithms.
    \item Solid-state disks (SSDs): no moving parts, performance varies little among algorithms.
    \item SSDs: often use simple FCFS strategy.
    \item Data storage/transmission: complex, frequently result in errors.
    \item \textbf{Error detection}: attempts to spot problems, alert system for corrective action, avoid error propagation.
    \item \textbf{Error correction}: detects and repairs problems (depends on correction data, corruption amount).
    \item Storage devices: partitioned into one or more chunks of space.
    \item Each partition: can hold a volume or be part of a multidevice volume.
    \item File systems: created in volumes.
    \item OS manages storage device's blocks.
    \item New devices: typically pre-formatted.
    \item Device partitioned, file systems created.
    \item Boot blocks: allocated to store system's bootstrap program (if device contains OS).
    \item Block/page corrupted: system must lock out or logically replace with spare.
    \item Efficient swap space: key to good performance in some systems.
    \item Some systems: dedicate raw partition to swap space.
    \item Others: use file within file system.
    \item Still others: provide both options (user/admin decision).
    \item Large systems storage: secondary storage devices frequently made redundant via RAID algorithms.
    \item RAID algorithms: allow more than one drive for operation, allow continued operation/automatic recovery from drive failure.
    \item RAID algorithms: organized into different levels (each provides reliability/high transfer rates combination).
    \item \textbf{Object storage}: used for big data problems (e.g., Internet indexing, cloud photo storage).
    \item Objects: self-defining collections of data, addressed by object ID (not file name).
    \item Typically uses replication for data protection.
    \item Computes based on data: on systems where copy of data exists.
    \item Horizontally scalable: for vast capacity and easy expansion.
\end{itemize}
