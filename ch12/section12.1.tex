\section{Overview}

\subsection{Things to learn}
\begin{itemize}
    \item Explore OS I/O subsystem structure.
    \item Discuss I/O hardware principles and complexities.
    \item Explain I/O hardware and software performance aspects.
\end{itemize}

\subsection{Introduction}
\begin{itemize}
    \item Computer's main jobs: I/O and computing.
    \item Often, I/O is primary, computing incidental (e.g., browsing, editing).
    \item OS role in I/O: manage and control I/O operations and devices.
    \item Topics covered:
    \begin{itemize}
        \item I/O hardware basics: constraints on OS internal facilities.
        \item OS I/O services and application I/O interface.
        \item Bridging gap between hardware and application interfaces.
        \item UNIX System V STREAMS mechanism: dynamic driver code pipelines.
        \item I/O performance and OS design principles for improvement.
    \end{itemize}
\end{itemize}

\subsection{Overview}
\begin{itemize}
    \item Device control is major OS design concern.
    \item Wide variation in I/O device function/speed (mouse, hard disk, flash drive, tape robot) requires varied control methods.
    \item These methods form kernel's I/O subsystem, separating kernel from device management complexities.
    \item I/O device technology trends:
    \begin{itemize}
        \item Increasing standardization of software/hardware interfaces: helps incorporate new device generations.
        \item Increasingly broad variety of I/O devices: challenge to incorporate new, unlike devices.
    \end{itemize}
    \item Challenge met by hardware/software techniques:
    \begin{itemize}
        \item Basic I/O hardware elements (ports, buses, device controllers) accommodate diverse devices.
        \item Kernel structured with \textbf{device-driver} modules to encapsulate device details.
        \item Device drivers provide uniform device-access interface to I/O subsystem, similar to system calls for applications.
    \end{itemize}
\end{itemize}

\subsection{Section glossary}
\begin{tabular}{p{0.3\textwidth}p{0.7\textwidth}}
    \toprule
    \textbf{Term} & \textbf{Definition} \\
    \midrule
    \textbf{device driver} & OS component providing uniform access and managing I/O to various devices. \\
    \bottomrule
\end{tabular}
