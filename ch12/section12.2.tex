\section{I/O hardware}

\subsection{Device Communication}
\begin{itemize}
    \item Computers operate many device types: storage (disks, tapes), transmission (network, Bluetooth), human-interface (screen, keyboard, mouse, audio).
    \item Specialized devices exist (e.g., jet steering).
    \item Device communicates via signals over cable/air.
    \item Connection point: \textbf{port} (e.g., serial port).
    \item Devices sharing wires: \textbf{bus} (e.g., PCI bus).
    \item Bus: set of wires, rigidly defined protocol for messages (electrical voltages, timings).
    \item \textbf{Daisy chain}: devices connected in string (A to B, B to C), usually operates as a bus.
\end{itemize}

\subsection{Bus Structures}
\begin{itemize}
    \item Buses vary in signaling, speed, throughput, connection.
    \item Typical PC bus structure:
    \begin{itemize}
        \item \textbf{PCIe bus}: connects processor-memory subsystem to fast devices.
        \item \textbf{Expansion bus}: connects slow devices (keyboard, serial, USB ports).
        \item \textbf{Serial-attached SCSI (SAS)} bus: connects disks to SAS controller.
    \end{itemize}
    \item PCIe: flexible bus, data over "lanes".
    \begin{itemize}
        \item Lane: two signaling pairs (receive/transmit), full-duplex byte stream.
        \item Data packets: eight-bit byte format, simultaneous in both directions.
        \item Physical links: 1, 2, 4, 8, 12, 16, or 32 lanes (e.g., x8).
        \item Multiple "generations" (e.g., PCIe gen3 x8: 8 GB/s throughput).
    \end{itemize}
\end{itemize}

\subsection{Controllers}
\begin{itemize}
    \item \textbf{Controller}: electronics operating a port, bus, or device.
    \item Serial-port controller: simple, single chip.
    \item \textbf{Fibre channel (FC)} bus controller: complex, often separate circuit board (\textbf{host bus adapter (HBA)}).
    \item HBA: contains processor, microcode, private memory for FC protocol.
    \item Some devices have built-in controllers (e.g., disk drive board).
    \item Disk controller: implements disk-side protocol (SAS, SATA), microcode, processor for tasks (bad-sector mapping, prefetching, buffering, caching).
\end{itemize}

\subsection{Memory-mapped I/O}
\begin{itemize}
    \item Processor communicates with controller via registers (data, control).
    \item Methods:
    \begin{itemize}
        \item Special I/O instructions: transfer byte/word to I/O port address.
        \item \textbf{Memory-mapped I/O}: device-control registers mapped into processor address space.
        \item CPU uses standard data-transfer instructions to read/write registers at mapped locations.
    \end{itemize}
    \item Past PCs: mixed I/O instructions and memory-mapped I/O.
    \item Graphics controller: I/O ports for control, large memory-mapped region for screen contents.
    \item Writing millions of bytes to graphics memory faster than millions of I/O instructions.
    \item Trend: systems moved to memory-mapped I/O for efficiency.
    \item Today: most I/O via device controllers using memory-mapped I/O.
\end{itemize}

\subsection{I/O Device Control Registers}
\begin{itemize}
    \item Typically four registers: status, control, data-in, data-out.
    \item \textbf{Data-in register}: read by host for input.
    \item \textbf{Data-out register}: written by host to send output.
    \item \textbf{Status register}: bits read by host, indicate states (command complete, byte available, error).
    \item \textbf{Control register}: written by host to start command or change device mode (e.g., full/half-duplex, parity, word length, speed).
    \item Data registers: 1-4 bytes.
    \item Some controllers have FIFO chips: hold several bytes, expand capacity, buffer data bursts.
\end{itemize}

\subsection{Polling}
\begin{itemize}
    \item Host-controller interaction: handshaking.
    \item Example (2 bits for coordination):
    \begin{enumerate}
        \item Host reads `busy` bit until clear.
        \item Host sets `write` bit in `command` register, writes byte to `data-out` register.
        \item Host sets `command-ready` bit.
        \item Controller notices `command-ready` set, sets `busy` bit.
        \item Controller reads command, reads `data-out` byte, performs I/O.
        \item Controller clears `command-ready`, clears `error` bit, clears `busy` bit.
    \end{enumerate}
    \item Loop repeated for each byte.
    \item Step 1: host is \textbf{busy-waiting} or \textbf{polling} (repeatedly reading `status` register).
    \item Polling efficient if controller/device fast.
    \item Inefficient if wait is long and other CPU tasks pending.
    \item Risk of data loss if host waits too long (buffer overflow).
    \item Polling efficient for basic operation (3 CPU cycles).
    \item Inefficient if rarely finds device ready.
    \item Alternative: hardware controller notifies CPU when ready via \textbf{interrupt}.
\end{itemize}

\subsection{Interrupts}
\begin{itemize}
    \item Basic mechanism:
    \begin{itemize}
        \item CPU senses \textbf{interrupt-request line} after each instruction.
        \item Controller asserts signal: CPU saves state, jumps to \textbf{interrupt-handler routine} at fixed address.
        \item Handler determines cause, processes, restores state, executes `return from interrupt`.
        \item Device controller \textit{raises} interrupt, CPU \textit{catches} and \textit{dispatches}, handler \textit{clears}.
    \end{itemize}
    \item Modern systems: hundreds of interrupts/second.
    \item Sophisticated interrupt handling features needed:
    \begin{enumerate}
        \item Defer interrupt handling during critical processing.
        \item Efficient dispatch to proper handler (avoid polling all devices).
        \item Multilevel interrupts: distinguish high/low priority, respond urgently.
        \item Instruction to get OS attention directly (\textbf{traps}) for page faults, errors (division by zero).
    \end{enumerate}
    \item Features provided by CPU and \textbf{interrupt-controller hardware}.
    \item CPU has two interrupt request lines:
    \begin{itemize}
        \item \textbf{Nonmaskable interrupt}: for unrecoverable errors (e.g., memory errors).
        \item \textbf{Maskable}: can be turned off by CPU for critical sequences, used by device controllers.
    \end{itemize}
    \item Interrupt mechanism accepts \textbf{address}: selects handler from \textbf{interrupt vector}.
    \item Interrupt vector: table of memory addresses of specialized handlers.
    \item Purpose: reduce need for single handler to search all sources.
    \item More devices than vector elements: \textbf{interrupt chaining}.
    \item \textbf{Interrupt chaining}: each vector element points to head of handler list; handlers called until one services request.
    \item Compromise: avoids huge table, improves dispatch efficiency.
    \item Intel Pentium event-vector table:
    \begin{itemize}
        \item 0-31 (nonmaskable): error conditions, page faults, debugging.
        \item 32-255 (maskable): device-generated interrupts.
    \end{itemize}
    \item \textbf{Interrupt priority levels}: defer low-priority, allow high-priority preemption.
    \item OS interaction with interrupts:
    \begin{itemize}
        \item Boot time: probes buses, installs handlers into vector.
        \item During I/O: controllers raise interrupts (output complete, input available, failure).
        \item Handles \textbf{exceptions}: division by zero, protected memory access, privileged instruction from user mode.
        \item Events trigger urgent, self-contained routines.
    \end{itemize}
    \item Interrupt handling split:
    \begin{itemize}
        \item \textbf{First-level interrupt handler (FLIH)}: context switch, state storage, queuing.
        \item \textbf{Second-level interrupt handler (SLIH)}: performs actual handling.
    \end{itemize}
    \item Other uses:
    \begin{itemize}
        \item Virtual memory paging: page fault raises interrupt, suspends process, handler fetches page, schedules another process.
        \item System calls: library routines build data structure, execute \textbf{software interrupt} or \textbf{trap}.
        \item Trap: saves user state, switches to kernel mode, dispatches to kernel routine.
        \item Trap priority: low compared to device interrupts (less urgent).
        \item Kernel flow control: disk read completion (high-priority handler records status, clears interrupt, starts next I/O, raises low-priority interrupt; low-priority handler copies data, calls scheduler).
    \end{itemize}
    \item Threaded kernel architecture (e.g., Solaris): interrupt handlers as kernel threads, high scheduling priorities, preemption, concurrent execution on multiprocessor.
    \item Summary: interrupts handle asynchronous events, trap to supervisor mode. Use priority system. Device controllers, hardware faults, system calls raise interrupts. Efficient handling crucial for performance. Interrupt-driven I/O common, polling for high-throughput, sometimes combined.
\end{itemize}

\subsection{Direct memory access}
\begin{itemize}
    \item For large transfers (e.g., disk), \textbf{programmed I/O (PIO)} (CPU watching status bits, feeding data byte-by-byte) is wasteful.
    \item Offload work to \textbf{direct-memory-access (DMA)} controller.
    \item Initiate DMA: host writes DMA command block to memory (source, destination, byte count).
    \item Command block can be complex: list of non-contiguous sources/destinations (\textbf{scatter-gather}).
    \item CPU writes command block address to DMA controller, continues other work.
    \item DMA controller operates memory bus directly, performs transfers without main CPU.
    \item Standard component in modern computers.
    \item Target address in kernel space is straightforward.
    \item User space target: risk of modification during transfer.
    \item Data to user space: second copy operation (\textbf{double buffering}) from kernel to user memory, inefficient.
    \item Trend: OS moved to memory-mapping for direct I/O between devices and user address space.
    \item Handshaking (DMA controller and device controller): \textbf{DMA-request} and \textbf{DMA-acknowledge} wires.
    \item Device places signal on DMA-request when word available.
    \item DMA controller seizes memory bus, places address, signals DMA-acknowledge.
    \item Device receives DMA-acknowledge, transfers word, removes DMA-request.
    \item Transfer finished: DMA controller interrupts CPU.
    \item DMA controller seizing bus: CPU momentarily prevented from main memory access (\textbf{cycle stealing}).
    \item Cycle stealing slows CPU, but DMA generally improves total system performance.
    \item Some architectures use physical addresses for DMA, others \textbf{direct virtual memory access (DVMA)}.
    \item DVMA: uses virtual addresses, translates to physical. Can transfer between memory-mapped devices without CPU/main memory.
\end{itemize}

\subsection{I/O hardware summary}
\begin{itemize}
    \item Key concepts:
    \begin{itemize}
        \item Bus
        \item Controller
        \item I/O port and its registers
        \item Handshaking (host and device controller)
        \item Handshaking execution: polling or interrupts
        \item Offloading large transfers to DMA controller
    \end{itemize}
    \item Challenge for OS implementers: wide variety of devices, unique capabilities, control-bit definitions, protocols.
    \item Questions: how to attach new devices without OS rewrite? How to provide uniform I/O interface to applications?
\end{itemize}

\subsection{Section glossary}
\begin{tabular}{p{0.3\textwidth}p{0.7\textwidth}}
    \toprule
    \textbf{Term} & \textbf{Definition} \\
    \midrule
    \textbf{port} & Connection point for devices to attach to computers. \\
    \textbf{PHY} & Physical hardware component connecting to a network (OSI layer 1). \\
    \textbf{bus} & Communication system connecting computer components (CPU, I/O devices) for data/command transfer. \\
    \textbf{daisy chain} & Devices connected in a string (A to B, B to C). \\
    \textbf{PCIe bus} & Common computer I/O bus connecting CPU to I/O devices. \\
    \textbf{expansion bus} & Computer bus for connecting slow devices (e.g., keyboards). \\
    \textbf{serial-attached SCSI (SAS)} & Common type of I/O bus. \\
    \textbf{SAS} & Common type of I/O bus. \\
    \textbf{controller} & Special processor managing I/O devices. \\
    \textbf{fibre channel (FC)} & Storage I/O bus used in data centers to connect computers to storage arrays. \\
    \textbf{host bus adapter (HBA)} & Device controller installed in host bus port for device connection. \\
    \textbf{memory-mapped I/O} & Device I/O method where device-control registers map into processor address space. \\
    \textbf{data-in register} & Device I/O register for host to get input. \\
    \textbf{data-out register} & Device I/O register for host to send output. \\
    \textbf{status register} & Device I/O register indicating status. \\
    \textbf{control register} & Device I/O register for host to place commands. \\
    \textbf{busy waiting} & Thread/process continuously uses CPU while waiting; I/O loop reading status. \\
    \textbf{polling} & I/O loop where I/O thread continuously reads status waiting for I/O completion. \\
    \textbf{interrupt} & Hardware mechanism for device to notify CPU it needs attention. \\
    \textbf{interrupt-request line} & Hardware connection to CPU for signaling interrupts. \\
    \textbf{interrupt-handler routine} & OS routine called when interrupt signal received. \\
    \textbf{interrupt-controller hardware} & Computer hardware components for interrupt management. \\
    \textbf{nonmaskable interrupt} & Interrupt that cannot be delayed or blocked (e.g., unrecoverable memory error). \\
    \textbf{maskable} & Describes an interrupt that can be delayed or blocked. \\
    \textbf{interrupt vector} & OS data structure indexed by interrupt address, pointing to handlers. \\
    \textbf{interrupt chaining} & Mechanism where interrupt vector element points to list of handlers. \\
    \textbf{interrupt priority level} & Prioritization of interrupts for handling order. \\
    \textbf{exception} & Software-generated interrupt by error or user program request for OS service. \\
    \textbf{first-level interrupt handler (FLIH)} & Interrupt handler for reception and queuing of interrupts. \\
    \textbf{second-level interrupt handler (SLIH)} & Interrupt handler that actually handles interrupts. \\
    \textbf{software interrupt} & Software-generated interrupt; also called a trap. \\
    \textbf{trap} & Software interrupt. \\
    \textbf{programmed I/O (PIO)} & Data transfer method where CPU transfers data one byte at a time. \\
    \textbf{direct memory access (DMA)} & Operation allowing device controllers to transfer large data directly to/from main memory. \\
    \textbf{scatter-gather} & I/O method specifying multiple sources/destinations in one command. \\
    \textbf{double buffering} & Copying data twice (e.g., device to kernel, then kernel to process), or using two buffers. \\
    \textbf{cycle stealing} & Device (e.g., DMA controller) using bus, temporarily preventing CPU access. \\
    \textbf{direct virtual memory access (DVMA)} & DMA using virtual addresses as transfer sources/destinations. \\
    \bottomrule
\end{tabular}
