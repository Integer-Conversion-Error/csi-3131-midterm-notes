\section{Application I/O interface}

\subsection{Structuring Techniques and Interfaces}
\begin{itemize}
    \item Goal: treat I/O devices uniformly.
    \item Example: open file on disk without knowing disk type.
    \item Add new devices without OS disruption.
    \item Approach: abstraction, encapsulation, software layering.
    \item Abstract away device differences by identifying general kinds.
    \item Each kind accessed via standardized functions (\textbf{interface}).
    \item Differences encapsulated in kernel modules (\textbf{device drivers}).
    \item Device drivers: custom-tailored to specific devices, export standard interfaces.
    \item Purpose of device-driver layer: hide differences among device controllers from kernel I/O subsystem.
    \item I/O system calls encapsulate device behavior in generic classes, hiding hardware differences from applications.
    \item Benefits: simplifies OS development, allows hardware manufacturers to design compatible devices or write drivers for popular OS.
    \item Device may ship with multiple drivers (Windows, Linux, Mac).
\end{itemize}

\subsection{Device Characteristics}
\begin{itemize}
    \item Devices vary on many dimensions:
    \begin{itemize}
        \item \textbf{Character-stream or block}:
        \begin{itemize}
            \item Character-stream: transfers bytes one by one.
            \item Block: transfers a block of bytes as a unit.
        \end{itemize}
        \item \textbf{Sequential or random access}:
        \begin{itemize}
            \item Sequential: fixed order determined by device.
            \item Random-access: user seeks to any storage location.
        \end{itemize}
        \item \textbf{Synchronous or asynchronous}:
        \begin{itemize}
            \item Synchronous: predictable response times, coordinated.
            \item Asynchronous: irregular/unpredictable response times, not coordinated.
        \end{itemize}
        \item \textbf{Sharable or dedicated}:
        \begin{itemize}
            \item Sharable: used concurrently by several processes/threads.
            \item Dedicated: cannot be used concurrently.
        \end{itemize}
        \item \textbf{Speed of operation}: few bytes/sec to GB/sec.
        \item \textbf{Read-write, read only, write once}:
        \begin{itemize}
            \item Read-write: both input/output.
            \item Read only: only one data transfer direction.
            \item Write once: written once, then read-only.
        \end{itemize}
    \end{itemize}
\end{itemize}

\subsection{Application Access Conventions}
\begin{itemize}
    \item OS hides many differences, groups devices into conventional types.
    \item Major access conventions:
    \begin{itemize}
        \item Block I/O
        \item Character-stream I/O
        \item Memory-mapped file access
        \item Network sockets
    \end{itemize}
    \item Special system calls for devices like time-of-day clock, timer, graphical display, video, audio.
    \item Most OS have an \textbf{escape} or \textbf{back door}: transparently passes arbitrary commands to device driver.
    \item UNIX: `ioctl()` system call.
    \begin{itemize}
        \item Enables application to access any driver functionality without new system call.
        \item Three arguments: device identifier (major/minor numbers), command integer, pointer to data structure.
        \item Major number: device type, routes I/O requests to driver.
        \item Minor number: device instance.
    \end{itemize}
\end{itemize}

\subsection{Block and Character Devices}
\begin{itemize}
    \item \textbf{Block-device interface}: for disk drives and block-oriented devices.
    \item Commands: `read()`, `write()`, `seek()` (for random-access).
    \item Applications usually access via file-system interface.
    \item OS/special applications (e.g., DBMS) may prefer \textbf{raw I/O}: direct access as linear array of blocks, no file system.
    \item Raw I/O avoids extra buffering and redundant locking.
    \item Compromise: OS allows mode on file that disables buffering/locking (UNIX: \textbf{direct I/O}).
    \item Memory-mapped file access: layered on block-device drivers.
    \item Access disk storage via byte array in main memory.
    \item System call maps file to memory, returns virtual address.
    \item Data transfers only when needed (demand-paged virtual memory access), efficient.
    \item Convenient for programmers: simple read/write to memory.
    \item Used for kernel services (e.g., executing program, swap space access).
    \item \textbf{Character-stream interface}: for keyboards, mice, modems, printers, audio boards.
    \item Basic system calls: `get()`, `put()` one character.
    \item Libraries built on top: line-at-a-time access, buffering, editing.
    \item Convenient for spontaneous input/linear output streams.
\end{itemize}

\subsection{Network Devices}
\begin{itemize}
    \item Network I/O interface differs from disk I/O (`read()`-`write()`-`seek()`).
    \item Common interface: network \textbf{socket} (UNIX, Windows).
    \item Socket interface system calls:
    \begin{itemize}
        \item Create socket.
        \item Connect local socket to remote address.
        \item Listen for remote application connection.
        \item Send/receive packets.
        \item `select()`: manages set of sockets, returns info on ready sockets (packet waiting, room to send).
    \end{itemize}
    \item `select()` eliminates polling/busy waiting for network I/O.
    \item Facilitates distributed applications using any network hardware/protocol.
    \item Other IPC/network communication approaches: half-duplex pipes, full-duplex FIFOs, full-duplex STREAMS, message queues, sockets (UNIX).
\end{itemize}

\subsection{Clocks and Timers}
\begin{itemize}
    \item Hardware clocks/timers provide: current time, elapsed time, set timer for operation X at time T.
    \item Used heavily by OS and time-sensitive applications.
    \item System calls not standardized.
    \item \textbf{Programmable interval timer}: hardware to measure elapsed time, trigger operations.
    \item Set to wait, then generate interrupt (once or periodically).
    \item Used by scheduler (preempt process), disk I/O (flush dirty cache), network (cancel slow operations).
    \item OS provides user process interface for timers.
    \item Supports more timer requests than hardware channels by simulating virtual clocks.
    \item Kernel maintains sorted list of wanted interrupts, sets timer for earliest.
    \item Hardware clock: high-frequency counter.
    \item Can be read from device register (high-resolution clock), offers accurate time intervals.
    \item \textbf{High-performance event timer (HPET)}: modern PCs, 10-megahertz range.
    \item Comparators trigger interrupts when value matches HPET.
    \item Precision limited by timer resolution + virtual clock overhead.
    \item System clock drift corrected by protocols (e.g., \textbf{network time protocol (NTP)}).
\end{itemize}

\subsection{Nonblocking and Asynchronous I/O}
\begin{itemize}
    \item \textbf{Blocking} system call: calling thread suspended, moved to wait queue. Resumes after completion.
    \item Physical I/O actions: generally asynchronous (varying/unpredictable time).
    \item OS provides blocking calls for application interface (easier to write).
    \item \textbf{Nonblocking} I/O needed for some user processes (e.g., UI, video app).
    \item Overlap execution with I/O: multithreaded application (some threads block, others execute).
    \item Nonblocking I/O system calls: return quickly, indicate bytes transferred, don't halt thread.
    \item Asynchronous system call: returns immediately, doesn't wait for I/O. Thread continues.
    \item I/O completion communicated via variable setting, signal/software interrupt, or callback.
    \item Difference: nonblocking `read()` returns available data immediately (full, fewer, none); asynchronous `read()` requests full transfer to complete later.
    \item Asynchronous activities throughout modern OS (secondary storage, network I/O).
    \item OS buffers I/O, returns to application, completes request later (optimizes performance).
    \item Limit on buffer time (e.g., UNIX flushes every 30 seconds).
    \item Applications can request buffer flush.
    \item Data consistency: kernel reads from buffers before I/O, ensures data returned to reader.
    \item Multiple threads to same file may need locking protocols for consistency.
    \item I/O system calls can indicate synchronous execution for immediate requests.
    \item `select()` for network sockets: nonblocking behavior, specifies max waiting time (0 for polling).
    \item `select()` overhead: only checks if I/O possible, needs `read()`/`write()` after.
    \item Mach: blocking multiple-read call, returns on first completion.
\end{itemize}

\subsection{Vectored I/O}
\begin{itemize}
    \item \textbf{Vectored I/O}: one system call performs multiple I/O operations involving multiple locations.
    \item UNIX `readv`: accepts vector of multiple buffers, reads to/writes from vector.
    \item Also called \textbf{scatter-gather}.
    \item Benefits:
    \begin{itemize}
        \item Avoids context-switching and system-call overhead.
        \item No need to transfer data to larger buffer first.
        \item Some versions provide atomicity (all I/O done without interruption, avoids data corruption).
    \end{itemize}
    \item Programmers use scatter-gather for increased throughput, decreased overhead.
\end{itemize}

\subsection{Section glossary}
\begin{tabular}{p{0.3\textwidth}p{0.7\textwidth}}
    \toprule
    \textbf{Term} & \textbf{Definition} \\
    \midrule
    \textbf{escape} & Method of passing arbitrary commands when interface lacks standard method. \\
    \textbf{back door} & Method of passing arbitrary commands when interface lacks standard method. \\
    \textbf{block-device interface} & Interface for I/O to block devices. \\
    \textbf{raw I/O} & Direct access to secondary storage as array of blocks, no file system. \\
    \textbf{direct I/O} & Block I/O bypassing OS block features (buffering, locking). \\
    \textbf{character-stream interface} & Interface for I/O to character devices (e.g., keyboards). \\
    \textbf{socket} & Endpoint for communication; interface for network I/O. \\
    \textbf{programmable interval timer} & Hardware timer provided by many CPUs. \\
    \textbf{high-performance event timer (HPET)} & Hardware timer provided by some CPUs. \\
    \textbf{network time protocol (NTP)} & Network protocol for synchronizing system clocks. \\
    \textbf{blocking} & I/O request that suspends calling thread until I/O completes. \\
    \textbf{nonblocking} & I/O request that returns immediately with available data, allowing thread to continue. \\
    \textbf{Vectored I/O} & One system call performs multiple I/O operations involving multiple locations. \\
    \textbf{scatter-gather} & I/O method specifying multiple sources/destinations in one command structure. \\
    \bottomrule
\end{tabular}
