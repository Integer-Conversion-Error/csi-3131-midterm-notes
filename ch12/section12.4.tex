\section{Kernel I/O subsystem}

\subsection{I/O Services}
\begin{itemize}
    \item Kernels provide many I/O-related services.
    \item Services provided by kernel's I/O subsystem: scheduling, buffering, caching, spooling, device reservation, error handling.
    \item Builds on hardware and device-driver infrastructure.
    \item I/O subsystem responsible for protecting itself from errant processes and malicious users.
\end{itemize}

\subsection{I/O Scheduling}
\begin{itemize}
    \item Scheduling I/O requests: determine good execution order.
    \item Application system call order rarely best.
    \item Benefits: improve overall system performance, fair device access, reduce average waiting time.
    \item Example: disk arm near beginning, 3 apps request blocks (end, beginning, middle). OS can serve 2, 3, 1 to reduce arm travel.
    \item Implementation: maintain wait queue for each device.
    \item Blocking I/O system call: request placed on device queue.
    \item I/O scheduler rearranges queue for efficiency, average response time.
    \item OS may prioritize delay-sensitive requests (e.g., virtual memory subsystem over applications).
    \item Asynchronous I/O: kernel tracks many requests.
    \item Uses \textbf{device-status table}: entry for each I/O device.
    \item Table entry: device type, address, state (not functioning, idle, busy).
    \item If busy, request type and parameters stored in table entry.
    \item Scheduling I/O operations improves computer efficiency.
    \item Also uses storage space in main memory/storage hierarchy via buffering, caching, spooling.
\end{itemize}

\subsection{Buffering}
\begin{itemize}
    \item \textbf{Buffer}: memory area storing data transferred between devices or device/application.
    \item Reasons for buffering:
    \begin{enumerate}
        \item Cope with speed mismatch (producer/consumer).
        \item Example: network (slow) to SSD (fast). Buffer accumulates network bytes, then writes to SSD in single operation.
        \item \textbf{Double buffering}: two buffers used to decouple producer/consumer, relax timing. Network fills buffer 1, drive writes buffer 1; network fills buffer 2, drive writes buffer 2.
        \item Provide adaptations for different data-transfer sizes (e.g., network fragmentation/reassembly).
        \item Support \textbf{copy semantics} for application I/O.
        \item Copy semantics: data written to disk is version at time of system call, independent of subsequent application buffer changes.
        \item OS guarantees copy semantics: `write()` copies application data to kernel buffer before returning. Disk write from kernel buffer.
        \item Copying data between kernel/application space common despite overhead, due to clean semantics.
        \item More efficient: virtual memory mapping and copy-on-write page protection.
    \end{enumerate}
\end{itemize}

\subsection{Caching}
\begin{itemize}
    \item \textbf{Cache}: region of fast memory holding copies of data.
    \item Access to cached copy more efficient than original.
    \item Example: process instructions on disk, cached in physical memory, copied to CPU caches.
    \item Difference from buffer: buffer may hold only copy, cache holds copy of item residing elsewhere.
    \item Caching and buffering distinct, but memory region can serve both.
    \item OS uses main memory buffers for disk data (copy semantics, efficient scheduling).
    \item These buffers also act as cache: improve I/O efficiency for shared files or rapid write/reread.
    \item Kernel checks buffer cache for file I/O request: if available, avoids/defers physical disk I/O.
    \item Disk writes accumulated in buffer cache for seconds: allows efficient write schedules.
\end{itemize}

\subsection{Spooling and Device Reservation}
\begin{itemize}
    \item \textbf{Spool}: buffer holding output for device (e.g., printer) that cannot accept interleaved data streams.
    \item Printer serves one job at a time, but multiple apps may print concurrently.
    \item OS intercepts printer output: each application's output spooled to separate secondary storage file.
    \item Application finishes: spooling system queues spool file for printer output.
    \item Spooling system copies queued files to printer one at a time.
    \item Managed by system daemon process or in-kernel thread.
    \item OS provides control interface: display queue, remove jobs, suspend printing.
    \item Some devices (tape drives, printers) cannot multiplex I/O requests.
    \item Spooling coordinates concurrent output.
    \item Explicit coordination facilities: OS supports exclusive device access.
    \item VMS: process allocates idle device, deallocates when done.
    \item Other OS: limit one open file handle to such device.
    \item Many OS provide functions for processes to coordinate exclusive access (e.g., Windows `wait` for device object, `OpenFile()` parameter for access types).
    \item Applications responsible for avoiding deadlock.
\end{itemize}

\subsection{Error Handling}
\begin{itemize}
    \item Protected memory OS guards against hardware/application errors, prevents system failure from minor malfunctions.
    \item Devices/I/O transfers can fail: transient (network overloaded) or permanent (defective controller).
    \item OS compensates for transient failures: disk `read()` retry, network `send()` resend.
    \item Permanent failure of important component: OS unlikely to recover.
    \item I/O system call returns success/failure bit.
    \item UNIX: `errno` integer variable returns error code (hundreds of values: argument out of range, bad pointer, file not open).
    \item Some hardware provides detailed error info, but many OS don't convey to application.
    \item SCSI protocol error reporting:
    \begin{itemize}
        \item \textbf{Sense key}: general nature of failure (hardware error, illegal request).
        \item Additional sense code: category of failure (bad command parameter, self-test failure).
        \item Additional sense-code qualifier: more detail (which parameter, which subsystem failed).
    \end{itemize}
    \item Many SCSI devices maintain internal error-log pages (seldom requested).
\end{itemize}

\subsection{I/O Protection}
\begin{itemize}
    \item Errors related to protection.
    \item User process may disrupt system by illegal I/O instructions.
    \item Prevention: all I/O instructions are privileged.
    \item Users cannot issue I/O instructions directly; must use OS.
    \item User program executes system call for OS to perform I/O.
    \item OS (monitor mode) checks request validity, performs I/O, returns to user.
    \item Memory-mapped and I/O port memory locations protected from user access by memory-protection system.
    \item Kernel cannot deny all user access (e.g., graphics games need direct access to memory-mapped graphics memory).
    \item Kernel might provide locking mechanism to allocate graphics memory section to one process at a time.
\end{itemize}

\subsection{Kernel Data Structures}
\begin{itemize}
    \item Kernel keeps state info on I/O components via in-kernel data structures.
    \item Example: open-file table structure.
    \item Tracks network connections, character-device communications, other I/O activities.
    \item UNIX file-system access to various entities (user files, raw devices, process address spaces).
    \item `read()` semantics differ for each entity.
    \item UNIX encapsulates differences using object-oriented technique.
    \item Open-file record: contains dispatch table with pointers to appropriate routines based on file type.
    \item Some OS use message-passing for I/O (e.g., Windows).
    \item I/O request converted to message, sent through kernel to I/O manager, then device driver.
    \item Message contents may change.
    \item Message-passing overhead vs. procedural techniques (shared data structures).
    \item Benefits: simplifies I/O system structure/design, adds flexibility.
\end{itemize}

\subsection{Power Management}
\begin{itemize}
    \item Data centers: power costs, greenhouse gas emissions, heat generation (cooling uses twice as much electricity as powering equipment).
    \item OS role in power use:
    \begin{itemize}
        \item Cloud computing: adjust processing loads, evacuate user processes, idle/power off systems.
        \item Analyze load, power off components (CPUs, external I/O devices) if low and hardware-enabled.
        \item CPU cores suspended/resumed based on load (state saved/restored).
        \item Servers: disabling unneeded cores decreases electricity/cooling needs.
    \end{itemize}
    \item Mobile computing: power management high priority (maximize battery life).
    \item Android power management features:
    \begin{enumerate}
        \item Power collapse: deep sleep state, marginally more power than off, responds to external stimuli, quick wake-up.
        \item Achieved by powering off individual components (screen, speakers, I/O subsystem), CPU in lowest sleep state.
        \item Idle Android phone uses little power, wakes quickly for calls.
        \item Component-level power management: infrastructure understands component relationships and usage.
        \item Android builds device tree (physical-device topology).
        \item Each component associated with device driver, tracks usage (e.g., I/O pending to flash, open audio reference).
        \item If component unused, turned off. If all components on bus unused, bus off. If all components in device tree unused, system enters power collapse.
        \item Wakelocks: applications temporarily prevent power collapse.
        \item Applications acquire/release wakelocks. Kernel prevents power collapse while wakelock held.
        \item Example: Android Market holds wakelock during app update.
    \end{enumerate}
    \item Power management based on device management.
    \item Boot time: firmware analyzes hardware, creates device tree in RAM.
    \item Kernel uses device tree to load drivers, manage devices.
    \item Additional activities: hot-plug (add/subtract devices), device state management, power management.
    \item Modern computers use \textbf{advanced configuration and power interface (ACPI)} firmware.
    \item ACPI: industry standard, provides callable routines for kernel (device state discovery/management, error management, power management).
    \item Example: kernel calls device driver, which calls ACPI routines, which talk to device.
\end{itemize}

\subsection{Kernel I/O Subsystem Summary}
\begin{itemize}
    \item I/O subsystem coordinates extensive services for applications and kernel.
    \item Supervises:
    \begin{itemize}
        \item Management of name space for files/devices.
        \item Access control to files/devices.
        \item Operation control (e.g., modem cannot `seek()`).
        \item File-system space allocation.
        \item Device allocation.
        \item Buffering, caching, spooling.
        \item I/O scheduling.
        \item Device-status monitoring, error handling, failure recovery.
        \item Device-driver configuration and initialization.
        \item Power management of I/O devices.
    \end{itemize}
    \item Upper levels of I/O subsystem access devices via uniform interface from device drivers.
\end{itemize}

\subsection{Section glossary}
\begin{tabular}{p{0.3\textwidth}p{0.7\textwidth}}
    \toprule
    \textbf{Term} & \textbf{Definition} \\
    \midrule
    \textbf{device-status table} & Kernel data structure tracking status and queues of operations for devices. \\
    \textbf{buffer} & Memory area storing data being transferred. \\
    \textbf{double buffering} & Copying data twice or using two buffers to decouple producers/consumers. \\
    \textbf{copy semantics} & Meaning assigned to data copying (e.g., if data can be modified after write request). \\
    \textbf{cache} & Temporary copy of data in fast memory to improve performance. \\
    \textbf{spool} & Buffer holding output for device that cannot accept interleaved data streams. \\
    \textbf{sense key} & SCSI protocol info in status register indicating error. \\
    \textbf{advanced configuration and power interface (ACPI)} & Firmware managing hardware aspects (power, device info). \\
    \bottomrule
\end{tabular}
