\section{Transforming I/O requests to hardware operations}

\subsection{Connecting Application Requests to Hardware}
\begin{itemize}
    \item How OS connects application request to network wires or disk sector.
    \item Example: reading file from disk.
    \item Application refers to data by file name.
    \item File system maps file name through directories to space allocation.
    \item MS-DOS for FAT: name maps to number, indicates entry in file-access table, tells which disk blocks allocated.
    \item UNIX: name maps to inode number, inode contains space-allocation info.
    \item How is connection made from file name to disk controller (hardware port address or memory-mapped registers)?
\end{itemize}

\subsection{Device Naming and Lookup}
\begin{itemize}
    \item MS-DOS for FAT: first part of file name (before colon) identifies hardware device (e.g., `C:` for primary hard disk).
    \item `C:` mapped to specific port address via device table.
    \item Device name space separate from file-system name space (due to colon separator).
    \item Easy to associate extra functionality (e.g., spooling for printer files).
    \item UNIX: device name space incorporated in regular file-system name space.
    \item No clear separation of device portion in path name.
    \item UNIX uses \textbf{mount table}: associates path name prefixes with specific device names.
    \item Resolve path name: lookup in mount table for longest matching prefix.
    \item Mount table entry gives device name (also in file-system name space).
    \item Lookup device name: finds `<major, minor>` device number, not inode.
    \item Major device number: identifies device driver to handle I/O.
    \item Minor device number: passed to device driver to index into device table.
    \item Device-table entry: gives port address or memory-mapped address of device controller.
    \item Modern OS flexibility: multiple stages of lookup tables.
    \item Mechanisms passing requests between applications/drivers are general.
    \item New devices/drivers can be introduced without kernel recompilation.
    \item Some OS load device drivers on demand.
    \item Boot time: system probes buses, loads necessary drivers (immediately or on first request).
    \item Devices added after boot: detected by error, kernel inspects, loads driver dynamically.
    \item Dynamic loading/unloading: more complex kernel algorithms, device-structure locking, error handling.
\end{itemize}

\subsection{Life Cycle of a Blocking Read Request}
\begin{enumerate}
    \item Process issues blocking `read()` system call to file descriptor of opened file.
    \item Kernel system-call code checks parameters. If data in buffer cache, data returned, I/O completed.
    \item Else, physical I/O performed. Process removed from run queue, placed on wait queue for device. I/O request scheduled.
    \item I/O subsystem sends request to device driver (subroutine call or in-kernel message).
    \item Device driver allocates kernel buffer space, schedules I/O. Sends commands to device controller by writing to device-control registers.
    \item Device controller operates device hardware for data transfer.
    \item Driver may poll for status/data, or set up DMA transfer to kernel memory. DMA controller generates interrupt on transfer completion.
    \item Correct interrupt handler receives interrupt via interrupt-vector table, stores data, signals device driver, returns.
    \item Device driver receives signal, determines completed I/O request, status, signals kernel I/O subsystem.
    \item Kernel transfers data/return codes to requesting process's address space. Moves process from wait queue to ready queue.
    \item Moving process to ready queue unblocks it. Scheduler assigns CPU, process resumes at system call completion.
\end{enumerate}

\subsection{Section glossary}
\begin{tabular}{p{0.3\textwidth}p{0.7\textwidth}}
    \toprule
    \textbf{Term} & \textbf{Definition} \\
    \midrule
    \textbf{mount table} & In-memory data structure with info about each mounted volume, tracks file systems and access. \\
    \bottomrule
\end{tabular}
