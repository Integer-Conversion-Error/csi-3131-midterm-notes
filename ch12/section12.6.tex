\section{Streams}

\subsection{STREAMS Mechanism}
\begin{itemize}
    \item UNIX System V (and subsequent releases) has \textbf{STREAMS} mechanism.
    \item Enables dynamic assembly of driver code pipelines.
    \item Stream: full-duplex connection between device driver and user-level process.
    \item Components:
    \begin{itemize}
        \item \textbf{Stream head}: interfaces with user process.
        \item \textbf{Driver end}: controls the device.
        \item Zero or more \textbf{stream modules}: between stream head and driver end.
    \end{itemize}
    \item Each component: pair of queues (read queue, write queue).
    \item Data transfer: message passing between queues.
\end{itemize}

\subsection{Modules and Flow Control}
\begin{itemize}
    \item Modules provide STREAMS processing functionality.
    \item Modules are \textit{pushed} onto a stream using `ioctl()` system call.
    \item Example: open USB device (keyboard) via stream, push module for input editing.
    \item Message exchange between queues in adjacent modules.
    \item To prevent queue overflow: queue may support \textbf{flow control}.
    \item Without flow control: queue accepts all messages, immediately sends to adjacent queue without buffering.
    \item With flow control: buffers messages, does not accept messages without sufficient buffer space.
    \item Involves control message exchanges between adjacent module queues.
\end{itemize}

\subsection{User Process Interaction}
\begin{itemize}
    \item User process writes data to device: `write()` or `putmsg()` system call.
    \item `write()`: writes raw data to stream.
    \item `putmsg()`: allows user to specify a message.
    \item Stream head copies data into message, delivers to next module's queue.
    \item Copying continues to driver end, then device.
    \item User process reads data from stream head: `read()` or `getmsg()` system call.
    \item `read()`: stream head gets message, returns ordinary data (unstructured byte stream).
    \item `getmsg()`: message returned to process.
    \item STREAMS I/O is asynchronous (or nonblocking) except when communicating with stream head.
    \item Writing to stream: user process blocks (if next queue uses flow control) until room to copy message.
    \item Reading from stream: user process blocks until data available.
\end{itemize}

\subsection{Driver End and Benefits}
\begin{itemize}
    \item Driver end (like stream head/modules) has read/write queue.
    \item Driver end must respond to interrupts (e.g., frame ready from network).
    \item Unlike stream head (may block), driver end must handle all incoming data.
    \item Drivers must support flow control.
    \item If device buffer full: device typically drops incoming messages (e.g., network card).
    \item Benefit of STREAMS: framework for modular, incremental device drivers and network protocols.
    \item Modules reusable by different streams/devices (e.g., networking module for Ethernet and 802.11 wireless).
    \item Supports message boundaries and control info between modules (not just unstructured byte stream).
    \item Most UNIX variants support STREAMS; preferred for protocols/device drivers.
    \item Example: System V UNIX and Solaris implement socket mechanism using STREAMS.
\end{itemize}

\subsection{Section glossary}
\begin{tabular}{p{0.3\textwidth}p{0.7\textwidth}}
    \toprule
    \textbf{Term} & \textbf{Definition} \\
    \midrule
    \textbf{STREAMS} & UNIX I/O feature allowing dynamic assembly of driver code pipelines. \\
    \textbf{stream head} & Interface between STREAMS and user processes. \\
    \textbf{driver end} & Interface between STREAMS and the controlled device. \\
    \textbf{stream modules} & Modules of functionality loadable into a STREAM. \\
    \textbf{flow control} & Method to pause a sender of I/O; limits data flow rate. \\
    \bottomrule
\end{tabular}
