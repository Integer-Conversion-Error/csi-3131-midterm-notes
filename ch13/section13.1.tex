\section{File concept}

\subsection*{Things to learn}
\begin{itemize}
    \item Explain file system function.
    \item Describe file system interfaces.
    \item Discuss file system design tradeoffs: access methods, sharing, locking, directory structures.
    \item Explore file system protection.
\end{itemize}

\subsection*{Introduction}
\begin{itemize}
    \item \textbf{File}: collection of related information defined by its creator.
    \item OS maps files onto physical mass-storage devices.
    \item File system: describes how files map to physical devices, how accessed/manipulated.
    \item File systems designed for efficient access (physical storage can be slow).
    \item Other requirements: file sharing support, remote access.
    \item File system: most visible OS aspect for users.
    \item Provides mechanism for on-line storage/access to OS data/programs and user data.
    \item Consists of:
    \begin{itemize}
        \item Collection of files (storing related data).
        \item Directory structure (organizes, provides info about files).
    \end{itemize}
    \item Most file systems live on storage devices (nonvolatile, persistent).
\end{itemize}

\subsection*{File Concept}
\begin{itemize}
    \item Computers store info on various media (NVM, HDDs, tapes, optical disks).
    \item OS provides uniform logical view of stored info.
    \item OS abstracts physical properties to define logical storage unit: the \textbf{file}.
    \item Files mapped by OS onto physical devices.
    \item Storage devices usually nonvolatile, so contents persistent.
    \item File: named collection of related information recorded on secondary storage.
    \item User perspective: smallest allotment of logical secondary storage (data written only within a file).
    \item Commonly: files represent programs (source/object) and data.
    \item Data files: numeric, alphabetic, alphanumeric, binary.
    \item Files can be free form (text) or rigidly formatted.
    \item Generally: file is sequence of bits, bytes, lines, or records; meaning defined by creator/user.
    \item File concept is extremely general.
    \item Use stretched beyond original confines (e.g., UNIX `proc` file system uses file-system interfaces for system info access).
    \item Info in file defined by creator.
    \item File has defined structure, depends on type.
    \item \textbf{Text file}: sequence of characters organized into lines.
    \item \textbf{Source file}: sequence of functions (declarations, executable statements).
    \item \textbf{Executable file}: series of code sections loader can bring into memory and execute.
\end{itemize}

\subsection*{File Attributes}
\begin{itemize}
    \item File named for human users, referred by name (e.g., `example.c`).
    \item Some systems differentiate case, others not.
    \item File independent of creator process, user, system.
    \item Example: `example.c` created by one user, edited by another, copied to USB/email/network, still `example.c`.
    \item Second copy independent if no sharing/synchronization.
    \item File attributes vary by OS, typically include:
    \begin{itemize}
        \item \textbf{Name}: symbolic, human-readable.
        \item \textbf{Identifier}: unique tag (number), non-human-readable, identifies file within file system.
        \item \textbf{Type}: for systems supporting different file types.
        \item \textbf{Location}: pointer to device and file location on device.
        \item \textbf{Size}: current size (bytes, words, blocks), possibly max allowed.
        \item \textbf{Protection}: access-control info (read, write, execute).
        \item \textbf{Timestamps and user identification}: creation, last modification, last use (useful for protection, security, monitoring).
    \end{itemize}
    \item Newer file systems support \textbf{extended file attributes}: character encoding, file checksum.
    \item \textbf{File info window}: GUI view of file metadata (e.g., macOS).
    \item Info about all files kept in directory structure, on same device as files.
    \item Directory entry: file name, unique identifier (locates other attributes).
    \item Directory size can be large (MBs/GBs).
    \item Directories stored on device, brought into memory piecemeal.
\end{itemize}

\subsection*{File Operations}
\begin{itemize}
    \item File is an abstract data type.
    \item OS provides system calls: create, write, read, reposition, delete, truncate files.
    \item \textbf{Creating a file}:
    \begin{enumerate}
        \item Find space in file system.
        \item Make entry for new file in directory.
    \end{enumerate}
    \item \textbf{Opening a file}:
    \begin{itemize}
        \item All operations except create/delete require `open()` first.
        \item Returns file handle used as argument in other calls.
    \end{itemize}
    \item \textbf{Writing a file}:
    \begin{itemize}
        \item System call: open file handle, info to write.
        \item System searches directory for file location.
        \item System keeps \textbf{write pointer} to next write location (sequential).
        \item Write pointer updated after each write.
    \end{itemize}
    \item \textbf{Reading a file}:
    \begin{itemize}
        \item System call: file handle, memory location for next block.
        \item Directory searched for entry.
        \item System keeps \textbf{read pointer} to next read location (sequential).
        \item Read pointer updated after each read.
        \item Current operation location: per-process \textbf{current-file-position pointer} (shared by read/write).
    \end{itemize}
    \item \textbf{Repositioning within a file}:
    \begin{itemize}
        \item Current-file-position pointer repositioned to given value.
        \item No actual I/O involved.
        \item Also known as file \textbf{seek}.
    \end{itemize}
    \item \textbf{Deleting a file}:
    \begin{itemize}
        \item Search directory for named file.
        \item Release all file space for reuse.
        \item Erase/mark directory entry as free.
        \item \textbf{Hard links}: multiple names for same file; actual content deleted only when last link deleted.
    \end{itemize}
    \item \textbf{Truncating a file}:
    \begin{itemize}
        \item Erase contents but keep attributes.
        \item Reset file length to zero, release file space.
    \end{itemize}
    \item These 7 are minimal set. Other common: appending, renaming.
    \item Primitive operations combine for others (e.g., copy file).
    \item Operations to get/set file attributes (e.g., length, owner).
    \item To avoid constant directory searching: `open()` system call before first use.
    \item OS keeps \textbf{open-file table}: info about all open files.
    \item File specified by index into table (no searching).
    \item File closed: OS removes entry from open-file table, releases locks.
    \item `create()` and `delete()` work with closed files.
    \item Some systems implicitly open/close files (job termination).
    \item Most systems require explicit `open()`/`close()`.
    \item `open()`: takes file name, searches directory, copies entry to open-file table.
    \item `open()` accepts access-mode info (create, read-only, read-write, append-only).
    \item Mode checked against file permissions. If allowed, file opened.
    \item `open()` returns pointer to open-file table entry; used in all I/O ops.
    \item `open()`/`close()` complicated with simultaneous opens by multiple processes.
    \item OS uses two levels of internal tables:
    \begin{itemize}
        \item Per-process table: tracks files process has open, current file pointer, access rights, accounting.
        \item System-wide open-file table: process-independent info (disk location, access dates, size).
    \end{itemize}
    \item File opened by another process: new entry in process's table points to system-wide entry.
    \item Open-file table has \textbf{open count}: number of processes with file open.
    \item `close()` decreases count; when zero, entry removed.
    \item File locks: allow one process to lock file/sections, prevent others.
    \item Useful for shared files (e.g., system log).
    \item \textbf{Shared lock}: multiple processes acquire concurrently (like reader lock).
    \item \textbf{Exclusive lock}: only one process at a time (like writer lock).
    \item Not all OS provide both types.
    \item \textbf{Mandatory} vs. \textbf{advisory} file-locking mechanisms.
    \item Mandatory: OS prevents other processes from accessing locked file (e.g., Windows).
    \item Advisory: OS does not prevent access; text editor must manually acquire lock (e.g., UNIX).
    \item Mandatory: OS ensures locking integrity. Advisory: developers ensure locks.
    \item File locks require same precautions as process synchronization (e.g., hold exclusive locks only during access, avoid deadlock).
\end{itemize}

\subsection*{File Types}
\begin{itemize}
    \item OS should recognize/support file types for reasonable operations.
    \item Example: prevent outputting binary-object program as garbage.
    \item Common technique: include type as part of file name (name.extension).
    \item User/OS can tell type from name (e.g., `resume.docx`, `server.c`).
    \item OS uses extension to indicate type and allowed operations (e.g., `.com`, `.exe`, `.sh` for execution).
    \item `.sh` is a \textbf{shell script} (ASCII commands).
    \item Application programs use extensions (e.g., Java compilers expect `.java`).
    \item Extensions not always required; user can omit, app looks for expected extension.
    \item Extensions are "hints" to applications, not OS-enforced.
    \item macOS: each file has type (`.app`), creator attribute (program that created it).
    \item Creator attribute set by OS during `create()` call, enforced.
    \item Example: word processor file has word processor as creator; double-click opens app, loads file.
    \item UNIX: \textbf{magic number} at beginning of some binary files indicates data type (e.g., image format).
    \item Text magic number for text files (e.g., shell language).
    \item Not all files have magic numbers; system features not solely based on this.
    \item UNIX does not record creating program.
    \item UNIX allows file-name-extension hints, but not enforced/depended on by OS; aid users.
\item File types indicate internal structure.
\item Source/object files: structures match expectations of programs reading them.
\item Certain files must conform to OS-understood structure (e.g., executable file structure for loading/running).
\item Some OS extend this: system-supported file structures with special operations.
\item Disadvantage of OS supporting multiple file structures: large, cumbersome OS.
\item OS needs code for each supported structure.
\item Every file may need to be defined as one of supported types.
\item New applications with unsupported structures: problems.
\item Example: encrypted file not ASCII text, not executable binary. May need to circumvent/misuse OS file-type mechanism.
\item Some OS impose minimal file structures (UNIX, Windows).
\item UNIX: each file is sequence of 8-bit bytes; no OS interpretation.
\item Provides maximum flexibility but little support; application must interpret structure.
\item All OS must support at least one structure: executable file.
\end{itemize}

\subsection*{Internal File Structure}
\begin{itemize}
    \item Internally, locating offset within file can be complicated for OS.
    \item Disk systems: well-defined block size (sector size).
    \item All disk I/O in units of one block (physical record); all blocks same size.
    \item Physical record size unlikely to match desired logical record length.
    \item Logical records may vary in length.
    \item Solution: packing logical records into physical blocks.
    \item UNIX: all files are streams of bytes. Each byte individually addressable by offset.
    \item File system automatically packs/unpacks bytes into physical disk blocks (e.g., 512 bytes/block).
    \item Logical record size, physical block size, packing technique determine records per block.
    \item Packing by user app or OS. File considered sequence of blocks.
    \item Basic I/O functions operate in terms of blocks.
    \item Conversion from logical records to physical blocks: simple software problem.
    \item Disk space allocated in blocks: some portion of last block wasted (\textbf{internal fragmentation}).
    \item Example: 512-byte blocks, 1,949-byte file $\rightarrow$ 4 blocks (2,048 bytes); 99 bytes wasted.
    \item All file systems suffer internal fragmentation; larger block size $\rightarrow$ greater fragmentation.
\end{itemize}

\subsection*{Section glossary}
\begin{tabular}{p{0.3\textwidth}p{0.7\textwidth}}
    \toprule
    \textbf{Term} & \textbf{Definition} \\
    \midrule
    \textbf{file} & Smallest logical storage unit; collection of related information. \\
    \textbf{text file} & File containing text (alphanumeric characters). \\
    \textbf{source file} & File containing program source code. \\
    \textbf{executable file} & File containing program ready for loading/execution. \\
    \textbf{extended file attributes} & Extended metadata (character encoding, checksums). \\
    \textbf{file info window} & GUI view of file metadata. \\
    \textbf{write pointer} & Location in file for next write. \\
    \textbf{read pointer} & Location in file for next read. \\
    \textbf{current-file-position pointer} & Per-process pointer to next read/write location. \\
    \textbf{seek} & Operation of changing current file-position pointer. \\
    \textbf{hard links} & File-system links where file has two+ names to same inode. \\
    \textbf{open-file table} & OS data structure with details of every open file. \\
    \textbf{open count} & Number of processes with an open file. \\
    \textbf{shared lock} & File lock allowing concurrent acquisition by multiple processes. \\
    \textbf{exclusive lock} & File lock allowing only one process to acquire at a time. \\
    \textbf{advisory file-lock mechanism} & File-locking system where OS does not enforce locking. \\
    \textbf{shell script} & File containing set series of commands specific to shell. \\
    \textbf{magic number} & Number at start of file indicating data type. \\
    \textbf{internal fragmentation} & Wasted disk space in last block of file due to block allocation. \\
    \bottomrule
\end{tabular}
