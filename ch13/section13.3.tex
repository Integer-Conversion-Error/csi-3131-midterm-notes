\section{Directory structure}

Directory: symbol table translating file names to file control blocks.
Organization must allow:
\begin{itemize}
    \item Insert entries
    \item Delete entries
    \item Search for named entry
    \item List all entries
\end{itemize}

Operations on a directory:
\begin{itemize}
    \item \textbf{Search for a file}: Find entry for particular file; find files matching pattern.
    \item \textbf{Create a file}: Add new files to directory.
    \item \textbf{Delete a file}: Remove file from directory; may leave hole, defragmentation needed.
    \item \textbf{List a directory}: List files and their entry contents.
    \item \textbf{Rename a file}: Change file name when contents/use changes; may change position.
    \item \textbf{Traverse the file system}: Access every directory/file; for backup or space release.
\end{itemize}

\subsection{Single-level directory}
\begin{itemize}
    \item Simplest structure: all files in same directory.
    \item Easy to support and understand.
    \item Limitations:
    \begin{itemize}
        \item Files must have unique names (name collision problem for multiple users).
        \item Difficult for single user to remember many file names.
    \end{itemize}
\end{itemize}

\subsection{Two-level directory}
\begin{itemize}
    \item Separate directory for each user.
    \item Each user has own \textbf{user file directory (UFD)}.
    \item System's \textbf{main file directory (MFD)} indexed by user name/account, points to UFD.
    \item When user refers to file, only their UFD searched.
    \item Different users can have same file names (unique within each UFD).
    \item OS searches only user's UFD for create/delete.
    \item UFDs created/deleted by special system program (restricted to administrators).
    \item Disadvantages:
    \begin{itemize}
        \item Isolates users; disadvantage for cooperation.
        \item To access another user's file, must specify user name and file name.
        \item Two-level directory as a tree (MFD root, UFDs descendants, files leaves).
        \item User name + file name = \textbf{path name}.
        \item Example: \texttt{/userb/test.txt} or \texttt{C:\textbackslash{}userb\textbackslash{}test}.
    \end{itemize}
    \item System files:
    \begin{itemize}
        \item Copying system files to each UFD wastes space.
        \item Solution: special user directory for system files (e.g., user 0).
        \item OS first searches local UFD, then special system directory.
        \item Sequence of directories searched: \textbf{search path}.
        \item Search path can be extended; users can have own search paths.
    \end{itemize}
\end{itemize}

\subsection{Tree-structured directories}
\begin{itemize}
    \item Generalization of two-level directory to arbitrary height.
    \item Most common directory structure.
    \item Root directory; every file has unique path name.
    \item Directory (or subdirectory) contains files or subdirectories.
    \item Directory often treated as special file; one bit defines entry as file (0) or subdirectory (1).
    \item Special system calls to create/delete directories.
    \item Each process has a \textbf{current directory}.
    \item Reference to file: current directory searched.
    \item If not in current directory: specify path name or change current directory.
    \item Initial current directory from accounting file.
    \item Path names:
    \begin{itemize}
        \item \textbf{Absolute path name}: begins at root (e.g., "/"), follows path down.
        \item \textbf{Relative path name}: defines path from current directory.
        \item Example: if current is \texttt{/spell/mail}, \texttt{prt/first} is same as \texttt{/spell/mail/prt/first}.
    \end{itemize}
    \item User defines subdirectories for organization (e.g., by topic, info type).
    \item Deletion of a directory:
    \begin{itemize}
        \item If empty: entry simply deleted.
        \item If not empty:
        \begin{itemize}
            \item Some systems: only delete if empty (user must delete contents recursively first).
            \item Others (e.g., UNIX \texttt{rm -r}): delete directory and all its files/subdirectories recursively. More convenient, but dangerous.
        \end{itemize}
    \end{itemize}
    \item Users can access other users' files by specifying path name or changing current directory.
\end{itemize}

\subsection{Acyclic-graph directories}
\begin{itemize}
    \item Allows directories to share subdirectories and files.
    \item No cycles (loops) in the graph.
    \item Shared file: one actual file exists, changes visible to all.
    \item Shared subdirectory: new files appear in all shared subdirectories.
    \item Implementation:
    \begin{itemize}
        \item \textbf{Link}: pointer to another file/subdirectory (e.g., absolute/relative path name).
        \item \textbf{Resolve}: use path name in link to locate real file.
        \item OS ignores links during directory traversal to preserve acyclic structure.
        \item Duplicate all info in both sharing directories: entries identical, but consistency issues on modification.
    \end{itemize}
    \item Problems:
    \begin{itemize}
        \item Multiple absolute path names for same file (aliasing).
        \item Traversing entire file system: avoid traversing shared structures more than once.
        \item Deletion: when can space be deallocated?
        \begin{itemize}
            \item Deleting file leaves dangling pointers if other links exist.
            \item Symbolic links: deletion of link doesn't affect original file. If original deleted, links dangle.
            \item Preserve file until all references deleted: use \textbf{reference count}.
            \item Increment count on new link/entry, decrement on deletion. Delete when count is 0.
            \item UNIX uses for \textbf{hard links}.
        \end{itemize}
    \end{itemize}
\end{itemize}

\subsection{General graph directory}
\begin{itemize}
    \item Allows cycles in the directory structure.
    \item Primary advantage of acyclic graph: simpler traversal and deletion algorithms.
    \item Problems with cycles:
    \begin{itemize}
        \item Infinite loops during search/traversal.
        \item Reference count may not be 0 even if file/directory is inaccessible.
        \item Requires \textbf{garbage collection} to determine when space can be reallocated (time consuming for disk-based systems).
    \end{itemize}
    \item Avoiding cycles: computationally expensive to detect.
    \item Simpler: bypass links during directory traversal.
\end{itemize}

\begin{table}[h!]
\centering
\caption{Section glossary}
\label{tab:section_glossary}
\begin{tabular}{p{0.3\textwidth}p{0.7\textwidth}}
\toprule
\rowcolor{gray!20} \textbf{Term} & \textbf{Definition} \\
\midrule
\textbf{user file directory (UFD)} & Per-user directory of files in two-level directory implementation. \\
\textbf{main file directory (MFD)} & Index pointing to each UFD in two-level directory implementation. \\
\textbf{path name} & File-system name for a file, containing mount-point and directory-entry info to locate it (e.g., "C:/foo/bar.txt"). \\
\textbf{search path} & Sequence of directories searched for an executable file when a command is executed. \\
\textbf{absolute path name} & Path name starting at the top of the file system hierarchy. \\
\textbf{relative path name} & Path name starting at a relative location (e.g., current directory). \\
\textbf{acyclic graph} & Directory structure implementation that contains no cycles (loops). \\
\textbf{link} & File that has no contents but points to another file. \\
\textbf{resolve} & To follow a link and find the target file. \\
\textbf{hard links} & File-system links where a file has two or more names pointing to the same inode. \\
\textbf{garbage collection} & Recovery of space containing no-longer-valid data. \\
\bottomrule
\end{tabular}
\end{table}
