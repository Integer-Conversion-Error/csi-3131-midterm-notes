\section{Protection}

Information safety:
\begin{itemize}
    \item Physical damage: \textbf{reliability} (duplicate copies, backups).
    \item Improper access: \textbf{protection}.
\end{itemize}

Protection mechanisms:
\begin{itemize}
    \item User name/password authentication.
    \item Encrypting secondary storage.
    \item Firewalling network access.
    \item Multiuser systems: advanced mechanisms for valid data access.
\end{itemize}

\subsection{Types of access}
\begin{itemize}
    \item Need for controlled access.
    \item Protection limits types of file access.
    \item Operations controlled:
    \begin{itemize}
        \item \textbf{Read}: Read from file.
        \item \textbf{Write}: Write or rewrite file.
        \item \textbf{Execute}: Load and execute file.
        \item \textbf{Append}: Write new info at end of file.
        \item \textbf{Delete}: Delete file, free space.
        \item \textbf{List}: List name and attributes.
        \item \textbf{Attribute change}: Change file attributes.
    \end{itemize}
    \item Higher-level functions (rename, copy, edit) often implemented by system programs using lower-level calls. Protection at lower level.
\end{itemize}

\subsection{Access control}
\begin{itemize}
    \item Access dependent on user identity.
    \item Most general scheme: \textbf{access-control list (ACL)}.
    \item ACL specifies user names and allowed access types.
    \item OS checks ACL; allows if listed, denies otherwise.
    \item Advantages: complex access methodologies.
    \item Disadvantages:
    \begin{itemize}
        \item Lengthy lists (tedious to construct, especially if users unknown).
        \item Variable-size directory entries (complicated space management).
    \end{itemize}
    \item Condensed ACL: three user classifications:
    \begin{itemize}
        \item \textbf{Owner}: User who created file.
        \item \textbf{Group}: Set of users sharing file, needing similar access.
        \item \textbf{Other}: All other users.
    \end{itemize}
    \item Common approach: combine ACLs with owner, group, universe scheme (e.g., Solaris).
    \item UNIX permissions:
    \begin{itemize}
        \item Three fields: owner, group, universe.
        \item Each field: three bits \texttt{rwx} (read, write, execute).
        \item \texttt{r} for read, \texttt{w} for write, \texttt{x} for execution.
        \item Example: \texttt{rwx} for owner, \texttt{rw-} for group, \texttt{r--} for others.
        \item \texttt{d} as first character indicates subdirectory.
        \item Sample listing shows links, owner, group, size, date, name.
    \end{itemize}
    \item Combining ACLs and permissions:
    \begin{itemize}
        \item User interface challenge: how to show optional ACLs.
        \item Solaris: "+" appended to regular permissions (e.g., \texttt{-rw-r--r--+}).
        \item Commands like \texttt{setfacl} and \texttt{getfacl} manage ACLs.
        \item Windows: GUI for ACL management.
        \item Precedence: ACLs typically take precedence over group permissions (specificity priority).
    \end{itemize}
\end{itemize}

\subsection{Other protection approaches}
\begin{itemize}
    \item Password with each file:
    \begin{itemize}
        \item Effective if passwords random and changed often.
        \item Disadvantages: many passwords to remember; single password for all files (all-or-none protection).
        \item Some systems: password with subdirectory.
        \item More commonly: encryption of partition/files, with key password management.
    \end{itemize}
    \item Directory protection in multilevel structures:
    \begin{itemize}
        \item Control creation/deletion of files in directory.
        \item Control user's ability to determine file existence (listing directory contents).
        \item If path name refers to file, user needs access to both directory and file.
        \item Different access rights depending on path name in acyclic/general graphs.
    \end{itemize}
\end{itemize}

\begin{table}[h!]
\centering
\caption{Section glossary}
\label{tab:section_glossary}
\begin{tabular}{p{0.3\textwidth}p{0.7\textwidth}}
\toprule
\rowcolor{gray!20} \textbf{Term} & \textbf{Definition} \\
\midrule
\textbf{access-control list} & A list of user names allowed to access a file. \\
\bottomrule
\end{tabular}
\end{table}
