\section{Memory-mapped files}

Alternative file access method: \textbf{memory mapping} a file.
\begin{itemize}
    \item Treat file I/O as routine memory accesses using virtual memory techniques.
    \item Can lead to significant performance increases.
\end{itemize}

\subsection{Basic mechanism}
\begin{itemize}
    \item Map disk block to page(s) in memory.
    \item Initial access: demand paging, page fault.
    \item Page-sized portion of file read into physical page.
    \item Subsequent reads/writes: handled as routine memory accesses.
    \item Simplifies and speeds up file access by avoiding \texttt{read()} and \texttt{write()} system call overhead.
    \item Writes to memory-mapped file not necessarily immediate to secondary storage.
    \item Generally, updates written back when file closed.
    \item Under memory pressure, intermediate changes may go to swap space.
    \item Some OS (e.g., Solaris) memory-map all file I/O, even with standard calls, to kernel address space.
    \item Multiple processes can map same file concurrently for data sharing.
    \item Writes by one process visible to others mapping same section.
    \item Implemented by virtual memory map pointing to same physical page.
    \item Supports copy-on-write: processes share read-only, get own copies for modification.
    \item Processes use mutual exclusion for shared data coordination.
    \item Shared memory often implemented by memory mapping files.
\end{itemize}

\subsection{Shared memory in the Windows API}
\begin{itemize}
    \item Outline for shared memory using memory-mapped files:
    \begin{enumerate}
        \item Create a \textbf{file mapping} for the file.
        \item Establish a \textbf{view} of the mapped file in process's virtual address space.
    \end{enumerate}
    \item Second process opens and creates view of same mapped file.
    \item Mapped file acts as shared-memory object for inter-process communication.
    \item Example: Producer writes, Consumer reads.
    \item Steps:
    \begin{enumerate}
        \item Open file with \texttt{CreateFile()} (returns \texttt{HANDLE}).
        \item Create file mapping with \texttt{CreateFileMapping()} (uses file \texttt{HANDLE}).
        \item Establish view with \texttt{MapViewOfFile()} (uses mapped object \texttt{HANDLE}).
    \end{enumerate}
    \item \texttt{CreateFileMapping()} creates a \textbf{named shared-memory object} (e.g., \texttt{SharedObject}).
    \item \texttt{MapViewOfFile()} returns pointer to shared-memory object; accesses to this memory are accesses to the file.
    \item Entire file or portion can be mapped.
    \item Mapped file may be demand-paged.
    \item Both processes remove view with \texttt{UnmapViewOfFile()}.
\end{itemize}

\begin{table}[h!]
\centering
\caption{Section glossary}
\label{tab:section_glossary}
\begin{tabular}{p{0.3\textwidth}p{0.7\textwidth}}
\toprule
\rowcolor{gray!20} \textbf{Term} & \textbf{Definition} \\
\midrule
\textbf{memory mapping} & File-access method where file is mapped into process memory space for direct memory access. \\
\textbf{file mapping} & In Windows, the first step in memory-mapping a file. \\
\textbf{view} & In Windows, an address range mapped in shared memory; second step in memory-mapping a file. \\
\textbf{named shared-memory object} & In Windows API, a section of a memory-mapped file accessible by name from multiple processes. \\
\bottomrule
\end{tabular}
\end{table}
