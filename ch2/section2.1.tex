\section*{2.1 Operating-system services}
\addcontentsline{toc}{section}{2.1 Operating-system services}

\subsection*{Introduction}
\addcontentsline{toc}{subsection}{Introduction}
An operating system (OS) provides the environment for program execution. OS designs vary, but defining system goals is crucial before design begins. We can view an OS by its services, user/programmer interface, or internal components. This chapter explores these aspects, covering OS services, their provision, debugging, design methodologies, creation, and booting process.

\subsection*{Chapter objectives}
\addcontentsline{toc}{subsection}{Chapter objectives}
\begin{itemize}
    \item Identify services provided by an operating system.
    \item Illustrate how system calls are used to provide operating system services.
    \item Compare and contrast monolithic, layered, microkernel, modular, and hybrid strategies for designing operating systems.
    \item Illustrate the process for booting an operating system.
    \item Apply tools for monitoring operating system performance.
    \item Design and implement kernel modules for interacting with a Linux kernel.
\end{itemize}

\subsection*{Operating-system services}
\addcontentsline{toc}{subsection}{Operating-system services}
An OS provides an environment for program execution, offering services to programs and users. While specific services vary, common classes exist. These services also simplify programming tasks.

OS services helpful to the user:
\begin{itemize}
    \item \textbf{User interface (UI):} Provides interaction via GUI, touch-screen, or CLI.
    \item \textbf{Program execution:} Loads, runs, and terminates programs.
    \item \textbf{I/O operations:} Manages program access to files and I/O devices for efficiency and protection.
    \item \textbf{File-system manipulation:} Handles file/directory operations (read, write, create, delete, search, list) and permissions.
    \item \textbf{Communications:} Enables inter-process communication (IPC) via shared memory or message passing, locally or across networks.
    \item \textbf{Error detection:} Constantly detects and corrects errors in hardware, I/O devices, and user programs, taking appropriate action (e.g., halting, terminating process, returning error code).
\end{itemize}
OS functions ensuring efficient system operation by sharing computer resources among multiple processes:
\begin{itemize}
    \item \textbf{Resource allocation:} Manages allocation of CPU cycles, memory, file storage, and I/O devices to multiple running processes, using scheduling routines.
    \item \textbf{Logging:} Tracks resource usage for accounting or accumulating statistics to improve computing services.
    \item \textbf{Protection and security:} Controls access to system resources and defends against external/internal attacks (e.g., viruses, DoS). Requires user authentication and safeguards for I/O devices.
\end{itemize}

\subsection*{Section glossary}
\addcontentsline{toc}{subsection}{Section glossary}
\rowcolors{2}{gray!10}{white}
\centering
\begin{tabular}{>{\raggedright}p{0.35\textwidth} >{\raggedright\arraybackslash}p{0.55\textwidth}}
\toprule
\textbf{Term} & \textbf{Definition} \\
\midrule
\textbf{user interface (UI)} & A method by which a user interacts with a computer. \\
\textbf{graphical user interface (GUI)} & A computer interface comprising a window system with a pointing device to direct I/O, choose from menus, and make selections and, usually, a keyboard to enter text. \\
\textbf{touch-screen interface} & A user interface in which touching a screen allows the user to interact with the computer. \\
\textbf{command-line interface (CLI)} & A method of giving commands to a computer based on a text input device (such as a keyboard). \\
\textbf{shared memory} & In interprocess communication, a section of memory shared by multiple processes and used for message passing. \\
\textbf{message passing} & In interprocess communication, a method of sharing data in which messages are sent and received by processes. Packets of information in predefined formats are moved between processes or between computers. \\
\bottomrule
\end{tabular}
\vspace{\baselineskip}
