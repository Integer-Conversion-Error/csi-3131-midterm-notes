\section*{2.11 Summary}
\addcontentsline{toc}{section}{2.11 Summary}

\begin{itemize}
    \item An operating system provides an environment for the execution of programs by providing services to users and programs.
    \item The three primary approaches for interacting with an operating system are (1) command interpreters, (2) graphical user interfaces, and (3) touch-screen interfaces.
    \item System calls provide an interface to the services made available by an operating system. Programmers use a system call's application programming interface (API) for accessing system-call services.
    \item System calls can be divided into six major categories: (1) process control, (2) file management, (3) device management, (4) information maintenance, (5) communications, and (6) protection.
    \item The standard C library provides the system-call interface for UNIX and Linux systems.
    \item Operating systems also include a collection of system programs that provide utilities to users.
    \item A \textbf{linker} combines several relocatable object modules into a single binary executable file. A \textbf{loader} loads the executable file into memory, where it becomes eligible to run on an available CPU.
    \item There are several reasons why applications are operating-system specific. These include different binary formats for program executables, different instruction sets for different CPUs, and system calls that vary from one operating system to another.
    \item An operating system is designed with specific goals in mind. These goals ultimately determine the operating system's policies. An operating system implements these policies through specific mechanisms.
    \item A \textbf{monolithic} operating system has no structure; all functionality is provided in a single, static binary file that runs in a single address space. Although such systems are difficult to modify, their primary benefit is efficiency.
    \item A \textbf{layered} operating system is divided into a number of discrete layers, where the bottom layer is the hardware interface and the highest layer is the user interface. Although layered software systems have had some success, this approach is generally not ideal for designing operating systems due to performance problems.
    \item The \textbf{microkernel} approach for designing operating systems uses a minimal kernel; most services run as user-level applications. Communication takes place via message passing.
    \item A \textbf{modular} approach for designing operating systems provides operating-system services through modules that can be loaded and removed during run time. Many contemporary operating systems are constructed as \textbf{hybrid systems} using a combination of a monolithic kernel and modules.
    \item A \textbf{boot loader} loads an operating system into memory, performs initialization, and begins system execution.
    \item The performance of an operating system can be monitored using either \textbf{counters} or \textbf{tracing}. Counters are a collection of system-wide or per-process statistics, while tracing follows the execution of a program through the operating system.
\end{itemize}
