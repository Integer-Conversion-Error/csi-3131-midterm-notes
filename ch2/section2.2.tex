\section*{2.2 User and operating-system interface}
\addcontentsline{toc}{section}{2.2 User and operating-system interface}

Users interface with the OS via three fundamental approaches: command-line interface (CLI) or \textbf{command interpreter}, and two forms of graphical user interface (GUI).

\subsection*{Command interpreters}
\addcontentsline{toc}{subsection}{Command interpreters}
\begin{itemize}
    \item Most OS (Linux, UNIX, Windows) treat the command interpreter as a special program run at process initiation or user login.
    \item Systems with multiple command interpreters are known as \textbf{shells} (e.g., C shell, Bourne-Again shell, Korn shell on UNIX/Linux).
    \item Shells provide similar functionality; choice is personal preference.
    \item Main function: get and execute user commands, often manipulating files (create, delete, list, print, copy, execute).
    \item Commands can be implemented:
    \begin{itemize}
        \item \textbf{Interpreter contains code:} Interpreter holds execution code, increasing its size.
        \item \textbf{System programs:} Most commands are separate system programs (e.g., UNIX `rm file.txt` executes the `rm` program). Allows easy addition of new commands without changing the interpreter.
    \end{itemize}
\end{itemize}

\subsection*{Graphical user interface}
\addcontentsline{toc}{subsection}{Graphical user interface}
\begin{itemize}
    \item A second interface is the \textbf{graphical user interface (GUI)}.
    \item Users interact with a mouse-based window-and-menu system using a \textbf{desktop} metaphor.
    \item \textbf{Icons} represent programs, files, and functions; clicking them or selecting from menus invokes actions.
    \item GUIs originated from Xerox PARC (1973, Xerox Alto) and became widespread with Apple Macintosh (1980s).
    \item macOS adopted the Aqua interface.
    \item Microsoft Windows (Version 1.0) added a GUI to MS-DOS, with later versions enhancing functionality.
    \item Traditionally CLI-dominated UNIX systems now offer GUIs like KDE and GNOME, available under open-source licenses.
\end{itemize}

\subsection*{Touch-screen interface}
\addcontentsline{toc}{subsection}{Touch-screen interface}
\begin{itemize}
    \item Smartphones and tablets typically use a touch-screen interface, as CLI/mouse systems are impractical.
    \item Users interact via \textbf{gestures} (e.g., pressing, swiping).
    \item Most mobile devices simulate keyboards on the touch screen.
    \item Apple iPhone/iPad use the \textbf{Springboard} touch-screen interface.
\end{itemize}

\subsection*{Choice of interface}
\addcontentsline{toc}{subsection}{Choice of interface}
\begin{itemize}
    \item Choice between CLI and GUI is often personal preference.
    \item \textbf{System administrators} and \textbf{power users} prefer CLI for efficiency and faster access, especially for repetitive tasks via \textbf{shell scripts}.
    \item Some systems offer only a subset of functions via GUI, reserving less common tasks for CLI.
    \item Most Windows users prefer the GUI; recent Windows versions offer both standard GUI and touch-screen interfaces.
    \item macOS (based on a UNIX kernel) now provides both the Aqua GUI and a command-line interface.
    \item Mobile systems (iOS, Android) predominantly use touch-screen interfaces.
    \item The user interface is typically separate from the core OS structure; this book focuses on providing adequate service to user programs, not UI design.
\end{itemize}

\subsection*{Section glossary}
\addcontentsline{toc}{subsection}{Section glossary}
\rowcolors{2}{gray!10}{white}
\centering
\begin{tabular}{>{\raggedright}p{0.35\textwidth} >{\raggedright\arraybackslash}p{0.55\textwidth}}
\toprule
\textbf{Term} & \textbf{Definition} \\
\midrule
\textbf{command interpreter} & OS component interpreting user commands. \\
\textbf{shell} & A command interpreter on systems with multiple choices. \\
\textbf{desktop} & GUI workspace on screen. \\
\textbf{icons} & Images representing objects in GUI. \\
\textbf{folder} & File system component for grouping files. \\
\textbf{gestures} & Motions causing computer actions (e.g., "pinching"). \\
\textbf{Springboard} & The iOS touch-screen interface. \\
\textbf{system administrators} & Users who configure, monitor, and manage systems. \\
\textbf{power users} & Users with deep system knowledge. \\
\textbf{shell script} & File containing a series of commands for a specific shell. \\
\bottomrule
\end{tabular}
\vspace{\baselineskip}
