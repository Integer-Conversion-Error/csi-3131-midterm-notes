\section*{2.3 System calls}
\addcontentsline{toc}{section}{2.3 System calls}

\subsection*{Example}
\addcontentsline{toc}{subsection}{Example}
\begin{itemize}
    \item System calls provide an interface to OS services, typically as C/C++ functions, sometimes assembly.
    \item Even simple programs (e.g., file copy) make extensive use of system calls.
    \item A file copy program (\texttt{cp in.txt out.txt}) involves:
    \begin{itemize}
        \item Obtaining file names (command line, user input, GUI).
        \item Opening input file, creating/opening output file (with error handling for non-existence, protection, or existing files).
        \item Looping to read from input and write to output (with error handling for EOF, hardware failure, disk space).
        \item Closing files, displaying messages, and terminating the program.
    \end{itemize}
    \item System calls are fundamental for program interaction with the OS.
\end{itemize}

\subsection*{Application programming interface}
\addcontentsline{toc}{subsection}{Application programming interface}
\begin{itemize}
    \item Application developers typically use an \textbf{application programming interface} (\textit{API}) rather than direct system calls.
    \item An API specifies functions, parameters, and return values available to programmers (e.g., Windows API, POSIX API, Java API).
    \item APIs are accessed via libraries (e.g., \textbf{libc} for C programs on UNIX/Linux).
    \item API functions invoke actual system calls in the kernel (e.g., \texttt{CreateProcess()} calls \texttt{NTCreateProcess()}).
    \item Benefits of using APIs:
    \begin{itemize}
        \item \textbf{Portability:} Programs can run on any system supporting the same API.
        \item \textbf{Simplicity:} APIs are often less detailed and easier to use than raw system calls.
    \end{itemize}
    \item The \textbf{run-time environment} (\textit{RTE}) provides a \textbf{system-call interface} that links API calls to OS system calls.
    \item The system-call interface uses a table indexed by system call numbers to invoke the correct kernel function.
    \item Details of system call implementation are hidden from the programmer by the API and managed by the RTE.
    \item Parameters to system calls can be passed via registers, memory blocks, or a \textbf{stack} (using \textbf{push} and \textbf{pop} operations).
\end{itemize}

\subsection*{Types of system calls}
\addcontentsline{toc}{subsection}{Types of system calls}
System calls are broadly categorized into six types:
\begin{itemize}
    \item \textbf{Process control:}
    \begin{itemize}
        \item Terminate program normally (\texttt{end()}) or abnormally (\texttt{abort()}).
        \item Debugging tools (\textbf{debugger}) can examine memory dumps from abnormal terminations.
        \item Load and execute other programs (\texttt{load()}, \texttt{execute()}).
        \item Create new processes (\texttt{create\_process()}, \texttt{fork()}).
        \item Control process attributes (priority, execution time) (\texttt{get\_process\_attributes()}, \texttt{set\_process\_attributes()}).
        \item Terminate created processes (\texttt{terminate\_process()}).
        \item Wait for processes to finish (\texttt{wait\_time()}, \texttt{wait\_event()}) or signal events (\texttt{signal\_event()}).
        \item \textbf{Lock} shared data for integrity (\texttt{acquire\_lock()}, \texttt{release\_lock()}).
        \item Examples: Arduino (single-tasking, \textbf{sketch}, \textbf{boot loader}); FreeBSD (multitasking, \texttt{fork()}, \texttt{exec()}, \texttt{exit()}).
    \end{itemize}
    \item \textbf{File management:}
    \begin{itemize}
        \item Basic operations: \texttt{create()}, \texttt{delete()}, \texttt{open()}, \texttt{read()}, \texttt{write()}, \texttt{reposition()}, \texttt{close()} files.
        \item Similar operations apply to directories.
        \item Get/set file/directory attributes (\texttt{get\_file\_attributes()}, \texttt{set\_file\_attributes()}).
    \end{itemize}
    \item \textbf{Device management:}
    \begin{itemize}
        \item Request (\texttt{request()}) and release (\texttt{release()}) devices for exclusive use.
        \item Read, write, and reposition devices, similar to files.
        \item Many OS (e.g., UNIX) merge files and devices into a combined structure.
    \end{itemize}
    \item \textbf{Information maintenance:}
    \begin{itemize}
        \item Transfer information between user program and OS.
        \item Examples: current \texttt{time()} and \texttt{date()}, OS version, free memory/disk space.
        \item Debugging aids: \texttt{dump()} memory, \texttt{strace} (Linux), \textbf{single step} CPU mode.
        \item Get/set process information (\texttt{get\_process\_attributes()}, \texttt{set\_process\_attributes()}).
    \end{itemize}
    \item \textbf{Communications:}
    \begin{itemize}
        \item \textbf{Message-passing model:} Processes exchange messages directly or via mailboxes.
        \begin{itemize}
            \item Requires opening connections, knowing \textbf{host name} and \textbf{process name}.
            \item \textbf{Daemons} (\textbf{server}) wait for connections from \textbf{clients}.
            \item Use \texttt{read\_message()} and \texttt{write\_message()} for exchange.
        \end{itemize}
        \item \textbf{Shared-memory model:} Processes access shared memory regions.
        \begin{itemize}
            \item Processes agree to remove memory access restrictions.
            \item Exchange information by reading/writing shared data.
            \item Faster communication, but requires protection and synchronization.
        \end{itemize}
        \item Most systems implement both models.
    \end{itemize}
    \item \textbf{Protection:}
    \begin{itemize}
        \item Control access to computer system resources.
        \item System calls: \texttt{set\_permission()}, \texttt{get\_permission()}, \texttt{allow\_user()}, \texttt{deny\_user()}.
        \item Essential for multiprogrammed, networked, and mobile systems.
    \end{itemize}
\end{itemize}
\rowcolors{2}{gray!10}{white}

\begin{tabular}{>{\raggedright}p{0.35\textwidth} >{\raggedright\arraybackslash}p{0.55\textwidth}}
\toprule
\textbf{Term} & \textbf{Definition} \\
\midrule
\textbf{system call} & Software-triggered interrupt allowing a process to request a kernel service. \\
\textbf{application programming interface (API)} & A set of commands, functions, and other tools that can be used by a programmer in developing a program. \\
\textbf{C library (libc)} & The standard UNIX/Linux system API for programs written in the C programming language. \\
\textbf{run-time environment (RTE)} & The full suite of software needed to execute applications written in a given programming language, including its compilers, libraries, and loaders. \\
\textbf{system-call interface} & An interface that serves as the link to system calls made available by the operating system and that is called by processes to invoke system calls. \\
\textbf{push} & The action of placing a value on a stack data structure. \\
\textbf{stack} & A sequentially ordered data structure that uses the last-in, first-out (LIFO) principle for adding and removing items; the last item placed onto a stack is the first item removed. \\
\textbf{process control} & A category of system calls. \\
\textbf{information maintenance} & A category of system calls. \\
\textbf{communications} & A category of system calls. \\
\textbf{protection} & A category of system calls. Any mechanism for controlling the access of processes or users to the resources defined by a computer system. \\
\textbf{debugger} & A system program designed to aid programmers in finding and correcting errors. \\
\textbf{bug} & An error in computer software or hardware. \\
\textbf{lock} & A mechanism that restricts access by processes or subroutines to ensure integrity of shared data. \\
\textbf{sketch} & An Arduino program. \\
\textbf{single step} & A CPU mode in which a trap is executed by the CPU after every instruction (to allow examination of the system state after every instruction); useful in debugging. \\
\textbf{message-passing model} & A method of interprocess communication in which messages are exchanged. \\
\textbf{host name} & A human-readable name for a computer. \\
\textbf{process name} & A human-readable name for a process. \\
\textbf{daemon} & A service that is provided outside of the kernel by system programs that are loaded into memory at boot time and run continuously. \\
\textbf{client} & A computer that uses services from other computers (such as a web client). The source of a communication. \\
\textbf{server} & In general, any computer, no matter the size, that provides resources to other computers.\\
\textbf{shared-memory model} & An interprocess communication method in which multiple processes share memory and use that memory for message passing. \\
\bottomrule
\end{tabular}
\vspace{\baselineskip}
