\section*{2.4 System services}
\addcontentsline{toc}{section}{2.4 System services}

\textbf{System services} (or \textbf{system utilities}) provide a convenient environment for program development and execution, categorized as:
\begin{itemize}
    \item \textbf{File management:} Create, delete, copy, rename, print, list, manipulate files/directories.
    \item \textbf{Status information:} Query system for date, time, memory/disk, user count; performance, logging, debugging. \textbf{Registry} for configuration.
    \item \textbf{File modification:} Text editors for content; search/transform commands.
    \item \textbf{Programming-language support:} Compilers, assemblers, debuggers, interpreters (C, C++, Java, Python) often OS-provided.
    \item \textbf{Program loading and execution:} Loaders (absolute, relocatable, linkage editors, overlay); debugging for high-level/machine languages.
    \item \textbf{Communications:} Virtual connections among processes, users, systems. Messages, web browsing, e-mail, remote login, file transfer.
    \item \textbf{Background services:} System programs launched at boot. Constantly running: \textbf{services}, \textbf{subsystems}, \textbf{daemons} (e.g., network, schedulers, error monitoring, print servers).
\end{itemize}
\textbf{Application programs} (e.g., web browsers, word processors) supplied. User's OS view shaped by these programs, not system calls; enables diverse interfaces (GUI, CLI) or dual-booting.

\subsection*{Section glossary}
\addcontentsline{toc}{subsection}{Section glossary}
\rowcolors{2}{gray!10}{white}
\begin{tabular}{>{\raggedright}p{0.35\textwidth} >{\raggedright\arraybackslash}p{0.55\textwidth}}
\toprule
\textbf{Term} & \textbf{Definition} \\
\midrule
\textbf{system service} & A collection of applications included with or added to an operating system to provide services beyond those provided by the kernel. \\
\textbf{system utility} & A collection of applications included with or added to an operating system to provide services beyond what are provided by the kernel. \\
\textbf{registry} & A file, set of files, or service used to store and retrieve configuration information. In Windows, the manager of hives of data. \\
\textbf{service} & A software entity running on one or more machines and providing a particular type of function to calling clients. In Android, an application component with no user interface; it runs in the background while executing long-running operations or performing work for remote processes. \\
\textbf{subsystem} & A subset of an operating system responsible for a specific function (e.g., memory management). \\
\textbf{application program} & A Program designed for end-user execution, such as a word processor, spreadsheet, compiler, or Web browser. \\
\bottomrule
\end{tabular}
\vspace{\baselineskip}
