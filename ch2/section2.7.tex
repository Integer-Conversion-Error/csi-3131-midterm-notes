\section*{2.7 Operating-system debugging}
\addcontentsline{toc}{section}{2.7 Operating-system debugging}

Debugging is a crucial task in operating system development, involving the detection and correction of errors. Errors can range from hardware failures to application bugs.

\subsection*{2.7.1: Section review questions.}
\addcontentsline{toc}{subsection}{2.7.1: Section review questions.}

\textbf{Debugging} is the process of finding and fixing errors, or \textbf{bugs}. Operating systems, being large and complex, are prone to bugs. Debugging can be challenging due to the difficulty of reproducing errors and the system's concurrent nature.

Common debugging tools and techniques:
\begin{itemize}
    \item \textbf{Log files:} System-generated files that record events, errors, and warnings. Analyzing these logs helps identify the cause of issues.
    \item \textbf{Core dump:} A snapshot of the memory of a process at the time of a crash. It contains the state of the program, including register values, stack, and memory, useful for post-mortem analysis.
    \item \textbf{Crash dump (or system dump):} Similar to a core dump but for the entire operating system. It captures the system's state when a kernel panic or system crash occurs.
    \item \textbf{Debugger:} A software tool that allows developers to step through code, inspect variables, and set breakpoints to understand program execution flow and identify bugs.
    \item \textbf{Tracing:} A technique to record the sequence of events or function calls during program execution, providing insights into behavior.
    \item \textbf{Profiling:} Analyzing program performance to identify bottlenecks and areas for optimization.
\end{itemize}

\textbf{Performance tuning} is a related activity focused on improving system efficiency. Tools for performance tuning often include:
\begin{itemize}
    \item \textbf{Profilers:} Identify which parts of a program consume the most resources (CPU, memory).
    \item \textbf{System monitors:} Track real-time system resource usage (CPU utilization, memory usage, disk I/O, network activity).
\item \textbf{Benchmarking tools:} Measure system performance under specific workloads to compare against baselines or other systems.
\end{itemize}

\subsection*{Section glossary}
\addcontentsline{toc}{subsection}{Section glossary}

\rowcolors{2}{gray!10}{white}
\begin{tabular}{>{\raggedright}p{0.35\textwidth} >{\raggedright\arraybackslash}p{0.55\textwidth}}
\toprule
\textbf{Term} & \textbf{Definition} \\
\midrule
\textbf{debugging} & The process of finding and fixing errors. \\
\textbf{bug} & An error in a program. \\
\textbf{core dump} & A file containing the state of a program when it crashed. \\
\textbf{crash dump} & A file containing the state of the operating system when it crashed. \\
\textbf{system dump} & A file containing the state of the operating system when it crashed. \\
\textbf{performance tuning} & An activity that improves the performance of a system. \\
\bottomrule
\end{tabular}
\vspace{\baselineskip}
