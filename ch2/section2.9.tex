\section*{2.9 Building and booting an operating system}
\addcontentsline{toc}{section}{2.9 Building and booting an operating system}

Operating systems are commonly designed to run on a class of machines with various peripheral configurations, rather than a single specific machine.

\subsection*{Operating-system generation}
\addcontentsline{toc}{subsection}{Operating-system generation}
When generating (or building) an operating system from scratch, the following steps are typically involved:
\begin{enumerate}
    \item Write or obtain the OS source code.
    \item Configure the OS for the target system.
    \item Compile the OS.
    \item Install the OS.
    \item Boot the computer with the new OS.
\end{enumerate}
Configuration involves specifying features, usually stored in a configuration file.
\begin{itemize}
    \item \textbf{System build:} Modifying source code and recompiling the entire OS for a tailored version.
    \item \textbf{Precompiled modules:} Selecting and linking precompiled object modules from a library. Faster, but less tailored.
    \item \textbf{Completely modular:} Selection occurs at execution time by setting parameters.
\end{itemize}
Most modern OSes for desktops, laptops, and mobile devices use the second approach, combining specific hardware generation with modular support (e.g., loadable kernel modules) for dynamic changes.

Building a Linux system from scratch typically involves:
\begin{enumerate}
    \item Downloading Linux source code (e.g., from \texttt{http://www.kernel.org}).
    \item Configuring the kernel using \texttt{make menuconfig} to generate a \texttt{.config} file.
    \item Compiling the main kernel using \texttt{make}, producing the \texttt{vmlinuz} kernel image.
    \item Compiling kernel modules using \texttt{make modules}.
    \item Installing kernel modules into \texttt{vmlinuz} using \texttt{make modules\_install}.
    \item Installing the new kernel using \texttt{make install}.
\end{enumerate}
The system will run the new OS upon reboot.

Alternatively, Linux can be installed as a virtual machine (VM):
\begin{itemize}
    \item \textbf{Building from scratch:} Similar to building Linux, but without OS compilation.
    \item \textbf{VM appliance:} Downloading a pre-built and configured OS appliance and installing it with virtualization software (e.g., VirtualBox, VMware).
\end{itemize}
Example for building a VM: download Ubuntu ISO, use VM software to boot from ISO, answer installation questions, install and boot.

\subsection*{System boot}
\addcontentsline{toc}{subsection}{System boot}
The process of starting a computer by loading the kernel is known as \textbf{booting}.
\begin{enumerate}
    \item A small piece of code (\textbf{bootstrap program} or \textbf{boot loader}) locates the kernel.
    \item The kernel is loaded into memory and started.
    \item The kernel initializes hardware.
    \item The root file system is mounted.
\end{enumerate}
\textbf{Multistage boot process:}
\begin{itemize}
    \item Initial boot loader in nonvolatile firmware (\textbf{BIOS}) loads a second boot loader from a \textbf{boot block} on disk.
    \item \textbf{UEFI} (Unified Extensible Firmware Interface) is a modern replacement for BIOS, offering better 64-bit/large disk support and faster single-stage booting.
\end{itemize}
The bootstrap program performs diagnostics, initializes system aspects (CPU registers, device controllers, memory), starts the OS, and mounts the root file system. The system is then considered \textbf{running}.

\textbf{GRUB} is an open-source bootstrap program for Linux/UNIX.
\begin{itemize}
    \item Boot parameters are set in a GRUB configuration file.
    \item Allows runtime changes (kernel parameters, selecting different kernels).
    \item Linux kernel image is compressed; extracted after loading.
    \item Creates a temporary RAM file system (\texttt{initramfs}) for necessary drivers/modules.
    \item Switches to the real root file system after drivers are installed.
    \item Creates the \texttt{systemd} process (initial process), then starts other services.
\end{itemize}
Booting mechanisms are boot loader-dependent (e.g., specific GRUB versions for BIOS/UEFI).

\textbf{Mobile system boot (e.g., Android):}
\begin{itemize}
    \item Different from PCs; vendors provide boot loaders (e.g., LK for Android).
    \item Uses compressed kernel image and initial RAM file system.
    \item Android maintains \texttt{initramfs} as the root file system (unlike Linux, which discards it).
    \item Starts the \texttt{init} process and creates services before displaying the home screen.
\end{itemize}
Most OS boot loaders (Windows, Linux, macOS, iOS, Android) provide \textbf{recovery mode} or \textbf{single-user mode} for diagnostics and repairs.

\subsection*{Section glossary}
\addcontentsline{toc}{subsection}{Section glossary}

\rowcolors{2}{gray!10}{white}
\begin{tabular}{>{\raggedright}p{0.35\textwidth} >{\raggedright\arraybackslash}p{0.55\textwidth}}
\toprule
\textbf{Term} & \textbf{Definition} \\
\midrule
\textbf{system build} & Creation of an operating-system build and configuration for a specific computer site. \\
\textbf{booting} & The procedure of starting a computer by loading the kernel. \\
\textbf{bootstrap program} & The program that allows the computer to start running by initializing hardware and loading the kernel. \\
\textbf{BIOS} & Code stored in firmware and run at boot time to start system operation. \\
\textbf{boot block} & A block of code stored in a specific location on disk with the instructions to boot the kernel stored on that disk. \\
\textbf{unified extensible firmware interface (UEFI)} & The modern replacement for BIOS containing a complete boot manager. \\
\textbf{running} & The state of the operating system after boot when all kernel initialization has completed and system services have started. \\
\textbf{GRUB} & A common open-source bootstrap loader that allows selection of boot partitions and options to be passed to the selected kernel. \\
\textbf{recovery mode} & A system boot state providing limited services and designed to enable the system admin to repair system problems and debug system startup. \\
\textbf{single-user mode} & A system boot state providing limited services and designed to enable the system admin to repair system problems and debug system startup. \\
\bottomrule
\end{tabular}
\vspace{\baselineskip}
