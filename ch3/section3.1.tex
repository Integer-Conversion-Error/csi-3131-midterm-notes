\phantomsection
\section{Process Concept}\label{sec:3.1}

\subsection{Process Definition and Characteristics}
\begin{itemize}
    \item \textbf{Process}: Program in execution.
    \item Requires resources: CPU time, memory, files, I/O devices.
    \item Unit of work in most systems.
    \item Systems comprise OS processes (system code) and user processes (user code), executing concurrently.
    \item Modern OS supports multiple \textit{threads} of control within processes; threads can run in parallel on multicore systems.
    \item OS schedules threads onto available processing cores.
\end{itemize}

\subsection{Program vs. Process}
\begin{itemize}
    \item \textbf{Program vs. Process}:
        \begin{itemize}
            \item Program: Passive entity (e.g., executable file on disk).
            \item Process: Active entity (program counter, associated resources).
            \item Program becomes a process when loaded into memory.
        \end{itemize}
    \item Multiple processes can be associated with the same program (e.g., multiple instances of a web browser); text sections are identical, but data, heap, and stack sections differ.
\end{itemize}

\subsection{Process Memory Layout}
\begin{itemize}
    \item \textbf{Process Memory Layout}:
        \begin{itemize}
            \item \textbf{Text section}: Executable code (fixed size).
            \item \textbf{Data section}: Global variables (fixed size).
            \item \textbf{Heap section}: Dynamically allocated memory during runtime (grows/shrinks).
            \item \textbf{Stack section}: Temporary data for function calls (parameters, return addresses, local variables) (grows/shrinks).
            \item Stack and heap grow towards each other; OS prevents overlap.
            \item \textbf{Activation record}: Pushed onto stack on function call, popped on return.
        \end{itemize}
\end{itemize}

\subsection{Process States}
\begin{itemize}
    \item \textbf{Process States}: A process changes state during execution.
        \begin{itemize}
            \item \textbf{New}: Process is being created.
            \item \textbf{Running}: Instructions are being executed (only one process per processor core at any instant).
            \item \textbf{Waiting}: Process is waiting for an event (e.g., I/O completion, signal reception).
            \item \textbf{Ready}: Process is waiting to be assigned to a processor.
            \item \textbf{Terminated}: Process has finished execution.
        \end{itemize}
\end{itemize}

\subsection{Process Control Block (PCB)}
\begin{itemize}
    \item \textbf{Process Control Block (PCB)} / \textbf{Task Control Block}:
        \begin{itemize}
            \item Data structure in OS representing each process.
            \item Contains information:
                \begin{itemize}
                    \item Process state.
                    \item Program counter.
                    \item CPU registers.
                    \item CPU-scheduling information (priority, queue pointers).
                    \item Memory-management information (base/limit registers, page/segment tables).
                    \item Accounting information (CPU/real time used, limits).
                    \item I/O status information (allocated devices, open files).
                \end{itemize}
            \item Serves as repository for data needed to start/restart a process.
        \end{itemize}
\end{itemize}

\subsection{Threads}
\begin{itemize}
    \item \textbf{Threads}:
        \begin{itemize}
            \item Single thread of execution: process performs one task at a time.
            \item Multithreaded process: multiple threads of execution, performs multiple tasks concurrently.
            \item Especially beneficial on multicore systems for parallel execution.
            \item PCB expanded to include per-thread information.
        \end{itemize}
\end{itemize}
\subsection*{Section glossary}
\rowcolors{2}{gray!10}{white}
\centering
\begin{tabular}{>{\raggedright}p{0.35\textwidth} >{\raggedright\arraybackslash}p{0.55\textwidth}}
\toprule
\textbf{Term} & \textbf{Definition} \\
\midrule
\textbf{process} & A program loaded into memory and executing. \\
\textbf{job} & A set of commands or processes executed by a batch system. \\
\textbf{user programs} & User-level programs, as opposed to system programs. \\
\textbf{task} & A process, a thread activity, or, generally, a unit of computation on a computer. \\
\textbf{program counter} & A CPU register indicating the main memory location of the next instruction to load and execute. \\
\textbf{text section} & The executable code of a program or process. \\
\textbf{data section} & The data part of a program or process; it contains global variables. \\
\textbf{heap section} & The section of process memory that is dynamically allocated during process run time; it stores temporary variables. \\
\textbf{stack section} & The section of process memory that contains the stack; it contains activation records and other temporary data. \\
\textbf{activation record} & A record created when a function or subroutine is called; added to the stack by the call and removed when the call returns. Contains function parameters, local variables, and the return address. \\
\textbf{executable file} & A file containing a program that is ready to be loaded into memory and executed. \\
\textbf{state} & The condition of a process, including its current activity as well as its associated memory and disk contents. \\
\textbf{process control block} & A per-process kernel data structure containing many pieces of information associated with the process. \\
\textbf{task control block} & A per-process kernel data structure containing many pieces of information associated with the process. \\
\textbf{thread} & A process control structure that is an execution location. A process with a single thread executes only one task at a time, while a multithreaded process can execute a task per thread. \\
\bottomrule
\end{tabular}
\vspace{\baselineskip}
