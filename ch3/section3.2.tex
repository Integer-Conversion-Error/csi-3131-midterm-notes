\phantomsection
\section{Process Scheduling}\label{sec:3.2}

\subsection{Introduction to Process Scheduling}
\begin{itemize}
    \item \textbf{Multiprogramming objective}: Maximize CPU utilization by always having a process running.
    \item \textbf{Time sharing objective}: Switch CPU core among processes frequently for user interaction.
    \item \textbf{Process scheduler}: Selects an available process for execution on a core.
    \item Single core: one running process; multicore: multiple running processes.
    \item \textbf{Degree of multiprogramming}: Number of processes currently in memory.
    \item \textbf{Process types}:
        \begin{itemize}
            \item \textbf{I/O-bound process}: Spends more time doing I/O than computations.
            \item \textbf{CPU-bound process}: Spends more time doing computations than I/O.
        \end{itemize}
\end{itemize}

\subsection{Scheduling Queues}
\begin{itemize}
    \item \textbf{Scheduling Queues}:
        \begin{itemize}
            \item \textbf{Ready queue}: Processes ready and waiting for CPU execution (linked list of PCBs).
            \item \textbf{Wait queue}: Processes waiting for a specific event (e.g., I/O completion).
            \item \textbf{Queueing diagram}: Visual representation of process flow (ready queue, wait queues, resources).
            \item \textbf{Process flow}: New $\rightarrow$ Ready $\rightarrow$ Dispatched (Running) $\rightarrow$ (I/O wait, Child wait, Interrupt/Time slice expire $\rightarrow$ Ready) $\rightarrow$ Terminated.
        \end{itemize}
\end{itemize}

\subsection{CPU Scheduling}
\begin{itemize}
    \item \textbf{CPU Scheduling}:
        \begin{itemize}
            \item \textbf{CPU scheduler} role: Selects process from ready queue, allocates CPU core.
            \item Executes frequently (e.g., every 100ms or more).
            \item May forcibly remove CPU from a process.
        \end{itemize}
\end{itemize}

\subsection{Swapping}
\begin{itemize}
    \item \textbf{Swapping}:
        \begin{itemize}
            \item Intermediate scheduling form.
            \item Remove process from memory to disk ("swapped out") to reduce degree of multiprogramming.
            \item Reintroduce process to memory ("swapped in") to continue execution.
            \item Typically used when memory is overcommitted.
        \end{itemize}
\end{itemize}

\subsection{Context Switch}
\begin{itemize}
    \item \textbf{Context Switch}:
        \begin{itemize}
            \item Interrupts cause OS to change CPU core task to kernel routine.
            \item System saves current \textbf{context} (CPU registers, process state, memory-management info) of running process in its PCB.
            \item \textbf{State save}: Copying current CPU state.
            \item \textbf{State restore}: Copying saved context to resume operations.
            \item Context switch: Switching CPU core from one process to another (state save old, state restore new).
            \item Pure overhead (no useful work done during switch).
            \item Speed varies (memory speed, registers to copy, special instructions). Typical: several microseconds.
            \item Hardware support (multiple register sets) can speed up.
            \item More complex OS/memory management $\rightarrow$ more work during context switch.
        \end{itemize}
\end{itemize}
\subsection*{Section glossary}
\rowcolors{2}{gray!10}{white}
\centering
\begin{tabular}{>{\raggedright}p{0.35\textwidth} >{\raggedright\arraybackslash}p{0.55\textwidth}}
\toprule
\textbf{Term} & \textbf{Definition} \\
\midrule
\textbf{process scheduler} & A scheduler that selects an available process (possibly from a set of several processes) for execution on a CPU. \\
\textbf{degree of multiprogramming} & The number of processes in memory. \\
\textbf{parent} & In a tree data structure, a node that has one or more nodes connected below it. \\
\textbf{children} & In a tree data structure, nodes connected below another node. \\
\textbf{siblings} & In a tree data structure, child nodes of the same parent. \\
\textbf{I/O-bound process} & A process that spends more of its time doing I/O than doing computations \\
\textbf{CPU-bound process} & A process that spends more time executing on CPU than it does performing I/O. \\
\textbf{ready queue} & The set of processes ready and waiting to execute. \\
\textbf{wait queue} & In process scheduling, a queue holding processes waiting for an event to occur before they need to be put on CPU. \\
\textbf{dispatched} & Selected by the process scheduler to be executed next. \\
\textbf{CPU scheduler} & Kernel routine that selects a thread from the threads that are ready to execute and allocates a core to that thread. \\
\textbf{swapping} & Moving a process between main memory and a backing store. A process may be swapped out to free main memory temporarily and then swapped back in to continue execution. \\
\textbf{context} & When describing a process, the state of its execution, including the contents of registers, its program counter, and its memory context, including its stack and heap. \\
\textbf{state save} & Copying a process's context to save its state in order to pause its execution in preparation for putting another process on the CPU. \\
\textbf{state restore} & Copying a process's context from its saved location to the CPU registers in preparation for continuing the process's execution. \\
\textbf{context switch} & The switching of the CPU from one process or thread to another; requires performing a state save of the current process or thread and a state restore of the other. \\
\textbf{foreground} & Describes a process or thread that is interactive (has input directed to it), such as a window currently selected as active or a terminal window currently selected to receive input. \\
\textbf{background} & Describes a process or thread that is not currently interactive (has no interactive input directed to it), such as one not currently being used by a user. In the Grand Central Dispatch Apple OS scheduler, the scheduling class representing tasks that are not time sensitive and are not visible to the user. \\
\textbf{split-screen} & Running multiple foreground processes (e.g., on an iPad) but splitting the screen among the processes. \\
\textbf{service} & A software entity running on one or more machines and providing a particular type of function to calling clients. In Android, an application component with no user interface; it runs in the background while executing long-running operations or performing work for remote processes. \\
\bottomrule
\end{tabular}
\vspace{\baselineskip}
