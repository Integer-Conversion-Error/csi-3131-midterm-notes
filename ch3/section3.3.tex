\phantomsection
\section{Operations on Processes}\label{sec:3.3}

\subsection{Introduction to Process Operations}
\begin{itemize}
    \item Processes can execute concurrently; created and deleted dynamically.
    \item OS provides mechanisms for process creation and termination.
\end{itemize}

\subsection{Process Creation}
\begin{itemize}
    \item Parent process creates child processes, forming a \textbf{tree}.
    \item Each process has a unique \textbf{process identifier (pid)}.
    \item Child resources: from OS or subset of parent's. Parent may partition/share resources.
    \item Parent may pass initialization data to child.
    \item \textbf{Execution possibilities}:
        \begin{enumerate}
            \item Parent executes concurrently with children.
            \item Parent waits until children terminate.
        \end{enumerate}
    \item \textbf{Address-space possibilities}:
        \begin{enumerate}
            \item Child is a duplicate of parent (same program/data).
            \item Child has a new program loaded.
        \end{enumerate}
\end{itemize}

\subsubsection{UNIX/Linux Process Creation}
\begin{itemize}
    \item \textbf{UNIX/Linux Process Creation}:
        \begin{itemize}
            \item \texttt{fork()}: creates new process (child) as copy of parent's address space.
            \item Both parent/child continue after \texttt{fork()}.
            \item \texttt{fork()} return: 0 for child, child's pid for parent.
            \item \texttt{exec()}: replaces process's memory space with new program (loads binary). Does not return on success.
            \item \texttt{wait()}: parent waits for child's termination.
            \item Child inherits privileges, scheduling attributes, open files.
        \end{itemize}
\end{itemize}

\subsubsection{Windows Process Creation}
\begin{itemize}
    \item \textbf{Windows Process Creation}:
        \begin{itemize}
            \item \texttt{CreateProcess()}: similar to \texttt{fork()}, but loads specified program into child's address space at creation.
            \item Uses \texttt{STARTUPINFO} and \texttt{PROCESS\_INFORMATION} structures.
            \item \texttt{WaitForSingleObject()}: parent waits for child's completion.
        \end{itemize}
\end{itemize}

\subsection{Process Termination}
\begin{itemize}
    \item Process finishes, uses \texttt{exit()} to ask OS to delete it.
    \item Returns status value to waiting parent (via \texttt{wait()}).
    \item All process resources deallocated.
    \item Another process can terminate a process (e.g., \texttt{TerminateProcess()} in Windows), usually only by parent.
    \item \textbf{Reasons for parent terminating child}:
        \begin{itemize}
            \item Child exceeded resource usage.
            \item Child's task no longer required.
            \item Parent exiting (some OS don't allow child to continue).
        \end{itemize}
    \item \textbf{Cascading termination}: If parent terminates, all children also terminate (OS initiated).
\end{itemize}

\subsubsection{UNIX/Linux Termination}
\begin{itemize}
    \item \textbf{UNIX/Linux Termination}:
        \begin{itemize}
            \item \texttt{exit(status)}: terminates process, provides exit status.
            \item \texttt{wait(\&status)}: parent waits for child, gets exit status, returns child's pid.
            \item \textbf{Zombie process}: Terminated process whose parent hasn't called \texttt{wait()}. Entry remains in process table. Brief state.
            \item \textbf{Orphan process}: Child of a parent that terminated without calling \texttt{wait()}.
            \item \texttt{init} (or \texttt{systemd}) becomes new parent for orphans, periodically calls \texttt{wait()} to clean up.
        \end{itemize}
\end{itemize}

\subsubsection{Android Process Hierarchy}
\begin{itemize}
    \item \textbf{Android Process Hierarchy} (for resource constraints/termination):
        \begin{itemize}
            \item \textbf{Foreground process}: Visible, user interacting (most important).
            \item \textbf{Visible process}: Not directly visible, but performing activity foreground process refers to.
            \item \textbf{Service process}: Similar to background, but apparent to user (e.g., streaming music).
            \item \textbf{Background process}: Performing activity, not apparent to user.
            \item \textbf{Empty process}: No active components (least important).
            \item Android terminates processes from least to most important to reclaim resources.
            \item Process state saved before termination, resumed if user navigates back.
        \end{itemize}
\end{itemize}
\subsection*{Section glossary}
\rowcolors{2}{gray!10}{white}
\centering
\begin{tabular}{>{\raggedright}p{0.35\textwidth} >{\raggedright\arraybackslash}p{0.55\textwidth}}
\toprule
\textbf{Term} & \textbf{Definition} \\
\midrule
\textbf{tree} & A data structure that can be used to represent data hierarchically; data values in a tree structure are linked through parent-child relationships. \\
\textbf{process identifier (pid)} & A unique value for each process in the system that can be used as an index to access various attributes of a process within the kernel. \\
\textbf{cascading termination} & A technique in which, when a process is ended, all of its children are ended as well. \\
\textbf{zombie} & A process that has terminated but whose parent has not yet called wait() to collect its state and accounting information. \\
\textbf{orphan} & The child of a parent process that terminates in a system that does not require a terminating parent to cause its children to be terminated. \\
\bottomrule
\end{tabular}
\vspace{\baselineskip}
