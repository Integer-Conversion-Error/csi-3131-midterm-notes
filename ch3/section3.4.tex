\phantomsection
\section{Interprocess communication}\label{sec:3.4}

Processes executing concurrently in the operating system may be either independent processes or cooperating processes. A process is \textbf{independent} if it does not share data with any other processes executing in the system. A process is \textbf{cooperating} if it can affect or be affected by the other processes executing in the system. Clearly, any process that shares data with other processes is a cooperating process.

There are several reasons for providing an environment that allows process cooperation:
\begin{itemize}
    \item \textbf{Information sharing.} Since several applications may be interested in the same piece of information (for instance, copying and pasting), we must provide an environment to allow concurrent access to such information.
    \item \textbf{Computation speedup.} If we want a particular task to run faster, we must break it into subtasks, each of which will be executing in parallel with the others. Notice that such a speedup can be achieved only if the computer has multiple processing cores.
    \item \textbf{Modularity.} We may want to construct the system in a modular fashion, dividing the system functions into separate processes or threads, as we discussed in the previous chapter Operating-System Structures.
\end{itemize}
Cooperating processes require an \textbf{interprocess communication} (\textit{IPC}) mechanism that will allow them to exchange data---that is, send data to and receive data from each other. There are two fundamental models of interprocess communication: \textbf{shared memory} and \textbf{message passing}. In the shared-memory model, a region of memory that is shared by the cooperating processes is established. Processes can then exchange information by reading and writing data to the shared region. In the message-passing model, communication takes place by means of messages exchanged between the cooperating processes.

Both of the models just mentioned are common in operating systems, and many systems implement both. Message passing is useful for exchanging smaller amounts of data, because no conflicts need be avoided. Message passing is also easier to implement in a distributed system than shared memory. (Although there are systems that provide distributed shared memory, we do not consider them in this text.) Shared memory can be faster than message passing, since message-passing systems are typically implemented using system calls and thus require the more time-consuming task of kernel intervention. In shared-memory systems, system calls are required only to establish shared-memory regions. Once shared memory is established, all accesses are treated as routine memory accesses, and no assistance from the kernel is required.

\subsection*{Chapter objectives}\addcontentsline{toc}{subsection}{Chapter objectives}
\begin{itemize}
    \item Describe and contrast interprocess communication using shared memory and message passing.
    \item Design programs that use pipes and POSIX shared memory to perform interprocess communication.
    \item Describe client-server communication using sockets and remote procedure calls.
\end{itemize}

\subsection*{Section glossary}\addcontentsline{toc}{subsection}{Section glossary}
\rowcolors{2}{gray!10}{white}
\centering
\begin{tabular}{>{\raggedright}p{0.35\textwidth} >{\raggedright\arraybackslash}p{0.55\textwidth}}
\toprule
\textbf{Term} & \textbf{Definition} \\
\midrule
\textbf{interprocess communication (IPC)} & Communication between processes. \\
\textbf{shared memory} & In interprocess communication, a section of memory shared by multiple processes and used for message passing. \\
\textbf{message passing} & In interprocess communication, a method of sharing data in which messages are sent and received by processes. Packets of information in predefined formats are moved between processes or between computers. \\
\textbf{browser} & A process that accepts input in the form of a URL (Uniform Resource Locator), or web address, and displays its contents on a screen. \\
\textbf{renderer} & A process that contains logic for rendering contents (such as web pages) onto a display. \\
\textbf{plug-in} & An add-on functionality that expands the primary functionality of a process (e.g., a web browser plug-in that displays a type of content different from what the browser can natively handle). \\
\textbf{sandbox} & A contained environment (e.g., a virtual machine). \\
\bottomrule
\end{tabular}
\vspace{\baselineskip}
