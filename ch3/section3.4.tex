\phantomsection
\section{Interprocess Communication (IPC)}\label{sec:3.4}

\subsection{Independent vs. Cooperating Processes}
\begin{itemize}
    \item Processes can be \textbf{independent} (no shared data) or \textbf{cooperating} (share data, affect each other).
\end{itemize}

\subsection{Reasons for Process Cooperation}
\begin{itemize}
    \item \textbf{Reasons for process cooperation}:
        \begin{itemize}
            \item \textbf{Information sharing}: Concurrent access to shared info (e.g., copy/paste).
            \item \textbf{Computation speedup}: Break task into subtasks for parallel execution (requires multiple cores).
            \item \textbf{Modularity}: Divide system functions into separate processes/threads.
        \end{itemize}
    \item Cooperating processes need \textbf{IPC mechanism} to exchange data.
\end{itemize}

\subsection{IPC Models}
\begin{itemize}
    \item \textbf{Two fundamental IPC models}:
        \begin{itemize}
            \item \textbf{Shared Memory}:
                \begin{itemize}
                    \item Region of memory shared by cooperating processes.
                    \item Exchange info by reading/writing to shared region.
                    \item Faster than message passing (after setup), no kernel intervention for access.
                \end{itemize}
            \item \textbf{Message Passing}:
                \begin{itemize}
                    \item Communication via messages exchanged between processes.
                    \item Useful for smaller data amounts, no conflicts.
                    \item Easier in distributed systems.
                    \item Typically uses system calls (kernel intervention), thus slower than shared memory.
                \end{itemize}
        \end{itemize}
\end{itemize}
\subsection*{Section glossary}
\rowcolors{2}{gray!10}{white}
\centering
\begin{tabular}{>{\raggedright}p{0.35\textwidth} >{\raggedright\arraybackslash}p{0.55\textwidth}}
\toprule
\textbf{Term} & \textbf{Definition} \\
\midrule
\textbf{interprocess communication (IPC)} & Communication between processes. \\
\textbf{shared memory} & In interprocess communication, a section of memory shared by multiple processes and used for message passing. \\
\textbf{message passing} & In interprocess communication, a method of sharing data in which messages are sent and received by processes. Packets of information in predefined formats are moved between processes or between computers. \\
\textbf{browser} & A process that accepts input in the form of a URL (Uniform Resource Locator), or web address, and displays its contents on a screen. \\
\textbf{renderer} & A process that contains logic for rendering contents (such as web pages) onto a display. \\
\textbf{plug-in} & An add-on functionality that expands the primary functionality of a process (e.g., a web browser plug-in that displays a type of content different from what the browser can natively handle). \\
\textbf{sandbox} & A contained environment (e.g., a virtual machine). \\
\bottomrule
\end{tabular}
\vspace{\baselineskip}
