\phantomsection
\section{IPC in Shared-Memory Systems}\label{sec:3.5}

\subsection{Shared Memory Setup}
\begin{itemize}
    \item \textbf{Shared Memory Setup}:
        \begin{itemize}
            \item Communicating processes establish a shared memory region.
            \item Region typically in address space of creator process.
            \item Other processes attach it to their address space.
            \item OS restriction on memory access removed by agreement.
            \item Processes responsible for data form, location, and concurrent access synchronization.
        \end{itemize}
\end{itemize}

\subsection{Producer-Consumer Problem}
\begin{itemize}
    \item \textbf{Producer-Consumer Problem}: Common paradigm for cooperating processes.
        \begin{itemize}
            \item \textbf{Producer} process: produces information.
            \item \textbf{Consumer} process: consumes information.
            \item Example: compiler (producer) $\rightarrow$ assembler (consumer).
            \item Metaphor for client-server paradigm (server=producer, client=consumer).
            \item Requires a buffer for items.
            \item \textbf{Buffer types}:
                \begin{itemize}
                    \item \textbf{Unbounded buffer}: No practical limit on size. Consumer may wait, producer always produces.
                    \item \textbf{Bounded buffer}: Fixed size. Consumer waits if empty, producer waits if full.
                \end{itemize}
        \end{itemize}
\end{itemize}

\subsection{Bounded Buffer Implementation}
\begin{itemize}
    \item \textbf{Bounded Buffer Implementation (Shared Memory Example)}:
        \begin{itemize}
            \item Shared variables: \texttt{buffer} (circular array), \texttt{in} (next free position), \texttt{out} (first full position).
            \item Buffer empty: \texttt{in == out}.
            \item Buffer full: \texttt{((in + 1) \% BUFFER\_SIZE) == out}.
            \item Allows \texttt{BUFFER\_SIZE - 1} items at most.
            \item Producer loop: produce item, wait if buffer full, add to buffer, update \texttt{in}.
            \item Consumer loop: wait if buffer empty, consume item, update \texttt{out}.
            \item \textbf{Issue}: Concurrent access to shared buffer requires synchronization (covered in later chapters).
        \end{itemize}
\end{itemize}
\subsection*{Section glossary}
\rowcolors{2}{gray!10}{white}
\centering
\begin{tabular}{>{\raggedright}p{0.35\textwidth} >{\raggedright\arraybackslash}p{0.55\textwidth}}
\toprule
\textbf{Term} & \textbf{Definition} \\
\midrule
\textbf{producer} & A process role in which the process produces information that is consumed by a consumer process. \\
\textbf{consumer} & A process role in which the process consumes information produced by a producer process. \\
\textbf{unbounded buffer} & A buffer with no practical limit on its memory size. \\
\textbf{bounded buffer} & A buffer with a fixed size. \\
\bottomrule
\end{tabular}
\vspace{\baselineskip}
