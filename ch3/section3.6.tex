\phantomsection
\section{IPC in Message-Passing Systems}\label{sec:3.6}

\subsection{Introduction to Message Passing}
\begin{itemize}
    \item Message passing: allows processes to communicate and synchronize without shared address space.
    \item Useful in distributed environments (processes on different computers).
    \item Provides `send(message)` and `receive(message)` operations.
    \item Messages can be fixed or variable size (trade-off: implementation complexity vs. programming ease).
    \item \textbf{Communication Link}: Must exist between communicating processes.
    \item \textbf{Logical Implementation Methods}:
        \begin{itemize}
            \item Direct or indirect communication.
            \item Synchronous or asynchronous communication.
            \item Automatic or explicit buffering.
        \end{itemize}
\end{itemize}

\subsection{Naming in Message Passing}
\begin{itemize}
    \item \textbf{Naming}: How processes refer to each other.
        \begin{itemize}
            \item \textbf{Direct Communication}: Explicitly name recipient/sender.
                \begin{itemize}
                    \item `send(P, message)`: Send to process P.
                    \item `receive(Q, message)`: Receive from process Q.
                    \item Properties: Link established automatically, exactly two processes per link, exactly one link per pair.
                    \item \textbf{Symmetry}: Both sender/receiver name each other.
                    \item \textbf{Asymmetry}: Only sender names recipient (`receive(id, message)` from any process).
                    \item Disadvantage: Limited modularity, \textbf{hard-coding} identifiers is undesirable.
                \end{itemize}
            \item \textbf{Indirect Communication}: Messages sent/received via \textbf{mailboxes} (or \textbf{ports}).
                \begin{itemize}
                    \item Mailbox: Object for placing/removing messages, unique ID.
                    \item `send(A, message)`: Send to mailbox A.
                    \item `receive(A, message)`: Receive from mailbox A.
                    \item Properties: Link only if shared mailbox, link may be associated with >2 processes, multiple links per pair.
                    \item Handling multiple receivers for one message:
                        \begin{itemize}
                            \item Link associated with $\le$ two processes.
                            \item At most one process executes `receive()` at a time.
                            \item System arbitrarily selects receiver (e.g., \textbf{round robin}).
                        \end{itemize}
                    \item Mailbox ownership:
                        \begin{itemize}
                            \item By process: Part of process address space. Owner receives, user sends. Disappears on owner termination.
                            \item By OS: Independent existence. OS provides create/send/receive/delete mechanisms. Can have multiple receivers.
                        \end{itemize}
                \end{itemize}
        \end{itemize}
\end{itemize}

\subsection{Synchronization in Message Passing}
\begin{itemize}
    \item \textbf{Synchronization}: `send()` and `receive()` primitives.
        \begin{itemize}
            \item \textbf{Blocking} (synchronous):
                \begin{itemize}
                    \item \textbf{Blocking send}: Sender blocked until message received by receiver/mailbox.
                    \item \textbf{Blocking receive}: Receiver blocked until message available.
                \end{itemize}
            \item \textbf{Nonblocking} (asynchronous):
                \begin{itemize}
                    \item \textbf{Nonblocking send}: Sender sends message, resumes operation.
                    \item \textbf{Nonblocking receive}: Receiver retrieves valid message or null.
                \end{itemize}
            \item \textbf{Rendezvous}: Both `send()` and `receive()` are blocking.
        \end{itemize}
\end{itemize}

\subsection{Buffering in Message Passing}
\begin{itemize}
    \item \textbf{Buffering}: Temporary queue for messages.
        \begin{itemize}
            \item \textbf{Zero capacity}: Queue length 0. Sender blocks until recipient receives. No buffering.
            \item \textbf{Bounded capacity}: Finite length $n$. Sender blocks if queue full. Automatic buffering.
            \item \textbf{Unbounded capacity}: Potentially infinite length. Sender never blocks. Automatic buffering.
        \end{itemize}
\end{itemize}
\subsection*{Section glossary}
\rowcolors{2}{gray!10}{white}
\centering
\begin{tabular}{>{\raggedright}p{0.35\textwidth} >{\raggedright\arraybackslash}p{0.55\textwidth}}
\toprule
\textbf{Term} & \textbf{Definition} \\
\midrule
\textbf{direct communication} & In interprocess communication, a communication mode in which each process that wants to communicate must explicitly name the recipient or sender of the communication. \\
\textbf{blocking} & In interprocess communication, a mode of communication in which the sending process is blocked until the message is received by the receiving process or by a mailbox and the receiver blocks until a message is available. In I/O, a request that does not return until the I/O completes. \\
\textbf{nonblocking} & A type of I/O request that allows the initiating thread to continue while the I/O operation executes. In interprocess communication, a communication mode in which the sending process sends the message and resumes operation and the receiver process retrieves either a valid message or a null if no message is available. In I/O, a request that returns whatever data is currently available, even if it is less than requested. \\
\textbf{synchronous} & In interprocess communication, a mode of communication in which the sending process is blocked until the message is received by the receiving process or by a mailbox and the receiver blocks until a message is available. In I/O, a request that does not return until the I/O completes. \\
\textbf{asynchronous} & In I/O, a request that executes while the caller continues execution. \\
\textbf{rendezvous} & In interprocess communication, when blocking mode is used, the meeting point at which a send is picked up by a receive. \\
\textbf{communication link} & A link between processes that allows them to send messages to and receive messages from each other. \\
\textbf{symmetry} & In direct communication, a scheme in which both the sender process and the receiver process must name the other to communicate. \\
\textbf{asymmetry} & In direct communication, a scheme in which only the sender names the recipient; the recipient is not required to name the sender. \\
\textbf{hard-coding} & Techniques where identifiers must be explicitly stated. \\
\textbf{indirect communication} & A communication mode in which messages are sent to and received from mailboxes, or ports. \\
\textbf{mailboxes} & In indirect communication, objects into which messages can be placed by processes and from which messages can be removed. \\
\textbf{ports} & In indirect communication, objects into which messages can be placed by processes and from which messages can be removed. \\
\textbf{round robin} & An algorithm for selecting which process will receive a message (e.g., processes take turns receiving messages). \\
\bottomrule
\end{tabular}
\vspace{\baselineskip}
