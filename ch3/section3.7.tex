\phantomsection
\section{Examples of IPC Systems}\label{sec:3.7}

\subsection{Introduction to IPC Examples}
\begin{itemize}
    \item Explores four IPC systems: POSIX shared memory, Mach message passing, Windows IPC, Pipes.
\end{itemize}

\subsection{POSIX Shared Memory}
\begin{itemize}
    \item \textbf{POSIX Shared Memory}:
        \begin{itemize}
            \item Organized using memory-mapped files.
            \item \texttt{shm\_open(name, O\_CREAT | O\_RDWR, 0666)}: Creates/opens shared-memory object, returns file descriptor.
            \item \texttt{ftruncate(fd, size)}: Configures size of object in bytes.
            \item \texttt{mmap()}: Establishes memory-mapped file, returns pointer for access.
            \item Producer-Consumer model example: Producer writes, Consumer reads.
            \item \texttt{MAP\_SHARED} flag: Changes visible to all sharing processes.
            \item Writing: Use \texttt{sprintf()} to pointer, increment pointer by bytes written.
            \item Consumer uses \texttt{shm\_unlink(name)} to remove segment after access.
        \end{itemize}
\end{itemize}

\subsection{Mach Message Passing}
\begin{itemize}
    \item \textbf{Mach Message Passing}:
        \begin{itemize}
            \item Designed for distributed systems (used in MacOS, iOS).
            \item Supports \textbf{tasks} (like processes, but with multiple threads, fewer resources).
            \item Communication via \textbf{messages} sent to/received from \textbf{ports} (mailboxes).
            \item Ports: finite size, unidirectional. Two-way communication needs separate \textbf{reply} port.
            \item Each port: multiple senders, one receiver.
            \item Uses \textbf{kernel abstractions} for resources (tasks, threads, memory, processors).
            \item \textbf{Port rights}: Capabilities for task interaction (e.g., \texttt{MACH\_PORT\_RIGHT\_RECEIVE}).
            \item Creator of port is owner, only owner can receive. Owner can manipulate capabilities.
            \item Ownership at task level: all threads in same task share port rights.
            \item Special ports on task creation: \textbf{Task Self} (kernel has receive rights, task sends to kernel), \textbf{Notify} (kernel sends notifications to task).
            \item \texttt{mach\_port\_allocate()}: Creates new port, allocates queue space, identifies rights. Port rights are \textbf{names} (integer values, like UNIX file descriptors).
            \item \textbf{Bootstrap port}: Allows task to register port with system-wide \textbf{bootstrap server}. Other tasks can look up and obtain send rights.
            \item Port queue: finite, initially empty. Messages copied into queue.
            \item Messages delivered reliably, same priority. FIFO for same sender, no absolute ordering across senders.
            \item \textbf{Mach Message Fields}:
                \begin{itemize}
                    \item Fixed-size header: metadata (size, source/destination ports).
                    \item Variable-sized body: data.
                \end{itemize}
            \item \textbf{Message types}:
                \begin{itemize}
                    \item \textbf{Simple}: Unstructured user data, not interpreted by kernel.
                    \item \textbf{Complex}: Pointers to memory (out-of-line data) or transferring port rights. Useful for large data (avoids copying).
                \end{itemize}
            \item \texttt{mach\_msg()}: Standard API for send/receive (\texttt{MACH\_SEND\_MSG} or \texttt{MACH\_RCV\_MSG}).
            \item Flexible send/receive operations:
                \begin{enumerate}
                    \item Wait indefinitely if queue full.
                    \item Wait at most $n$ milliseconds.
                    \item Return immediately (do not wait).
                    \item Temporarily cache message (for server tasks, allows sending reply even if client port full).
                \end{enumerate}
            \item Performance: Mach avoids message copying by mapping sender's address space into receiver's (virtual memory techniques). Works for intrasystem messages.
        \end{itemize}
\end{itemize}

\subsection{Windows IPC}
\begin{itemize}
    \item \textbf{Windows IPC}:
        \begin{itemize}
            \item Modern design, modularity. Supports multiple \textbf{subsystems}.
            \item Applications are clients of subsystem servers, communicate via message passing.
            \item \textbf{Advanced Local Procedure Call (ALPC)}: Message-passing facility for same-machine communication. Optimized RPC.
            \item Uses \textbf{port object} for connection. Two types: \textbf{connection ports} (published by server, visible) and \textbf{communication ports} (private, client-server pair).
            \item Server creates channel (pair of communication ports) upon client connection request.
            \item Channels support callback mechanism.
            \item \textbf{ALPC Message-Passing Techniques}:
                \begin{enumerate}
                    \item Small messages ($\le$ 256 bytes): copied via port's message queue.
                    \item Larger messages: passed through \textbf{section object} (shared memory region).
                    \item Very large data: API allows server to read/write directly into client's address space.
                \end{enumerate}
            \item ALPC not part of Windows API; applications use standard RPC, which is handled indirectly by ALPC for same-system calls. Kernel services also use ALPC.
        \end{itemize}
\end{itemize}

\subsection{Pipes}
\begin{itemize}
    \item \textbf{Pipes}: Conduit for two processes to communicate. Early UNIX IPC.
        \begin{itemize}
            \item \textbf{Considerations}:
                \begin{enumerate}
                    \item Bidirectional or unidirectional?
                    \item If two-way, half duplex or full duplex?
                    \item Parent-child relationship required?
                    \item Network communication or same machine only?
                \end{enumerate}
        \end{itemize}
\end{itemize}

\subsubsection{Ordinary Pipes}
\begin{itemize}
            \item \textbf{Ordinary Pipes} (UNIX) / \textbf{Anonymous Pipes} (Windows):
                \begin{itemize}
                    \item Unidirectional (producer writes to \textbf{write end}, consumer reads from \textbf{read end}). Two pipes for two-way.
                    \item UNIX: \texttt{pipe(int fd[])} creates pipe (\texttt{fd[0]} read, \texttt{fd[1]} write). Accessed via \texttt{read()}, \texttt{write()}.
                    \item UNIX: Cannot be accessed outside creator process. Parent creates, child inherits via \texttt{fork()}. Parent/child close unused ends.
                    \item Windows: \texttt{CreatePipe()} creates. \texttt{ReadFile()}, \texttt{WriteFile()}.
                    \item Windows: Requires explicit inheritance (\texttt{SECURITY\_ATTRIBUTES}, \texttt{SetHandleInformation}). Redirect child's standard input/output.
                    \item Both: Require parent-child relationship. Same machine only.
                \end{itemize}
\end{itemize}

\subsubsection{Named Pipes}
\begin{itemize}
            \item \textbf{Named Pipes} (UNIX FIFOs):
                \begin{itemize}
                    \item More powerful than ordinary pipes.
                    \item Bidirectional communication, no parent-child relationship required.
                    \item Several processes can use. Continue to exist after processes terminate.
                    \item UNIX FIFOs: Created with \texttt{mkfifo()}, appear as files. Manipulated with \texttt{open()}, \texttt{read()}, \texttt{write()}, \texttt{close()}. Exist until explicitly deleted.
                    \item UNIX FIFOs: Half-duplex only. Same machine only (use sockets for intermachine).
                    \item Windows Named Pipes: Full-duplex. Communicating processes can be on same or different machines.
                    \item Windows Named Pipes: Byte- or message-oriented data. Created with \texttt{CreateNamedPipe()}, client connects with \texttt{ConnectNamedPipe()}. Communication with \texttt{ReadFile()}, \texttt{WriteFile()}.
                \end{itemize}
        \end{itemize}
\subsection*{Section glossary}
\rowcolors{2}{gray!10}{white}
\centering
\begin{tabular}{>{\raggedright}p{0.35\textwidth} >{\raggedright\arraybackslash}p{0.55\textwidth}}
\toprule
\textbf{Term} & \textbf{Definition} \\
\midrule
\textbf{message} & In networking, a communication, contained in one or more packets, that includes source and destination information to allow correct delivery. In message-passing communications, a packet of information with metadata about its sender and receiver. \\
\textbf{port} & A communication address; a system may have one IP address for network connections but many ports, each for a separate communication. In computer I/O, a connection point for devices to attach to computers. In software development, to move code from its current platform to another platform (e.g., between operating systems or hardware systems). In the Mach OS, a mailbox for communication. \\
\textbf{bootstrap port} & In Mach message passing, a predefined port that allows a task to register a port it has created. \\
\textbf{bootstrap server} & In Mach message passing, a system-wide service for registering ports. \\
\textbf{advanced local procedure call (ALPC)} & In Windows OS, a method used for communication between two processes on the same machine. \\
\textbf{connection port} & In Windows OS, a communications port used to maintain connection between two processes, published by a server process. \\
\textbf{communication port} & In Windows OS, a port used to send messages between two processes. \\
\textbf{section object} & The Windows data structure that is used to implement shared memory. \\
\textbf{pipe} & A logical conduit allowing two processes to communicate. \\
\textbf{write end} & In ordinary pipes, the end of the pipe to which the producer writes. \\
\textbf{read end} & In ordinary pipes, the end of the pipe from which the consumer reads. \\
\textbf{anonymous pipes} & Ordinary pipes on Windows systems. \\
\bottomrule
\end{tabular}
\vspace{\baselineskip}
