\phantomsection
\section{Communication in Client-Server Systems}\label{sec:3.8}

\subsection{Introduction to Client-Server Communication}
\begin{itemize}
    \item IPC techniques (shared memory, message passing) used in client-server systems.
    \item Focus on \textbf{sockets} and \textbf{Remote Procedure Calls (RPCs)}.
\end{itemize}

\subsection{Sockets}
\begin{itemize}
    \item \textbf{Sockets}:
        \begin{itemize}
            \item Endpoint for communication.
            \item Pair of sockets for network communication (one per process).
            \item Identified by IP address + port number.
            \item Client-server architecture: Server listens on specified port, accepts client connection.
            \item \textbf{Well-known ports}: Below 1024 (e.g., SSH 22, FTP 21, HTTP 80).
            \item Client assigned arbitrary port > 1024.
            \item Connection: unique pair of sockets (client IP:port, server IP:port).
            \item Java socket types:
                \begin{itemize}
                    \item \textbf{Connection-oriented (TCP)}: `Socket` class.
                    \item \textbf{Connectionless (UDP)}: `DatagramSocket` class.
                    \item `MulticastSocket`: subclass of `DatagramSocket`, sends to multiple recipients.
                \end{itemize}
            \item Example: Date server (TCP) listens on port 6013.
                \begin{itemize}
                    \item Server creates `ServerSocket`, `accept()` blocks until client connects.
                    \item `accept()` returns `Socket` for communication.
                    \item Server uses `PrintWriter` to write to socket (`println()`).
                    \item Client creates `Socket`, connects to server IP:port.
                    \item Client reads from socket using stream I/O.
                    \item \textbf{Loopback} (127.0.0.1): refers to self, allows client/server on same host to communicate via TCP/IP.
                \end{itemize}
            \item Sockets are low-level: unstructured stream of bytes. Application imposes data structure.
        \end{itemize}
\end{itemize}

\subsection{Remote Procedure Calls (RPCs)}
\begin{itemize}
    \item \textbf{Remote Procedure Calls (RPCs)}:
        \begin{itemize}
            \item Abstracts procedure-call mechanism for network connections. Built on IPC.
            \item Messages are well-structured (function identifier, parameters).
            \item Addressed to RPC daemon listening on a \textbf{port} on remote system.
            \item Client invokes remote procedure as if local.
            \item \textbf{Stub}: Client-side code that hides communication details.
                \begin{itemize}
                    \item Locates server port, \textbf{marshals} parameters (converts to machine-independent format).
                    \item Transmits message to server.
                    \item Server-side stub receives, invokes procedure.
                    \item Return values passed back similarly.
                    \item Windows: Stub code from \textbf{Microsoft Interface Definition Language (MIDL)}.
                \end{itemize}
            \item \textbf{Parameter marshaling}: Handles data representation differences (e.g., \textbf{big-endian} vs. \textbf{little-endian}).
                \begin{itemize}
                    \item Uses machine-independent representation (e.g., \textbf{External Data Representation (XDR)}).
                    \item Client: machine-dependent $\rightarrow$ XDR. Server: XDR $\rightarrow$ machine-dependent.
                \end{itemize}
            \item \textbf{Semantics of call}: RPCs can fail/duplicate due to network errors.
                \begin{itemize}
                    \item \textbf{Exactly once}: OS ensures message acted on once. Harder to implement. This typically involves the server implementing an "at most once" semantic and acknowledging the client, with the client resending until an acknowledgment is received.
                    \item \textbf{At most once}: Attach timestamp to message. Server keeps history, ignores repeated timestamps.
                \end{itemize}
            \item \textbf{Binding}: Client needs to know server port numbers.
                \begin{itemize}
                    \item Predetermined: Fixed port addresses at compile time. Less flexible.
                    \item Dynamic: \textbf{Rendezvous} (or \textbf{matchmaker}) daemon on fixed RPC port. Client requests port address from daemon. More flexible.
                \end{itemize}
            \item Useful for distributed file systems (RPC daemons/clients, messages for disk operations).
        \end{itemize}
\end{itemize}

\subsubsection{Android RPC}
\begin{itemize}
    \item \textbf{Android RPC}:
        \begin{itemize}
            \item IPC between processes on same system via \textbf{binder} framework.
            \item \textbf{Application component}: Basic building block for Android app.
            \item \textbf{Service}: Application component with no UI, runs in background (e.g., playing music).
            \item `bindService()`: client binds to service.
            \item `onBind()`: invoked on service, returns `Messenger` (for message passing) or interface (for RPC).
            \item `Messenger` service: one-way. Client provides `replyTo` field for service to reply.
            \item RPCs: `onBind()` returns interface representing remote methods.
            \item Uses Android Interface Definition Language (AIDL) to create stub files (client interface).
            \item Android binder framework handles parameter marshaling, transfer, invocation.
        \end{itemize}
\end{itemize}
\subsection*{Section glossary}
\rowcolors{2}{gray!10}{white}
\centering
\begin{tabular}{>{\raggedright}p{0.35\textwidth} >{\raggedright\arraybackslash}p{0.55\textwidth}}
\toprule
\textbf{Term} & \textbf{Definition} \\
\midrule
\textbf{socket} & An endpoint for communication. An interface for network I/O. \\
\textbf{connection-oriented socket (TCP)} & In Java, a mode of communication. \\
\textbf{connectionless socket (UDP)} & In Java, a mode of communication. \\
\textbf{loopback} & Communication in which a connection is established back to the sender. \\
\textbf{port} & A communication address; a system may have one IP address for network connections but many ports, each for a separate communication. In computer I/O, a connection point for devices to attach to computers. In software development, to move code from its current platform to another platform (e.g., between operating systems or hardware systems). In the Mach OS, a mailbox for communication. \\
\textbf{stub} & A small, temporary place-holder function replaced by the full function once its expected behavior is known. \\
\textbf{Microsoft Interface Definition Language} & The Microsoft text-based interface definition language; used, e.g., to write client stub code and descriptors for RPC. \\
\textbf{big-endian} & A system architecture in which the most significant byte in a sequence of bytes is stored first. \\
\textbf{little-endian} & A system architecture that stores the least significant byte first in a sequence of bytes. \\
\textbf{external data representation} & A system used to resolve differences when data are exchanged between big- and little-endian systems. \\
\textbf{matchmaker} & A function that matches a caller to a service being called (e.g., a remote procedure call attempting to find a server daemon). \\
\textbf{binder} & In Android RPC, a framework (system component) for developing object-oriented OS services and allowing them to communicate. \\
\textbf{application component} & In Android, a basic building block that provides utility to an Android app. \\
\textbf{service} & A software entity running on one or more machines and providing a particular type of function to calling clients. In Android, an application component with no user interface; it runs in the background while executing long-running operations or performing work for remote processes. \\
\bottomrule
\end{tabular}
\vspace{\baselineskip}
