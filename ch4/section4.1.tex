\section{Threads \& Concurrency}\label{sec:4.1}

\subsection{Overview}
\begin{itemize}
    \item A \textbf{thread} is a basic unit of CPU utilization, comprising a thread ID, program counter (PC), register set, and stack.
    \item Threads share with other threads of the same process: code section, data section, and OS resources (e.g., open files, signals).
    \item A traditional process has a single thread of control; a multithreaded process can perform multiple tasks concurrently.
    \item Modern operating systems provide features for processes to contain multiple threads of control.
    \item Identifying opportunities for parallelism using threads is crucial for modern multicore systems.
\end{itemize}

\subsection{Motivation}
\begin{itemize}
    \item Most modern software applications are multithreaded (e.g., web browsers, word processors, image thumbnail generators).
    \item Multithreaded applications can leverage multicore systems for parallel CPU-intensive tasks.
    \item \textbf{Web Server Example}:
        \begin{itemize}
            \item A busy web server may handle thousands of concurrent client requests.
            \item Single-threaded server: services one client at a time, leading to long wait times.
            \item Process-creation method: server creates a separate process for each request (time-consuming, resource-intensive).
            \item Multithreaded server: creates a new thread for each request, more efficient.
        \end{itemize}
    \item Most OS kernels are multithreaded (e.g., Linux kernel threads like \texttt{kthreadd} for device management, memory management, interrupt handling).
    \item Multithreading is beneficial for CPU-intensive problems in data mining, graphics, and AI.
\end{itemize}

\subsection{Benefits}
\begin{itemize}
    \item \textbf{Responsiveness}:
        \begin{itemize}
            \item Allows an application to continue running even if part is blocked or performing a lengthy operation.
            \item Increases responsiveness to the user (e.g., UI remains active during a time-consuming background task).
        \end{itemize}
    \item \textbf{Resource Sharing}:
        \begin{itemize}
            \item Threads share memory and resources of their parent process by default.
            \item More efficient than inter-process communication (shared memory, message passing).
            \item Allows multiple threads of activity within the same address space.
        \end{itemize}
    \item \textbf{Economy}:
        \begin{itemize}
            \item More economical to create and context-switch threads than processes.
            \item Thread creation consumes less time and memory than process creation.
            \item Context switching is typically faster between threads.
        \end{itemize}
    \item \textbf{Scalability}:
        \begin{itemize}
            \item Greater benefits in multiprocessor architectures, where threads can run in parallel on different processing cores.
            \item A single-threaded process runs on only one processor, regardless of available cores.
        \end{itemize}
\end{itemize}

\subsection*{Section glossary}
\rowcolors{2}{gray!10}{white}
\centering
\begin{tabular}{>{\raggedright}p{0.35\textwidth} >{\raggedright\arraybackslash}p{0.55\textwidth}}
\toprule
\textbf{Term} & \textbf{Definition} \\
\midrule
\textbf{single-threaded} & A process or program that has only one thread of control (and so executes on only one core at a time). \\
\textbf{multithreaded} & A term describing a process or program with multiple threads of control, allowing multiple simultaneous execution points. \\
\textbf{thread} & A basic unit of CPU utilization; it comprises a thread ID, a program counter (PC), a register set, and a stack. It shares with other threads belonging to the same process its code section, data section, and other operating-system resources, such as open files and signals. \\
\bottomrule
\end{tabular}
\vspace{\baselineskip}
