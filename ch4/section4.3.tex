\section{Multithreading Models}\label{sec:4.3}

\subsection{User Threads vs. Kernel Threads}
\begin{itemize}
    \item Support for threads can be provided at the user level (\textbf{user threads}) or by the kernel (\textbf{kernel threads}).
    \item \textbf{User threads}: Supported above the kernel and managed without kernel support.
    \item \textbf{Kernel threads}: Supported and managed directly by the operating system.
    \item Most contemporary operating systems (Windows, Linux, macOS) support kernel threads.
    \item A relationship must exist between user threads and kernel threads.
\end{itemize}

\subsection{Many-to-One Model}
\begin{itemize}
    \item Maps many user-level threads to one kernel thread.
    \item Thread management is done by the thread library in user space, making it efficient.
    \item \textbf{Drawbacks}:
        \begin{itemize}
            \item The entire process blocks if a user thread makes a blocking system call.
            \item Multiple user threads cannot run in parallel on multicore systems because only one kernel thread can access the kernel at a time.
        \end{itemize}
    \item \textbf{Example}: Green threads (Solaris, early Java).
    \item Rarely used now due to inability to leverage multicore systems.
\end{itemize}

\subsection{One-to-One Model}
\begin{itemize}
    \item Maps each user thread to a kernel thread.
    \item Provides more concurrency than the many-to-one model:
        \begin{itemize}
            \item Another thread can run when a thread makes a blocking system call.
            \item Allows multiple threads to run in parallel on multiprocessors.
        \end{itemize}
    \item \textbf{Drawback}: Creating a user thread requires creating a corresponding kernel thread, which can burden system performance if too many kernel threads are created.
    \item \textbf{Implementations}: Linux, Windows operating systems.
    \item Most operating systems currently use this model.
\end{itemize}

\subsection{Many-to-Many Model}
\begin{itemize}
    \item Multiplexes many user-level threads to a smaller or equal number of kernel threads.
    \item The number of kernel threads can be specific to the application or machine (e.g., more kernel threads on a system with more cores).
    \item \textbf{Advantages}:
        \begin{itemize}
            \item Developers can create as many user threads as needed.
            \item Corresponding kernel threads can run in parallel on a multiprocessor.
            \item When a thread performs a blocking system call, the kernel can schedule another thread.
        \end{itemize}
    \item \textbf{Two-level model}: A variation where many user-level threads are multiplexed to a smaller or equal number of kernel threads, but also allows a user-level thread to be bound to a kernel thread.
    \item Difficult to implement in practice.
    \item Less common now as limiting kernel threads is less important with increasing core counts, but some contemporary concurrency libraries still use this model.
\end{itemize}

\subsection*{Section glossary}
\rowcolors{2}{gray!10}{white}
\centering
\begin{tabular}{>{\raggedright}p{0.35\textwidth} >{\raggedright\arraybackslash}p{0.55\textwidth}}
\toprule
\textbf{Term} & \textbf{Definition} \\
\midrule
\textbf{user thread} & A thread running in user mode, managed without kernel support. \\
\textbf{kernel threads} & Threads running in kernel mode, supported and managed directly by the operating system. \\
\textbf{many-to-one model} & A multithreading model that maps many user-level threads to one kernel thread. \\
\textbf{one-to-one model} & A multithreading model that maps each user thread to a kernel thread. \\
\textbf{many-to-many model} & A multithreading model that multiplexes many user-level threads to a smaller or equal number of kernel threads. \\
\textbf{two-level model} & A variation of the many-to-many model that also allows a user-level thread to be bound to a kernel thread. \\
\bottomrule
\end{tabular}
\vspace{\baselineskip}
