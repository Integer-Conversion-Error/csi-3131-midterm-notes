\section{Implicit Threading}\label{sec:4.5}

\subsection{Introduction to Implicit Threading}
\begin{itemize}
    \item Designing applications with many threads (hundreds or thousands) is complex, involving challenges related to correctness (e.g., synchronization, deadlocks).
    \item \textbf{Implicit threading}: A strategy to address these difficulties by transferring thread creation and management from application developers to compilers and run-time libraries.
    \item Developers identify \textbf{tasks} (functions) that can run in parallel, and the run-time library maps these tasks to separate threads, typically using the many-to-many model.
    \item \textbf{Advantage}: Developers focus on identifying parallelizable tasks, while libraries handle the details of thread creation and management.
\end{itemize}

\subsection{Thread Pools}
\begin{itemize}
    \item A \textbf{thread pool} involves creating a number of threads at startup and placing them into a pool, where they wait for work.
    \item \textbf{Problem with creating a new thread per request (e.g., web server)}:
        \begin{itemize}
            \item Time and overhead to create and discard threads for each request.
            \item Unbounded number of concurrent threads can exhaust system resources (CPU, memory).
        \end{itemize}
    \item \textbf{How thread pools work}:
        \begin{itemize}
            \item Server receives request, submits it to the thread pool.
            \item An available thread from the pool is awakened to service the request immediately.
            \item If no thread is available, the task is queued.
            \item Once a thread completes its work, it returns to the pool.
            \item Works well for asynchronously executed tasks.
        \end{itemize}
    \item \textbf{Benefits of thread pools}:
        \begin{enumerate}
            \item Faster request servicing with existing threads.
            \item Limits the number of concurrent threads, preventing resource exhaustion.
            \item Separates task definition from thread creation mechanics, allowing flexible scheduling (e.g., delayed or periodic execution).
        \end{enumerate}
    \item Number of threads in the pool can be set heuristically or dynamically adjusted based on system load.
    \item \textbf{Windows Thread Pool API}:
        \begin{itemize}
            \item Functions like \texttt{QueueUserWorkItem()} submit a function to be executed by a thread from the pool.
            \item Parameters include a pointer to the function, a parameter for the function, and flags for thread management.
        \end{itemize}
    \item \textbf{Java Thread Pools (\texttt{java.util.concurrent})}:
        \begin{itemize}
            \item Provides various thread pool architectures:
                \begin{itemize}
                    \item \texttt{newSingleThreadExecutor()}: Pool of size 1.
                    \item \texttt{newFixedThreadPool(int size)}: Pool with a specified number of threads.
                    \item \texttt{newCachedThreadPool()}: Unbounded pool that reuses threads.
                \end{itemize}
            \item Created using factory methods in the \texttt{Executors} class, returning an \texttt{ExecutorService} object.
            \item \texttt{ExecutorService} extends \texttt{Executor} (with \texttt{execute()} method) and provides methods for managing pool termination (\texttt{shutdown()}).
        \end{itemize}
\end{itemize}

\subsection{Fork-Join}
\begin{itemize}
    \item The \textbf{fork-join} model is a synchronous threading strategy where a parent thread creates (forks) child threads and waits for them to terminate and join with it to combine results.
    \item An excellent candidate for implicit threading: parallel tasks are designated, and a library manages thread creation and task assignment.
    \item Similar to a synchronous version of thread pools, where the library determines the number of threads.
    \item \textbf{Java Fork-Join Library (Java 1.7+)}:
        \begin{itemize}
            \item Designed for recursive divide-and-conquer algorithms (e.g., Quicksort, Mergesort).
            \item Tasks are forked during the divide step, assigned smaller subsets of the problem, and execute concurrently.
            \item Problem is solved directly when small enough (below a \texttt{THRESHOLD}).
            \item Uses \texttt{ForkJoinPool} to manage worker threads.
            \item Tasks inherit from \texttt{ForkJoinTask} (abstract base class), which is extended by:
                \begin{itemize}
                    \item \texttt{RecursiveTask}: Returns a result via its \texttt{compute()} method.
                    \item \texttt{RecursiveAction}: Does not return a result (void \texttt{compute()}).
                \end{itemize}
            \item \texttt{fork()} method creates new tasks; \texttt{join()} method blocks until a task completes and returns its result.
            \item \textbf{Work Stealing}: Each thread in a \texttt{ForkJoinPool} maintains a queue of forked tasks; if a queue is empty, it can "steal" a task from another thread's queue to balance workload.
        \end{itemize}
\end{itemize}

\subsection{OpenMP}
\begin{itemize}
    \item A set of compiler directives and an API for C, C++, or FORTRAN, supporting parallel programming in shared-memory environments.
    \item Identifies \textbf{parallel regions} as blocks of code that may run in parallel.
    \item Developers insert compiler directives (e.g., \texttt{\#pragma omp parallel}) to instruct the OpenMP run-time library to execute regions in parallel.
    \item When \texttt{\#pragma omp parallel} is encountered, OpenMP creates as many threads as there are processing cores in the system, and all threads execute the parallel region simultaneously. Threads are terminated upon exiting the region.
    \item Provides directives for parallelizing loops (e.g., \texttt{\#pragma omp parallel for}) to divide work among threads.
    \item Allows manual control over the number of threads and identification of shared vs. private data.
    \item Available on various compilers for Linux, Windows, and macOS.
\end{itemize}

\subsection{Grand Central Dispatch (GCD)}
\begin{itemize}
    \item Developed by Apple for macOS and iOS.
    \item A combination of a run-time library, API, and language extensions for identifying code sections (tasks) to run in parallel.
    \item Manages most threading details.
    \item Schedules tasks by placing them on a \textbf{dispatch queue}. Tasks are assigned to available threads from a managed pool.
    \item \textbf{Types of Dispatch Queues}:
        \begin{itemize}
            \item \textbf{Serial queues}: Tasks removed in FIFO order; one task completes before the next is removed. Each process has a \textbf{main queue} (private dispatch queue); developers can create additional local serial queues. Useful for sequential execution.
            \item \textbf{Concurrent queues}: Tasks removed in FIFO order, but multiple tasks can execute in parallel. System-wide concurrent queues (global dispatch queues) are divided by quality-of-service classes:
                \begin{itemize}
                    \item \textbf{QOS\_CLASS\_USER\_INTERACTIVE}: User interface and event handling (small amount of work, responsive UI).
                    \item \textbf{QOS\_CLASS\_USER\_INITIATED}: User-associated tasks requiring longer processing (e.g., opening files/URLs); must complete for user interaction to continue.
                    \item \textbf{QOS\_CLASS\_UTILITY}: Longer-running tasks not requiring immediate results (e.g., importing data).
                    \item \textbf{QOS\_CLASS\_BACKGROUND}: Non-visible, non-time-sensitive tasks (e.g., indexing, backups).
                \end{itemize}
        \end{itemize}
    \item \textbf{Task Submission}:
        \begin{itemize}
            \item C, C++, Objective-C: Use a \textbf{block} (self-contained unit of work, specified by \texttt{\^{}\{\}}).
            \item Swift: Use a \textbf{closure} (similar to a block, self-contained unit of functionality).
            \item Tasks submitted using functions like \texttt{dispatch\_async()}.
        \end{itemize}
    \item Internally, GCD's thread pool uses POSIX threads and dynamically adjusts thread count. Implemented by \texttt{libdispatch}.
\end{itemize}

\subsection{Intel Threading Building Blocks (TBB)}
\begin{itemize}
    \item A template library for designing parallel C++ applications.
    \item Requires no special compiler or language support.
    \item Developers specify tasks, and the TBB task scheduler maps them to underlying threads.
    \item Provides load balancing and is cache-aware (prioritizes tasks with data in cache).
    \item Offers templates for parallel loop structures (\texttt{parallel\_for}), atomic operations, mutual exclusion locking, and concurrent data structures.
    \item \textbf{Parallel For Loops}:
        \begin{itemize}
            \item \texttt{parallel\_for (range, body)}: \texttt{range} defines the iteration space, \texttt{body} specifies an operation on a subrange.
            \item Example: \texttt{parallel\_for (size\_t(0), n, [=](size\_t i) \{apply(v[i]);\});}
            \item TBB divides loop iterations into "chunks" and assigns tasks to available threads.
            \item Developers identify parallelizable operations, and the library manages work division and parallel execution.
        \end{itemize}
    \item Available in commercial and open-source versions for Windows, Linux, and macOS.
\end{itemize}

\subsection*{Section glossary}
\rowcolors{2}{gray!10}{white}
\centering
\begin{tabular}{>{\raggedright}p{0.35\textwidth} >{\raggedright\arraybackslash}p{0.55\textwidth}}
\toprule
\textbf{Term} & \textbf{Definition} \\
\midrule
\textbf{implicit threading} & A programming model that transfers the creation and management of threading from application developers to compilers and run-time libraries. \\
\textbf{thread pool} & A number of threads created at process startup and placed in a pool, where they sit and wait for work. \\
\textbf{fork-join} & A strategy for thread creation in which the main parent thread creates (forks) one or more child threads and then waits for the children to terminate and join with it. \\
\textbf{parallel regions} & Blocks of code that may run in parallel, identified by OpenMP. \\
\textbf{dispatch queue} & An Apple OS feature for parallelizing code; blocks are placed in the queue by Grand Central Dispatch (GCD) and removed to be run by an available thread. \\
\textbf{main queue} & Apple OS per-process serial dispatch queue. \\
\textbf{user-interactive} & In the Grand Central Dispatch Apple OS scheduler, the scheduling class representing tasks that interact with the user. \\
\textbf{user-initiated} & In the Grand Central Dispatch Apple OS scheduler, the scheduling class representing tasks that interact with the user but need longer processing times than user-interactive tasks. \\
\textbf{utility} & In the Grand Central Dispatch Apple OS scheduler, the scheduling class representing tasks that require a longer time to complete but do not demand immediate results. \\
\textbf{background} & In the Grand Central Dispatch Apple OS scheduler, the scheduling class representing tasks that are not time sensitive and are not visible to the user. \\
\textbf{block} & In the Grand Central Dispatch Apple OS scheduler, a language extension (for C, C++, Objective-C) that allows designation of a self-contained unit of work to be submitted to dispatch queues. \\
\textbf{iteration space} & In Intel Threading Building Blocks, the range of elements that will be iterated. \\
\bottomrule
\end{tabular}
\vspace{\baselineskip}
