\section{Threading Issues}\label{sec:4.6}

\subsection{The \texttt{fork()} and \texttt{exec()} System Calls}
\begin{itemize}
    \item The semantics of \texttt{fork()} and \texttt{exec()} change in multithreaded programs.
    \item \textbf{\texttt{fork()}} considerations:
        \begin{itemize}
            \item Some UNIX systems offer two versions of \texttt{fork()}:
                \begin{itemize}
                    \item Duplicates all threads in the new process.
                    \item Duplicates only the calling thread in the new process.
                \end{itemize}
            \item Choice depends on application:
                \begin{itemize}
                    \item If \texttt{exec()} is called immediately after forking, duplicating only the calling thread is appropriate.
                    \item If the new process does not call \texttt{exec()}, duplicating all threads is appropriate.
                \end{itemize}
        \end{itemize}
    \item \textbf{\texttt{exec()}} behavior:
        \begin{itemize}
            \item If a thread invokes \texttt{exec()}, the specified program replaces the entire process, including all threads.
        \end{itemize}
\end{itemize}

\subsection{Signal Handling}
\begin{itemize}
    \item A \textbf{signal} in UNIX systems notifies a process of an event.
    \item Signals can be synchronous or asynchronous.
    \item \textbf{Signal Pattern}: Generated $\rightarrow$ Delivered $\rightarrow$ Handled.
    \item \textbf{Synchronous Signals}: Generated by an event within the running process (e.g., illegal memory access, division by zero); delivered to the same process that caused the signal.
    \item \textbf{Asynchronous Signals}: Generated by an event external to the running process (e.g., \texttt{<control><C>}, timer expiration); typically sent to another process.
    \item \textbf{Signal Handlers}:
        \begin{itemize}
            \item \textbf{Default signal handler}: Kernel-provided handler for each signal.
            \item \textbf{User-defined signal handler}: Overrides the default action to handle the signal.
        \end{itemize}
    \item \textbf{Signal Delivery in Multithreaded Programs}: More complex than single-threaded. Options include:
        \begin{enumerate}
            \item Deliver to the thread to which the signal applies.
            \item Deliver to every thread in the process.
            \item Deliver to certain threads in the process.
            \item Assign a specific thread to receive all signals for the process.
        \end{enumerate}
    \item Synchronous signals are delivered to the causing thread. Asynchronous signals may be sent to all threads (e.g., process termination) or to the first non-blocking thread.
    \item \textbf{UNIX Functions}:
        \begin{itemize}
            \item \texttt{kill(pid\_t pid, int signal)}: Delivers a signal to a specified process.
            \item \texttt{pthread\_kill(pthread\_t tid, int signal)}: Delivers a signal to a specified Pthread.
        \end{itemize}
    \item \textbf{Windows Emulation}: Windows does not explicitly support signals but emulates them using \textbf{asynchronous procedure calls} (APCs). An APC is delivered to a particular thread.
\end{itemize}

\subsection{Thread Cancellation}
\begin{itemize}
    \item \textbf{Thread cancellation}: Terminating a \textbf{target thread} before it has completed.
    \item \textbf{Scenarios}:
        \begin{itemize}
            \item Multiple threads searching a database; one finds result, others are canceled.
            \item User stops a web page from loading.
        \end{itemize}
    \item \textbf{Cancellation Scenarios}:
        \begin{itemize}
            \item \textbf{Asynchronous cancellation}: One thread immediately terminates the target thread.
            \item \textbf{Deferred cancellation}: The target thread periodically checks whether it should terminate, allowing orderly termination.
        \end{itemize}
    \item \textbf{Difficulties with Cancellation}:
        \begin{itemize}
            \item Resources allocated to canceled thread may not be fully reclaimed.
            \item Data shared with other threads may be left in an inconsistent state (especially with asynchronous cancellation).
        \end{itemize}
    \item \textbf{Pthreads Cancellation}:
        \begin{itemize}
            \item Initiated with \texttt{pthread\_cancel(tid)}. This is a request; actual cancellation depends on target thread's setup.
            \item Supports three cancellation modes (state and type):
                \begin{itemize}
                    \item \textbf{State}: Enabled or Disabled.
                    \item \textbf{Type}: Deferred or Asynchronous.
                \end{itemize}
            \item Default type is deferred cancellation.
            \item Cancellation occurs at a \textbf{cancellation point} (e.g., blocking system calls like \texttt{read()}).
            \item \texttt{pthread\_testcancel()}: Function to explicitly establish a cancellation point.
            \item \textbf{Cleanup handler}: A function invoked if a thread is canceled, allowing resource release before termination.
            \item Asynchronous cancellation is generally not recommended in Pthreads.
        \end{itemize}
    \item \textbf{Java Thread Cancellation}:
        \begin{itemize}
            \item Similar to deferred cancellation.
            \item Invoke \texttt{interrupt()} method on target thread to set its interruption status to true.
            \item Thread checks its interruption status using \texttt{isInterrupted()} method.
        \end{itemize}
\end{itemize}

\subsection{Thread-Local Storage (TLS)}
\begin{itemize}
    \item \textbf{Thread-local storage (TLS)}: Data unique to each thread, even though threads typically share process data.
    \item \textbf{Purpose}: When each thread needs its own copy of certain data (e.g., unique transaction ID for each transaction-processing thread).
    \item \textbf{Distinction from Local Variables}: TLS data are visible across function invocations, unlike local variables.
    \item Useful when thread creation is not controlled by the developer (e.g., thread pools).
    \item Similar to \texttt{static} data, but unique per thread (often declared as \texttt{static}).
    \item \textbf{Support}: Most thread libraries and compilers provide TLS support.
        \begin{itemize}
            \item Java: \texttt{ThreadLocal<T>} class with \texttt{set()} and \texttt{get()} methods.
            \item Pthreads: \texttt{pthread\_key\_t} for thread-specific keys to access TLS data.
            \item C\#: \texttt{[ThreadStatic]} attribute.
            \item \texttt{gcc} compiler: \texttt{\_thread} storage class keyword.
        \end{itemize}
\end{itemize}

\subsection{Scheduler Activations}
\begin{itemize}
    \item Concerns communication between the kernel and the thread library, especially for many-to-many and two-level models.
    \item Aims to dynamically adjust the number of kernel threads for optimal performance.
    \item \textbf{Lightweight Process (LWP)}: An intermediate data structure between user and kernel threads.
        \begin{itemize}
            \item Appears as a virtual processor to the user-thread library.
            \item User threads are scheduled onto LWPs.
            \item Each LWP is attached to a kernel thread, which the OS schedules on physical processors.
            \item If a kernel thread (and thus its LWP) blocks, the user-level thread also blocks.
            \item Applications may require multiple LWPs for I/O-intensive tasks.
        \end{itemize}
    \item \textbf{Scheduler Activation}: A communication scheme between user-thread library and kernel.
        \begin{itemize}
            \item Kernel provides LWPs to the application.
            \item Kernel informs application about events via an \textbf{upcall}.
            \item \textbf{Upcall handler}: Function in the thread library that handles upcalls, running on a virtual processor.
            \item \textbf{Example Upcall}: When an application thread is about to block, the kernel makes an upcall, allocates a new virtual processor, and the upcall handler saves the blocking thread's state and schedules another thread. When the blocking event completes, another upcall informs the library, and the unblocked thread becomes eligible to run.
        \end{itemize}
\end{itemize}

\subsection*{Section glossary}
\rowcolors{2}{gray!10}{white}
\centering
\begin{tabular}{>{\raggedright}p{0.35\textwidth} >{\raggedright\arraybackslash}p{0.55\textwidth}}
\toprule
\textbf{Term} & \textbf{Definition} \\
\midrule
\textbf{signal} & In UNIX and other operating systems, a means used to notify a process that an event has occurred. \\
\textbf{default signal handler} & The signal handler that receives signals unless a user-defined signal handler is provided by a process. \\
\textbf{user-defined signal handler} & The signal handler created by a process to provide non-default signal handling. \\
\textbf{asynchronous procedure call (APC)} & A facility that enables a user thread to specify a function that is to be called when the user thread receives notification of a particular event. \\
\textbf{thread cancellation} & Termination of a thread before it has completed. \\
\textbf{cancellation point} & With deferred thread cancellation, a point in the code at which it is safe to terminate the thread. \\
\textbf{clean-up handler} & A function that allows any resources a thread has acquired to be released before the thread is terminated. \\
\textbf{thread-local storage (TLS)} & Data available only to a given thread. \\
\textbf{lightweight process (LWP)} & A virtual processor-like data structure allowing a user thread to map to a kernel thread. \\
\textbf{scheduler activation} & A threading method in which the kernel provides an application with a set of LWPs, and the application can schedule user threads onto an available virtual processor and receive upcalls from the kernel to be informed of certain events. \\
\textbf{upcall} & A threading method in which the kernel sends a signal to a process thread to communicate an event. \\
\textbf{upcall handler} & A function in a process that handles upcalls. \\
\bottomrule
\end{tabular}
\vspace{\baselineskip}
