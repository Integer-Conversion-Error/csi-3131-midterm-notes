\section{Operating-System Examples}\label{sec:4.7}

\subsection{Windows Threads}
\begin{itemize}
    \item A Windows application runs as a separate process, which can contain one or more threads.
    \item Windows uses the one-to-one mapping model, where each user-level thread maps to an associated kernel thread.
    \item \textbf{General Components of a Thread}:
        \begin{itemize}
            \item Thread ID (unique identifier).
            \item Register set (processor status).
            \item Program counter.
            \item User stack (for user mode) and kernel stack (for kernel mode).
            \item Private storage area (for run-time libraries and DLLs).
        \end{itemize}
    \item The register set, stacks, and private storage area constitute the \textbf{context} of the thread.
    \item \textbf{Primary Data Structures of a Thread}:
        \begin{itemize}
            \item \textbf{ETHREAD} (executive thread block): Contains a pointer to the owning process and the thread's starting routine address; also points to the corresponding KTHREAD.
            \item \textbf{KTHREAD} (kernel thread block): Includes scheduling and synchronization information, the kernel stack, and a pointer to the TEB.
            \item \textbf{TEB} (thread environment block): A user-space data structure containing the thread identifier, user-mode stack, and an array for thread-local storage.
        \end{itemize}
    \item ETHREAD and KTHREAD exist in kernel space (only accessible by the kernel); TEB is in user space.
\end{itemize}

\subsection{Linux Threads}
\begin{itemize}
    \item Linux provides the \texttt{fork()} system call for duplicating a process.
    \item Linux also provides the \texttt{clone()} system call for creating threads.
    \item Linux does not distinguish between processes and threads; it uses the term \textbf{task} to refer to a flow of control.
    \item \textbf{\texttt{clone()} System Call}:
        \begin{itemize}
            \item Passed a set of flags that determine the level of sharing between parent and child tasks.
            \item \textbf{Examples of flags}:
                \begin{itemize}
                    \item \texttt{CLONE\_FS}: Shares file-system information.
                    \item \texttt{CLONE\_VM}: Shares the same memory space.
                    \item \texttt{CLONE\_SIGHAND}: Shares signal handlers.
                    \item \texttt{CLONE\_FILES}: Shares the set of open files.
                \end{itemize}
            \item If these flags are set, \texttt{clone()} behaves like thread creation (parent shares most resources).
            \item If no flags are set, \texttt{clone()} provides functionality similar to \texttt{fork()} (no sharing).
        \end{itemize}
    \item \textbf{Task Representation in Linux Kernel}:
        \begin{itemize}
            \item A unique kernel data structure (\texttt{struct task\_struct}) exists for each task.
            \item This structure contains pointers to other data structures (e.g., open files, signal handling, virtual memory) rather than storing the data directly.
            \item When \texttt{fork()} is invoked, a new task is created with copies of all associated data structures.
            \item When \texttt{clone()} is invoked, the new task points to the parent's data structures, depending on the flags, allowing varying levels of sharing.
        \end{itemize}
    \item The flexibility of \texttt{clone()} extends to creating Linux containers, which are virtualized systems running in isolation under a single Linux kernel.
\end{itemize}

\subsection*{Section glossary}
\rowcolors{2}{gray!10}{white}
\centering
\begin{tabular}{>{\raggedright}p{0.35\textwidth} >{\raggedright\arraybackslash}p{0.55\textwidth}}
\toprule
\textbf{Term} & \textbf{Definition} \\
\midrule
\textbf{context} & When describing a thread, the state of its execution, including its register set, program counter, user stack, kernel stack, and private storage area. \\
\textbf{ETHREAD} & Executive thread block; a primary data structure for a Windows thread, existing in kernel space. \\
\textbf{KTHREAD} & Kernel thread block; a primary data structure for a Windows thread, existing in kernel space, containing scheduling and synchronization information. \\
\textbf{TEB} & Thread environment block; a primary data structure for a Windows thread, existing in user space, containing thread-specific information. \\
\textbf{task} & In Linux, a term referring to a flow of control within a program, encompassing both processes and threads. \\
\textbf{clone()} & The Linux system call used to create threads, allowing varying levels of resource sharing with the parent task via flags. \\
\bottomrule
\end{tabular}
\vspace{\baselineskip}
