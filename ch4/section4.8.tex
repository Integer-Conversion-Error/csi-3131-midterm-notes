\section{Summary}\label{sec:4.8}

\begin{itemize}
    \item A thread is a basic unit of CPU utilization; threads belonging to the same process share many process resources, including code and data.
    \item There are four primary benefits to multithreaded applications: (1) responsiveness, (2) resource sharing, (3) economy, and (4) scalability.
    \item \textbf{Concurrency} exists when multiple threads are making progress. \textbf{Parallelism} exists when multiple threads are making progress simultaneously. On a single-CPU system, only concurrency is possible; parallelism requires a multicore system with multiple CPUs.
    \item Designing multithreaded applications presents several challenges, including dividing and balancing work, dividing data between threads, identifying data dependencies, and the increased difficulty of testing and debugging.
    \item \textbf{Data parallelism} distributes subsets of the same data across different computing cores and performs the same operation on each core. \textbf{Task parallelism} distributes tasks (not data) across multiple cores, with each task running a unique operation.
    \item User applications create user-level threads, which must be mapped to kernel threads for execution on a CPU. Common mapping models include many-to-one, one-to-one, and many-to-many.
    \item A \textbf{thread library} provides an API for creating and managing threads. Key thread libraries include Windows, Pthreads, and Java threading. Windows is specific to Windows systems, Pthreads is for POSIX-compatible systems (UNIX, Linux, macOS), and Java threads run on any system supporting a Java Virtual Machine.
    \item \textbf{Implicit threading} involves identifying tasks (not threads) and allowing languages or API frameworks to create and manage threads. Approaches include thread pools, fork-join frameworks, and Grand Central Dispatch. Implicit threading is increasingly common for developing concurrent and parallel applications.
    \item Threads can be terminated using either \textbf{asynchronous cancellation} (immediate termination) or \textbf{deferred cancellation} (target thread periodically checks for termination, allowing orderly shutdown). Deferred cancellation is generally preferred due to issues with resource reclamation and data consistency in asynchronous cancellation.
    \item Unlike many other operating systems, Linux does not distinguish between processes and threads, referring to both as \textbf{tasks}. The Linux \texttt{clone()} system call can create tasks that behave more like processes or threads, depending on the flags passed for resource sharing.
\end{itemize}
