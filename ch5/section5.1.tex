\section{CPU Scheduling}\label{sec:5.1}

\subsection{Basic Concepts}
\begin{itemize}
    \item CPU scheduling is fundamental to multiprogrammed operating systems, maximizing CPU utilization by switching the CPU among processes.
    \item On modern operating systems, kernel-level threads are scheduled, though "process scheduling" and "thread scheduling" are often used interchangeably.
    \item A process executes on a CPU's core; general terminology of scheduling a process to "run on a CPU" implies running on a core.
\end{itemize}

\subsection{Basic concepts}
\begin{itemize}
    \item In a single CPU core system, only one process runs at a time; others wait. Multiprogramming aims to keep a process running at all times to maximize CPU utilization.
    \item When one process waits (e.g., for I/O), the OS gives the CPU to another process. This continues, ensuring productive use of waiting time.
    \item On multicore systems, this concept extends to all processing cores.
    \item CPU scheduling is a fundamental OS function, central to its design.
\end{itemize}

\subsubsection{CPU-I/O burst cycle}
\begin{itemize}
    \item Process execution consists of a \textbf{cycle} of CPU execution and I/O wait, alternating between \textbf{CPU burst} and \textbf{I/O burst}.
    \item The final CPU burst ends with a system request to terminate execution.
    \item Durations of CPU bursts vary but tend to have an exponential or hyperexponential frequency curve (many short, few long bursts).
    \item I/O-bound programs have many short CPU bursts; CPU-bound programs have a few long CPU bursts. This distribution is important for CPU-scheduling algorithms.
\end{itemize}

\subsubsection{CPU scheduler}
\begin{itemize}
    \item When the CPU becomes idle, the operating system selects a process from the ready queue to execute.
    \item The \textbf{CPU scheduler} performs this selection and allocates the CPU to the chosen process.
    \item The ready queue is not necessarily FIFO; it can be a FIFO queue, priority queue, tree, or unordered linked list.
    \item Records in queues are generally process control blocks (PCBs).
\end{itemize}

\subsubsection{Preemptive and nonpreemptive scheduling}
\begin{itemize}
    \item CPU-scheduling decisions occur under four circumstances:
    \begin{enumerate}
        \item Process switches from running to waiting state (e.g., I/O request, \texttt{wait()} for child termination).
        \item Process switches from running to ready state (e.g., interrupt occurs).
        \item Process switches from waiting to ready state (e.g., I/O completion).
        \item Process terminates.
    \end{enumerate}
    \item For circumstances 1 and 4, no scheduling choice; a new process must be selected. Choices exist for 2 and 3.
    \item \textbf{Nonpreemptive} or \textbf{cooperative} scheduling: CPU allocated to a process until it releases it (terminates or switches to waiting state).
    \item \textbf{Preemptive} scheduling: CPU can be taken away from a process. Most modern OS (Windows, macOS, Linux, UNIX) use preemptive scheduling.
    \item Preemptive scheduling can cause race conditions with shared data (e.g., one process updates, is preempted, second process reads inconsistent data).
    \item Preemption affects OS kernel design:
        \begin{itemize}
            \item Nonpreemptive kernel: waits for system call completion or process block before context switch, ensuring simple kernel structure and consistent data. Poor for real-time computing.
            \item Preemptive kernel: requires mechanisms (e.g., mutex locks) to prevent race conditions when accessing shared kernel data structures. Most modern OS are fully preemptive in kernel mode.
        \end{itemize}
    \item Sections of code affected by interrupts must be guarded (e.g., disable interrupts at entry, reenable at exit) to prevent simultaneous use and data loss.
\end{itemize}

\subsubsection{Dispatcher}
\begin{itemize}
    \item The \textbf{dispatcher} is a component of the CPU-scheduling function.
    \item It gives control of the CPU's core to the process selected by the CPU scheduler.
    \item Functions include:
        \begin{itemize}
            \item Switching context from one process to another.
            \item Switching to user mode.
            \item Jumping to the proper location in the user program to resume that program.
        \end{itemize}
    \item The dispatcher should be fast, as it's invoked during every context switch.
    \item \textbf{Dispatch latency}: time for the dispatcher to stop one process and start another.
    \item Context switch frequency can be observed using tools like \texttt{vmstat} on Linux (system-wide) or \texttt{/proc} file system (per-process).
    \item \textbf{Voluntary context switch}: process gives up CPU (e.g., blocking for I/O).
    \item \textbf{Nonvoluntary context switch}: CPU taken from process (e.g., time slice expired, preempted by higher-priority process).
\end{itemize}

\subsection*{Section glossary}
\rowcolors{2}{gray!10}{white}
\centering
\begin{tabular}{>{\raggedright}p{0.35\textwidth} >{\raggedright\arraybackslash}p{0.55\textwidth}}
\toprule
\textbf{Term} & \textbf{Definition} \\
\midrule
\textbf{cycle} & Repeating loop \\
\textbf{CPU burst} & Scheduling process state in which the process executes on CPU. \\
\textbf{I/O burst} & Scheduling process state in which the CPU performs I/O. \\
\textbf{CPU scheduler} & Kernel routine that selects a thread from the threads that are ready to execute and allocates a core to that thread. \\
\textbf{nonpreemptive} & Under nonpreemptive scheduling, once a core has been allocated to a thread the thread keeps the core until it releases the core either by terminating or by switching to the waiting state. \\
\textbf{cooperative} & A form of scheduling in which threads voluntarily move from the running state. \\
\textbf{preemptive} & A form of scheduling in which processes or threads are involuntarily moved from the running state (by for example a timer signaling the kernel to allow the next thread to run). \\
\textbf{dispatcher} & The kernel routine that gives control of a core to the thread selected by the scheduler. \\
\textbf{dispatch latency} & The time it takes for the dispatcher to stop one thread and start another running. \\
\bottomrule
\end{tabular}
\vspace{\baselineskip}
