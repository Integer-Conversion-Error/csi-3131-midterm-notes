\section{Scheduling criteria}\label{sec:5.2}

\subsection{Overview}
\begin{itemize}
    \item Different CPU-scheduling algorithms have varying properties, favoring certain process classes.
    \item The choice of algorithm depends on the desired characteristics for comparison.
\end{itemize}

\subsection{Criteria for Comparison}
\begin{itemize}
    \item \textbf{CPU utilization}: Keep the CPU as busy as possible (ideally 40-90\% in real systems).
    \item \textbf{Throughput}: Number of processes completed per unit time.
    \item \textbf{Turnaround time}: Total time from process submission to completion (includes waiting in ready queue, CPU execution, and I/O).
    \item \textbf{Waiting time}: Total time a process spends waiting in the ready queue.
    \item \textbf{Response time}: Time from request submission until the first response is produced (for interactive systems).
\end{itemize}

\subsection{Optimization Goals}
\begin{itemize}
    \item Maximize CPU utilization and throughput.
    \item Minimize turnaround time, waiting time, and response time.
    \item Often, the goal is to optimize the average measure, but sometimes minimum or maximum values are preferred (e.g., minimizing maximum response time for guaranteed service).
    \item For interactive systems, minimizing the variance in response time may be more important than minimizing the average.
\end{itemize}

\subsection*{Section glossary}
\rowcolors{2}{gray!10}{white}
\centering
\begin{tabular}{>{\raggedright}p{0.35\textwidth} >{\raggedright\arraybackslash}p{0.55\textwidth}}
\toprule
\textbf{Term} & \textbf{Definition} \\
\midrule
\textbf{throughput} & Generally, the amount of work done over time. In scheduling, the number of threads completed per unit time. \\
\bottomrule
\end{tabular}
\vspace{\baselineskip}
