\section{Scheduling algorithms}\label{sec:5.3}

\subsection{Overview}
\begin{itemize}
    \item CPU scheduling involves deciding which process in the ready queue is allocated the CPU's core.
    \item Various CPU-scheduling algorithms exist.
    \item These algorithms are described in the context of a single processing core, capable of running one process at a time. Multiprocessor scheduling is discussed in Section 5.5.
    \item For demonstration, each process is associated with a CPU burst time, although in practice, this is not known in advance.
\end{itemize}

\subsection{First-come, first-served scheduling}
\begin{itemize}
    \item The simplest CPU-scheduling algorithm: \textbf{first-come, first-served} (\textbf{FCFS}).
    \item The process that requests the CPU first is allocated the CPU first.
    \item Implemented with a FIFO queue: process PCBs are linked to the tail, and the CPU is allocated to the process at the head.
    \item Average waiting time under FCFS is often long and can vary substantially with CPU burst times.
    \item \textbf{Convoy effect}: A CPU-bound process holds the CPU, causing I/O-bound processes to wait in the ready queue, leading to lower CPU and device utilization.
    \item FCFS is nonpreemptive; once allocated, a process keeps the CPU until it terminates or requests I/O. This is problematic for interactive systems.
\item A \textbf{Gantt chart} is a bar chart that illustrates a particular schedule, including the start and finish times of each of the participating processes.
\end{itemize}

\subsection{Shortest-job-first scheduling}
\begin{itemize}
    \item The \textbf{shortest-job-first} (\textbf{SJF}) scheduling algorithm associates each process with the length of its next CPU burst.
    \item The CPU is assigned to the process with the smallest next CPU burst. FCFS breaks ties.
    \item More accurately called the \textbf{shortest-next-CPU-burst} algorithm.
    \item SJF is provably optimal, providing the minimum average waiting time for a given set of processes.
    \item Cannot be implemented at the CPU scheduling level directly, as the length of the next CPU burst is unknown.
    \item Can be approximated by predicting the next CPU burst using an \textbf{exponential average} of previous CPU bursts:
    \[{\tau _{n + 1}} = \alpha {t_n} + \left( {1 - \alpha } \right){\tau _n}{\rm{.}}\]
    \begin{itemize}
        \item $t_n$: length of the $n$th CPU burst.
        \item $\tau_{n+1}$: predicted value for the next CPU burst.
        \item $\alpha$: parameter (0 $\le \alpha \le$ 1) controlling the weight of recent vs. past history.
        \item If $\alpha = 0$, $\tau_{n+1} = \tau_n$ (recent history has no effect).
        \item If $\alpha = 1$, $\tau_{n+1} = t_n$ (only most recent CPU burst matters).
        \item Commonly, $\alpha = 1/2$, weighting recent and past history equally.
    \end{itemize}
    \item SJF can be preemptive or nonpreemptive.
    \item Preemptive SJF is called \textbf{shortest-remaining-time-first} scheduling. It preempts the current process if a new process has a shorter remaining CPU burst.
\end{itemize}

\subsection{Round-robin scheduling}
\begin{itemize}
    \item The \textbf{round-robin} (\textbf{RR}) scheduling algorithm is similar to FCFS but includes preemption.
    \item A small unit of time, called a \textbf{time quantum} or \textbf{time slice} (typically 10-100 milliseconds), is defined.
    \item The ready queue is treated as a circular queue. The CPU scheduler allocates the CPU to each process for up to 1 time quantum.
    \item If a process finishes its CPU burst before the time quantum expires, it voluntarily releases the CPU.
    \item If a process's CPU burst is longer than 1 time quantum, it is preempted by a timer interrupt and moved to the tail of the ready queue.
    \item RR is a preemptive scheduling algorithm.
    \item If there are $n$ processes and a time quantum of $q$, each process gets $1/n$ of the CPU time in chunks of at most $q$ time units. Each process waits no longer than $(n-1)q$ time units.
    \item Performance depends heavily on time quantum size:
        \begin{itemize}
            \item Large quantum: RR degenerates to FCFS.
            \item Small quantum: results in many context switches, increasing overhead.
        \end{itemize}
    \item Time quantum should be large relative to context-switch time (e.g., context-switch time < 10\% of time quantum).
    \item Average turnaround time does not necessarily improve with increasing time quantum; it improves if most processes finish their CPU burst within one quantum.
    \item Rule of thumb: 80\% of CPU bursts should be shorter than the time quantum.
\end{itemize}

\subsection{Priority scheduling}
\begin{itemize}
    \item The SJF algorithm is a special case of the general \textbf{priority-scheduling} algorithm.
    \item A priority is associated with each process; the CPU is allocated to the process with the highest priority.
    \item Equal-priority processes are scheduled in FCFS order.
    \item SJF is a priority algorithm where priority is the inverse of the (predicted) next CPU burst (shorter burst = higher priority).
    \item Priorities can be defined internally (e.g., time limits, memory requirements) or externally (e.g., importance of process, payment).
    \item Priority scheduling can be preemptive (new higher-priority process preempts current) or nonpreemptive (new higher-priority process goes to head of queue).
    \item Major problem: \textbf{indefinite blocking} or \textbf{starvation}, where low-priority processes may wait indefinitely for the CPU.
    \item Solution to starvation: \textbf{aging}, which gradually increases the priority of processes that wait for a long time.
    \item Can combine RR and priority scheduling: highest-priority process executes, and processes with the same priority use RR.
\end{itemize}

\subsection{Multilevel queue scheduling}
\begin{itemize}
    \item \textbf{Multilevel queue} scheduling partitions the ready queue into several separate queues.
    \item Each queue may have its own scheduling algorithm (e.g., foreground/interactive processes with RR, background/batch processes with FCFS).
    \item Scheduling among queues is commonly fixed-priority preemptive scheduling (e.g., real-time queue has absolute priority over interactive queue).
    \item Processes are typically permanently assigned to a queue.
    \item Can also time-slice among queues (e.g., foreground queue gets 80\% CPU time, background gets 20\%).
\end{itemize}

\subsection{Multilevel feedback queue scheduling}
\begin{itemize}
    \item The \textbf{multilevel feedback queue} scheduling algorithm allows a process to move between queues.
    \item Separates processes by CPU burst characteristics: processes using too much CPU time are moved to lower-priority queues.
    \item This keeps I/O-bound and interactive processes (short CPU bursts) in higher-priority queues.
    \item Aging is implemented by moving processes that wait too long in lower-priority queues to higher-priority queues, preventing starvation.
    \item Example: Queue 0 (8ms quantum), Queue 1 (16ms quantum), Queue 2 (FCFS). Processes start in Queue 0, move to Queue 1 if not finished, then to Queue 2.
    \item Defined by: number of queues, scheduling algorithm for each queue, methods for upgrading/demoting processes, and method for initial queue entry.
    \item Most general and configurable CPU-scheduling algorithm, but also the most complex to define optimally.
\end{itemize}

\subsection*{Section glossary}
\rowcolors{2}{gray!10}{white}
\centering
\begin{tabular}{>{\raggedright}p{0.35\textwidth} >{\raggedright\arraybackslash}p{0.55\textwidth}}
\toprule
\textbf{Term} & \textbf{Definition} \\
\midrule
\textbf{First-come first-served (FCFS)} & The simplest scheduling algorithm - the thread that requests a core first is allocated the core first, and others following get cores in the order of their requests. \\
\textbf{Gantt chart} & A bar chart that is used in the text to illustrate a schedule. \\
\textbf{convoy effect} & A scheduling phenomenon in which threads wait for the one thread to get off a core, causing overall device and CPU utilization to be suboptimal. \\
\textbf{shortest-job-first (SJF)} & A scheduling algorithm that associates with each thread the length of the threads next CPU burst and schedules the shortest first. \\
\textbf{exponential average} & A calculation used in scheduling to estimate the next CPU burst time based on the previous burst times (with exponential decay on older values). \\
\textbf{shortest-remaining-time-first (SJRF)} & Similar to SJF, this scheduling algorithm optimizes for the shortest remaining time until thread completion. \\
\textbf{round-robin (RR)} & A scheduling algorithm that is designed especially for time-sharing systems - similar to FCFS scheduling, but preemption is added to enable the system to switch between threads. \\
\textbf{time quantum} & A small unit of time used by scheduling algorithms as a basis for determining when to preempt a thread from the CPU to allow another to run. \\
\textbf{time slice} & See time quantum. \\
\textbf{priority-scheduling} & A scheduling algorithm in which a priority is associated with each thread and the free CPU core is allocated to the thread with the highest priority. \\
\textbf{infinite blocking} & See starvation. \\
\textbf{starvation} & A scheduling risk in which a thread that is ready to run never gets put onto the CPU due to the scheduling algorithm - it is starved for CPU time. \\
\textbf{aging} & Aging is a solution to scheduling starvation and involves gradually increasing the priority of threads as they wait for CPU time. \\
\textbf{multilevel queue} & A multilevel queue scheduling algorithm partitions the ready queue into several separate queues. \\
\textbf{foreground} & A thread that is interactive and has input directed to it (such as a window currently selected as active or a terminal window that is currently selected to receive input). \\
\textbf{background} & A thread that is not currently interactive (has no interactive input directed to it) such as one in a batch job or not currently being used by a user. \\
\textbf{multilevel feedback queue} & The multilevel feedback queue scheduling algorithm that allows a process to move between queues. \\
\bottomrule
\end{tabular}
\vspace{\baselineskip}
