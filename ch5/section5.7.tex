\section{Operating-system examples}\label{sec:5.7}

\subsection{Overview}
\begin{itemize}
    \item This section describes the scheduling policies of Linux, Windows, and Solaris operating systems.
    \item The term "process scheduling" is used generally, referring to kernel threads (Solaris, Windows) or tasks (Linux).
\end{itemize}

\subsection{Example: Linux scheduling}
\begin{itemize}
    \item Linux scheduling history:
        \begin{itemize}
            \item Prior to Version 2.5: Traditional UNIX scheduling (poor SMP support, poor performance with many runnable processes).
            \item Version 2.5: Introduced O(1) scheduler (constant time regardless of tasks), improved SMP support, processor affinity, and load balancing.
            \item Version 2.6.23: \textbf{Completely Fair Scheduler} (CFS) became default.
        \end{itemize}
    \item Linux scheduling is based on \textbf{scheduling classes}, each with a specific priority.
    \item Scheduler selects the highest-priority task from the highest-priority scheduling class.
    \item Standard Linux kernels implement two classes: default (CFS) and real-time.
    \item CFS scheduler:
        \begin{itemize}
            \item Assigns a proportion of CPU time to each task based on its \textbf{nice value} (range -20 to +19; lower value = higher priority).
            \item Uses a \textbf{targeted latency}: an interval during which every runnable task should run at least once.
            \item Maintains \textbf{virtual run time} (\texttt{vruntime}) for each task, which decays based on priority (lower priority = higher decay rate).
            \item Selects the task with the smallest \texttt{vruntime} to run next.
            \item Higher-priority tasks can preempt lower-priority tasks.
            \item Handles I/O-bound vs. CPU-bound tasks by giving I/O-bound tasks higher priority due to their lower \texttt{vruntime} (they run for shorter bursts).
            \item Uses a red-black tree (balanced binary search tree) to store runnable tasks, keyed by \texttt{vruntime}. The leftmost node is the highest priority.
        \end{itemize}
    \item Linux real-time scheduling:
        \begin{itemize}
            \item Uses POSIX standard (\texttt{SCHED\_FIFO} or \texttt{SCHED\_RR}).
            \item Real-time tasks run at higher priority than normal tasks.
            \item Priority ranges: 0-99 for real-time (static), 100-139 for normal (based on nice values). Lower numeric value = higher priority.
        \end{itemize}
    \item CFS load balancing:
        \begin{itemize}
            \item Equalizes load among processing cores, is NUMA-aware, and minimizes thread migration.
            \item Load of a thread: combination of priority and average CPU utilization.
            \item Uses hierarchical \textbf{scheduling domains}: sets of CPU cores balanced against each other based on shared resources (e.g., L1, L2, L3 caches, \textbf{NUMA node}s).
            \item Balances loads within domains, starting at the lowest level. Reluctant to migrate threads between NUMA nodes to avoid memory latency penalties.
        \end{itemize}
\end{itemize}

\subsection{Example: Windows scheduling}
\begin{itemize}
    \item Windows uses a priority-based, preemptive scheduling algorithm.
    \item The \textbf{dispatcher} handles scheduling.
    \item A thread runs until preempted by higher priority, terminates, time quantum ends, or calls a blocking system call.
    \item 32-level priority scheme:
        \begin{itemize}
            \item Variable class: priorities 1-15.
            \item \textbf{Real-time class}: priorities 16-31.
            \item Priority 0: memory management thread.
        \end{itemize}
    \item Dispatcher uses a queue for each priority; executes an \textbf{idle thread} if no ready thread is found.
    \item Windows API priority classes for processes: \texttt{IDLE\_PRIORITY\_CLASS}, \texttt{BELOW\_NORMAL\_PRIORITY\_CLASS}, \texttt{NORMAL\_PRIORITY\_CLASS}, \texttt{ABOVE\_NORMAL\_PRIORITY\_CLASS}, \texttt{HIGH\_PRIORITY\_CLASS}, \texttt{REALTIME\_PRIORITY\_CLASS}.
    \item Thread priority is based on its process's priority class and its relative priority within that class (\texttt{IDLE}, \texttt{LOWEST}, \texttt{BELOW\_NORMAL}, \texttt{NORMAL}, \texttt{ABOVE\_NORMAL}, \texttt{HIGHEST}, \texttt{TIME\_CRITICAL}).
    \item Base priority: default is \texttt{NORMAL} relative priority for its class. Can be modified by \texttt{SetThreadPriority()}.
    \item Dynamic priority adjustments:
        \begin{itemize}
            \item Variable-priority threads: priority lowered when quantum expires (never below base priority).
            \item Priority boosted when released from a wait operation (amount depends on what it was waiting for).
            \item Foreground process (currently selected window) receives increased scheduling quantum (e.g., 3x longer).
        \end{itemize}
    \item \textbf{User-mode scheduling} (\textbf{UMS}): Allows applications to create and manage threads independently of the kernel (Windows 7+).
        \begin{itemize}
            \item More efficient for many threads, no kernel intervention.
            \item Predecessor: \textbf{fibers} (limited use due to shared thread environment block).
            \item UMS provides each user-mode thread with its own thread context.
            \item Not for direct programmer use; built upon by language libraries (e.g., Microsoft \textbf{Concurrency Runtime} (ConcRT) for C++).
        \end{itemize}
    \item Multiprocessor scheduling in Windows:
        \begin{itemize}
            \item Attempts to schedule threads on optimal cores, maintaining preferred/most recent processor.
            \item Uses \textbf{SMT sets} (sets of logical processors on the same CPU core, e.g., hyper-threaded cores).
            \item Assigns an \textbf{ideal processor} (preferred processor) to each thread, incrementing a seed value for new threads to distribute load across logical processors and SMT sets.
        \end{itemize}
\end{itemize}

\subsection{Example: Solaris scheduling}
\begin{itemize}
    \item Solaris uses priority-based thread scheduling with six classes: Time sharing (TS), Interactive (IA), Real time (RT), System (SYS), Fair share (FSS), Fixed priority (FP).
    \item Each class has different priorities and scheduling algorithms.
    \item Default class: Time sharing. Dynamically alters priorities and time slices using a multilevel feedback queue (inverse relationship: higher priority = smaller time slice).
    \item Interactive class: Same as time-sharing, but gives windowing applications higher priority.
    \item Dispatch table for TS/IA threads includes:
        \begin{itemize}
            \item Priority: Class-dependent (higher number = higher priority).
            \item Time quantum: Inverse relationship with priority.
            \item Time quantum expired: New (lower) priority for CPU-intensive threads.
            \item Return from sleep: Boosted priority for threads returning from I/O wait.
        \end{itemize}
    \item Real-time class: Highest priority; real-time processes run before any other class, guaranteeing bounded response time. Few processes belong to this class.
    \item System class: For kernel threads (e.g., scheduler, paging daemon); static priorities, reserved for kernel use.
    \item Fixed-priority class (Solaris 9+): Same priority range as time-sharing, but priorities are not dynamically adjusted.
    \item Fair-share class (Solaris 9+): Uses CPU \textbf{shares} (entitlement to CPU resources) instead of priorities, allocated to a \textbf{project} (set of processes).
    \item Scheduler converts class-specific priorities to global priorities and selects the highest global priority thread.
    \item Selected thread runs until it blocks, uses its time slice, or is preempted.
    \item Multiple threads with same priority use a round-robin queue.
    \item Solaris traditionally used many-to-many model, switched to one-to-one model with Solaris 9.
\end{itemize}

\subsection*{Section glossary}
\rowcolors{2}{gray!10}{white}
\centering
\begin{tabular}{>{\raggedright}p{0.35\textwidth} >{\raggedright\arraybackslash}p{0.55\textwidth}}
\toprule
\textbf{Term} & \textbf{Definition} \\
\midrule
\textbf{scheduling classes} & Scheduling in the Linux system is based on scheduling classes - each class is assigned a specific priority. \\
\textbf{nice value} & Nice values range from -20 to +19, where a numerically lower nice value indicates a higher relative scheduling priority. \\
\textbf{targeted latency} & Targeted latency is an interval of time during which every runnable thread should run at least once. \\
\textbf{virtual run time} & A Linux scheduling aspect in which it records how long each task has run by maintaining the virtual run time of each task. \\
\textbf{scheduling domain} & A set of CPU cores that can be balanced against one another. \\
\textbf{NUMA node} & One or more cores (for example, cores that share a cache) that are grouped together as a scheduling entity for affinity or other uses. \\
\textbf{real-time class} & A scheduling class that segreates real-time threads from other threads to schedule them separate and provide them with their needed priority. \\
\textbf{idle thread} & In some operating systems, If no ready thread is found, the dispatcher will execute a special thread called the idle thread that runs on the CPU until the CPU is needed for some other activity. \\
\textbf{User-Mode Scheduling (UMS)} & A Microsoft Windows 7 feature that allows applications to create and manage threads independently of the kernel. \\
\textbf{fibers} & The Microsoft Windows predecessor to User-Mode Scheduling allowed several user-mode threads (fibers) to be mapped to a single kernel thread. \\
\textbf{Concurrency Runtime (ConcRT)} & A Microsoft Windows concurrent programming framework for C++ that is designed for task-based parallelism on multicore processors. \\
\textbf{shares} & A scheduling concept in which CPU shares instead of priorities are used to make scheduling decisions, providing an entitlement to CPU time for a process or a set of processes. \\
\textbf{project} & A Solaris scheduling concept in which processes are grouped into a project and the project is scheduled. \\
\bottomrule
\end{tabular}
\vspace{\baselineskip}
