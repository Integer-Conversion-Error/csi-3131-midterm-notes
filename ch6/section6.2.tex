\section{The critical-section problem}
\begin{itemize}
    \item \textbf{Critical-section problem}: Designing a protocol for $n$ processes ($P_0, \dots, P_{n-1}$) to cooperatively share data and synchronize activity.
    \item Each process has a \textbf{critical section}:
    \begin{itemize}
        \item Code segment where shared data is accessed and updated.
        \item \textbf{Key rule}: Only one process allowed in its critical section at any given time (mutual exclusion).
    \end{itemize}
    \item \textbf{Protocol components for each process:}
    \begin{itemize}
        \item \textbf{Entry section}: Code to request permission to enter the critical section.
        \item \textbf{Critical section}: The shared data access/update code.
        \item \textbf{Exit section}: Code to cleanly exit the critical section.
        \item \textbf{Remainder section}: All other code.
    \end{itemize}
    \item \textbf{General process structure:}
\begin{verbatim}
while (true) {
 
    entry section
 
        critical section
 
    exit section
 
        remainder section
 
}
\end{verbatim}
    \item \textbf{Three requirements for a critical-section problem solution:}
    \begin{enumerate}
        \item \textbf{Mutual exclusion}: If $P_i$ is in its critical section, no other process can be.
        \item \textbf{Progress}:
        \begin{itemize}
            \item If no process is in its critical section, and some wish to enter.
            \item Only processes not in their remainder sections can participate in deciding who enters next.
            \item Selection cannot be postponed indefinitely.
        \end{itemize}
        \item \textbf{Bounded waiting}:
        \begin{itemize}
            \item A limit exists on how many times other processes can enter their critical sections.
            \item This limit applies after a process requests entry and before its request is granted.
        \end{itemize}
    \end{enumerate}
    \item \textbf{Assumptions}:
    \begin{itemize}
        \item Each process executes at a nonzero speed.
        \item No assumptions about the relative speeds of processes.
    \end{itemize}
    \item \textbf{Kernel Code and Race Conditions}:
    \begin{itemize}
        \item Many kernel-mode processes are active in the OS.
        \item \textbf{Kernel code} is susceptible to race conditions.
        \item \textbf{Example 1}: Kernel data structure for open files.
        \begin{itemize}
            \item Modified when files are opened/closed (add/remove from list).
            \item Simultaneous file opening by two processes could cause a race condition on this list.
        \end{itemize}
        \item \textbf{Example 2}: Two processes using `fork()` system call.
        \begin{itemize}
            \item Race condition on `next\_available\_pid` kernel variable.
            \item Without mutual exclusion, same PID could be assigned to two processes.
        \end{itemize}
        \item \textbf{Other prone kernel data structures}: Memory allocation, process lists, interrupt handling.
        \item Kernel developers must ensure OS is free from such race conditions.
    \end{itemize}
    \item \textbf{Single-core vs. Multiprocessor Environments}:
    \begin{itemize}
        \item \textbf{Single-core}: Critical-section problem could be solved by preventing interrupts during shared variable modification.
        \begin{itemize}
            \item Ensures current instruction sequence executes without preemption.
            \item No unexpected modifications to shared variable.
        \end{itemize}
        \item \textbf{Multiprocessor}: This solution is less feasible.
        \begin{itemize}
            \item Disabling interrupts on multiprocessor is time-consuming (message to all processors).
            \item Delays critical section entry, decreases system efficiency.
            \item Affects system clock if updated by interrupts.
        \end{itemize}
    \end{itemize}
    \item \textbf{Approaches to handle critical sections in OS}:
    \begin{enumerate}
        \item \textbf{Preemptive kernels}:
        \begin{itemize}
            \item Allows a process to be preempted while running in kernel mode.
            \item Must be carefully designed for shared kernel data to be race-condition free.
            \item Especially difficult for SMP architectures (two kernel-mode processes can run simultaneously).
            \item \textbf{Advantages}: More responsive (less risk of long kernel-mode runs), more suitable for real-time programming (real-time process can preempt kernel process).
        \end{itemize}
        \item \textbf{Nonpreemptive kernels}:
        \begin{itemize}
            \item Does not allow a process in kernel mode to be preempted.
            \item Kernel-mode process runs until it exits kernel mode, blocks, or voluntarily yields CPU.
            \item Essentially free from race conditions on kernel data structures (only one process active in kernel at a time).
        \end{itemize}
    \end{enumerate}
\end{itemize}

\subsection{Section glossary}
\rowcolors{2}{gray!10}{white}
\centering
\begin{tabular}{>{\raggedright}p{0.35\textwidth} >{\raggedright\arraybackslash}p{0.55\textwidth}}
\toprule
\textbf{Term} & \textbf{Definition} \\
\midrule
\textbf{critical section} & A section of code responsible for changing data that must only be executed by one thread or process at a time to avoid a race condition. \\
\textbf{entry section} & The section of code within a process that requests permission to enter its critical section. \\
\textbf{exit section} & The section of code within a process that cleanly exits the critical section. \\
\textbf{remainder section} & Whatever code remains to be processed after the critical and exit sections. \\
\textbf{preemptive kernel} & A type of kernel that allows a process to be preempted while it is running in kernel mode. \\
\textbf{nonpreemptive kernels} & A type of kernel that does not allow a process running in kernel mode to be preempted; a kernel-mode process will run until it exits kernel mode, blocks, or voluntarily yields control of the CPU. \\
\bottomrule
\end{tabular}
\vspace{\baselineskip}
