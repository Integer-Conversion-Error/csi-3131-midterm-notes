\section{Hardware support for synchronization}
\begin{itemize}
  \item Software-based solutions (like Peterson's) are not guaranteed on modern
        architectures.
  \item This section introduces three hardware instructions for critical-section
        problem support.
  \item These primitives can be used directly or as foundations for more abstract
        synchronization.

        \subsection{Memory barriers}
  \item \textbf{Problem}: Systems may reorder instructions, leading to unreliable data states.
  \item \textbf{Memory model}: How a computer architecture guarantees memory visibility to applications.
  \item \textbf{Categories of Memory Models}:
        \begin{enumerate}
          \item \textbf{Strongly ordered}: Memory modification on one processor is immediately visible to all others.
          \item \textbf{Weakly ordered}: Memory modifications on one processor may not be immediately visible to others.
        \end{enumerate}
  \item \textbf{Issue}: Kernel developers cannot assume memory modification visibility on shared-memory multiprocessors due to varying memory models.
  \item \textbf{Solution}: \textbf{Memory barriers} (or \textbf{memory fences}) are instructions that force memory changes to propagate to all other processors.
  \item \textbf{Functionality}: When a memory barrier is performed, all preceding loads and stores are completed before any subsequent load or store operations.
  \item \textbf{Benefit}: Even if instructions are reordered, memory barriers ensure store operations are completed and visible before future operations.
  \item \textbf{Example (Thread 1 with memory barrier):}
        \begin{verbatim}
while (!flag)
  memory\_barrier();
print x;
\end{verbatim}
        \begin{itemize}
          \item Guarantees `flag` is loaded before `x`.
        \end{itemize}
  \item \textbf{Example (Thread 2 with memory barrier):}
        \begin{verbatim}
x = 100;
memory\_barrier();
flag = true;
\end{verbatim}
        \begin{itemize}
          \item Ensures assignment to `x` occurs before assignment to `flag`.
        \end{itemize}
  \item \textbf{Application to Peterson's Solution}: A memory barrier could be placed between the first two assignment statements in the entry section to prevent reordering.
  \item \textbf{Note}: Memory barriers are low-level operations, typically used by kernel developers for specialized mutual exclusion code.

        \subsection{Hardware instructions}
  \item Modern systems provide special hardware instructions for \textbf{atomically}
        (as one uninterruptible unit) testing/modifying a word or swapping two words.
  \item These instructions simplify critical-section problem solving.
  \item \textbf{Abstracted Instructions}: `test\_and\_set()` and `compare\_and\_swap()` (CAS).

  \item \textbf{`test\_and\_set()` instruction}:
        \begin{itemize}
          \item \textbf{Definition}:
                \begin{verbatim}
boolean test\_and\_set(boolean *target) {
  boolean rv = *target;
  *target = true;
 
  return rv;
}
\end{verbatim}
          \item \textbf{Characteristic}: Executed atomically.
          \item \textbf{Behavior}: If two `test\_and\_set()` instructions run simultaneously (on different cores), they execute sequentially in an arbitrary order.
          \item \textbf{Mutual Exclusion Implementation}:
                \begin{itemize}
                  \item Declare `boolean lock`, initialized to `false`.
                  \item \textbf{Process $P\_i$ structure}:
                        \begin{verbatim}
do {
  while (test\_and\_set(&lock))
     ; /* do nothing */
 
     /* critical section */
 
  lock = false;
 
     /* remainder section */
} while (true);
\end{verbatim}
                \end{itemize}
        \end{itemize}

  \item \textbf{`compare\_and\_swap()` (CAS) instruction}:
        \begin{itemize}
          \item Operates on three operands atomically.
          \item \textbf{Definition}:
                \begin{verbatim}
int compare\_and\_swap(int *value, int expected, int new\_value) {
  int temp = *value;
 
  if (*value == expected)
    *value = new\_value;
 
  return temp;
}
\end{verbatim}
          \item \textbf{Functionality}: `value` is set to `new\_value` ONLY if `(*value == expected)` is true.
          \item \textbf{Return Value}: Always returns the original value of `value`.
          \item \textbf{Characteristic}: Executed atomically.
          \item \textbf{Behavior}: If two CAS instructions run simultaneously, they execute sequentially.
          \item \textbf{Mutual Exclusion Implementation (Basic)}:
                \begin{itemize}
                  \item Global variable `lock`, initialized to 0.
                  \item First process calling `compare\_and\_swap(\&lock, 0, 1)` sets `lock` to 1 and
                        enters critical section (original `lock` was 0).
                  \item Subsequent calls fail (current `lock` is not 0).
                  \item Process exits critical section: sets `lock` back to 0, allowing another
                        process.
                  \item \textbf{Process $P\_i$ structure}:
                        \begin{verbatim}
while (true) {
  while (compare\_and\_swap(&lock, 0, 1) != 0)
     ; /* do nothing */
 
     /* critical section */
 
     lock = 0;
 
     /* remainder section */
}
\end{verbatim}
                  \item \textbf{Limitation}: This basic algorithm satisfies mutual exclusion but NOT bounded waiting.
                \end{itemize}
          \item \textbf{Mutual Exclusion with Bounded Waiting using CAS}:
                \begin{itemize}
                  \item \textbf{Common data structures}:
                        \begin{verbatim}
boolean waiting[n];
int lock;
\end{verbatim}
                  \item `waiting` array elements initialized to `false`, `lock` initialized to 0.
                  \item \textbf{Mutual Exclusion Proof}:
                        
                          \item $P_i$ enters the critical section only if \texttt{waiting[i] == false || key == 0}.
                          \item `key` becomes 0 only if `compare\_and\_swap()` is executed.
                          \item First process executing CAS finds `key == 0`; others wait.
                          \item `waiting[i]` becomes `false` only when another process leaves critical section; only one `waiting[i]` is set to `false`, maintaining mutual exclusion.
                        
                  \item \textbf{Process $P\_i$ structure (with bounded waiting):}
                        \begin{verbatim}
while (true) {
  waiting[i] = true;
  key = 1;
  while (waiting[i] && key == 1)
    key = compare\_and\_swap(&lock,0,1);
  waiting[i] = false;
 
    /* critical section */
 
  j = (i + 1) % n;
  while ((j != i) && !waiting[j])
    j = (j + 1) % n;
 
  if (j == i)
    lock = 0;
  else
     waiting[j] = false;
 
     /* remainder section */
}
\end{verbatim}
                  \item \textbf{Progress Proof}: A process exiting critical section sets `lock` to 0 or `waiting[j]` to `false`, allowing a waiting process to proceed.
                  \item \textbf{Bounded-Waiting Proof}:

                  \item When a process leaves its critical section, it cyclically scans `waiting` array
                        ($i+1, \dots, n-1, 0, \dots, i-1$).
                  \item It designates the first process in the entry section (`waiting[j] == true`) as
                        the next to enter.
                  \item Any waiting process will enter its critical section within $n-1$ turns.

                \end{itemize}
          \item \textbf{Intel x86 Architecture}: `cmpxchg` assembly instruction implements `compare\_and\_swap()`.
                \begin{itemize}
                  \item `lock` prefix used to enforce atomic execution by locking the bus during destination operand update.
                  \item \textbf{General form}: `lock cmpxchg <destination operand>, <source operand>`
                \end{itemize}
        \end{itemize}

        \subsection{Atomic variables}
  \item `compare\_and\_swap()` is often a building block for other synchronization tools, not used directly for mutual exclusion.
  \item \textbf{Atomic variable}: A programming construct providing atomic operations on basic data types (integers, booleans).
  \item \textbf{Purpose}: Ensures mutual exclusion for single variable updates (e.g., counter increments) where data races might occur.
  \item \textbf{Implementation}: Often use `compare\_and\_swap()` operations.
  \item \textbf{Example: Incrementing atomic integer `sequence`}:
        \begin{verbatim}
increment(&sequence);
\end{verbatim}
  \item \textbf{`increment()` function using CAS}:
        \begin{verbatim}
void increment(atomic\_int *v)
{
  int temp;
 
  do {
    temp = *v;
  }
  while (temp != compare\_and\_swap(v, temp, temp+1));
}
\end{verbatim}
  \item \textbf{Limitation}: Atomic variables provide atomic updates but don't solve all race conditions.
  \item \textbf{Bounded-Buffer Problem Example (with atomic `count`):}
        \begin{itemize}
          \item If buffer is empty and two consumers wait for `count > 0`.
          \item Producer adds one item, `count` becomes 1 (atomically).
          \item Both consumers could exit their `while` loops and proceed to consume, even
                though `count` is only 1.
        \end{itemize}
  \item \textbf{Usage}: Commonly used in OS and concurrent applications, but often limited to single updates of shared data (counters, sequence generators).
  \item \textbf{Next}: Explore more robust tools for generalized race conditions.
\end{itemize}

\subsection{Section glossary}
\rowcolors{2}{gray!10}{white}
\centering
\begin{tabular}{>{\raggedright}p{0.35\textwidth} >{\raggedright\arraybackslash}p{0.55\textwidth}}
  \toprule
  \textbf{Term}            & \textbf{Definition}                                                                                                 \\
  \midrule
  \textbf{memory model}    & Computer architecture memory guarantee, usually either strongly ordered or weakly ordered.                          \\
  \textbf{memory barriers} & Computer instructions that force any changes in memory to be propagated to all other processors in the system.      \\
  \textbf{memory fences}   & Computer instructions that force any changes in memory to be propagated to all other processors in the system.      \\
  \textbf{atomically}      & A computer activity (such as a CPU instruction) that operates as one uninterruptable unit.                          \\
  \textbf{atomic variable} & A programming language construct that provides atomic operations on basic data types such as integers and booleans. \\
  \bottomrule
\end{tabular}
\vspace{\baselineskip}
