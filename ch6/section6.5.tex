\section{Mutex locks}
\begin{itemize}
    \item Hardware-based solutions (Section \ref{sec:6.4}) are complex and generally inaccessible to application programmers.
    \item Operating system designers build higher-level software tools.
    \item The simplest tool is the \textbf{mutex lock} (short for \textbf{mutual exclusion}).
    \item \textbf{Purpose}: Protect critical sections and prevent race conditions.
    \item \textbf{Usage}: A process must `acquire()` the lock before entering a critical section and `release()` it upon exiting.
    \item \textbf{General structure}:
\begin{verbatim}
while (true) {
 
   acquire lock
 
       critical section
 
   release lock
 
       remainder section
 
}
\end{verbatim}
    \item \textbf{Mutex Lock Properties}:
    \begin{itemize}
        \item Has a boolean variable `available`.
        \item `available = true`: Lock is available.
        \item `acquire()` succeeds if `available` is true, then sets `available = false`.
        \item Process attempting to acquire an unavailable lock is blocked until released.
    \end{itemize}
    \item \textbf{`acquire()` definition}:
\begin{verbatim}
acquire() {
   while (!available)
     ; /* busy wait */
   available = false;
}
\end{verbatim}
    \item \textbf{`release()` definition}:
\begin{verbatim}
release() {
  available = true;
}
\end{verbatim}
    \item \textbf{Atomicity}: Calls to `acquire()` and `release()` must be atomic.
    \begin{itemize}
        \item Can be implemented using the CAS operation (Section \ref{sec:6.4}).
    \end{itemize}
    \item \textbf{Lock Contention}:
    \begin{itemize}
        \item \textbf{Contended lock}: A thread blocks while trying to acquire the lock.
        \item \textbf{Uncontended lock}: Lock is available when a thread attempts to acquire it.
        \item \textbf{High contention}: Many threads attempt to acquire the lock.
        \item \textbf{Low contention}: Few threads attempt to acquire the lock.
        \item Highly contended locks decrease overall performance of concurrent applications.
    \end{itemize}
    \item \textbf{Spinlocks}:
    \begin{itemize}
        \item The type of mutex lock described is also called a \textbf{spinlock}.
        \item Process "spins" (loops continuously) while waiting for the lock.
        \item \textbf{Disadvantage}: Requires \textbf{busy waiting}.
        \begin{itemize}
            \item Wastes CPU cycles, especially problematic in single-CPU core multiprogramming systems.
            \item (Section \ref{sec:6.6} discusses strategies to avoid busy waiting by putting processes to sleep.)
        \end{itemize}
        \item \textbf{Advantage}: No context switch required when waiting on a lock.
        \begin{itemize}
            \item Context switches can be time-consuming.
            \item Preferable on multicore systems for short-duration locks.
            \item One thread can "spin" on one core while another performs its critical section on another core.
            \item Widely used in modern operating systems on multicore systems.
        \end{itemize}
        \item \textbf{Rule of thumb}: Use a spinlock if the lock will be held for less than two context switches (as waiting involves two context switches).
    \end{itemize}
    \item \textbf{Further Discussion}: Chapter Synchronization Examples will cover mutex locks in classical synchronization problems and their use in various operating systems and Pthreads.
\end{itemize}

\subsection*{Section glossary}
\rowcolors{2}{gray!10}{white}
\centering
\begin{tabular}{>{\raggedright\arraybackslash}p{0.35\textwidth} >{\raggedright\arraybackslash}p{0.55\textwidth}}
\toprule
\textbf{Term} & \textbf{Definition} \\
\midrule
\textbf{mutex lock} & A mutual exclusion lock; the simplest software tool for assuring mutual exclusion. \\
\textbf{contended} & A term describing the condition of a lock when a thread blocks while trying to acquire it. \\
\textbf{uncontended} & A term describing a lock that is available when a thread attempts to acquire it. \\
\textbf{busy waiting} & A practice that allows a thread or process to use CPU time continuously while waiting for something. An I/O loop in which an I/O thread continuously reads status information while waiting for I/O to complete. \\
\textbf{spinlock} & A locking mechanism that continuously uses the CPU while waiting for access to the lock. \\
\bottomrule
\end{tabular}
\vspace{\baselineskip}
