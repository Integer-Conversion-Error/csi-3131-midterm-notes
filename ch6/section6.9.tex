\section{Evaluation}
\begin{itemize}
    \item This section evaluates different synchronization tools for solving the critical-section problem.
    \item Given correct implementation and usage, these tools ensure mutual exclusion and address liveness issues.
    \item With the rise of concurrent programs on multicore systems, performance of synchronization tools is crucial.
    \item Choosing the right tool can be challenging. This section provides strategies for tool selection.

    \item \textbf{Hardware Solutions (Section \ref{sec:6.4})}:
    \begin{itemize}
        \item Considered very low-level.
        \item Typically used as foundations for constructing other synchronization tools (e.g., mutex locks).
        \item Recent focus on using CAS instruction for \textbf{lock-free} algorithms.
        \begin{itemize}
            \item \textbf{Lock-free algorithms}: Provide race condition protection without locking overhead.
            \item \textbf{Advantages}: Low overhead, ability to scale.
            \item \textbf{Disadvantage}: Often difficult to develop and test.
        \end{itemize}
    \end{itemize}

    \item \textbf{Priority Inversion}:
    \begin{itemize}
        \item More than a scheduling inconvenience, especially in real-time systems.
        \item Can cause a process to take longer than expected, leading to cascading failures and system failure.
        \item \textbf{Mars Pathfinder Example (1997)}:
        \begin{itemize}
            \item Sojourner rover experienced frequent computer resets.
            \item \textbf{Cause}: High-priority task ("bc\_dist") was delayed waiting for a shared resource held by a lower-priority task ("ASI/MET").
            \item The lower-priority task was preempted by multiple medium-priority tasks.
            \item "bc\_dist" stalled, "bc\_sched" detected the problem and initiated a reset.
            \item This was a typical case of priority inversion.
            \item \textbf{Solution}: VxWorks real-time OS (on Sojourner) had a global variable to enable priority inheritance on all semaphores. Setting this variable solved the problem.
        \end{itemize}
    \end{itemize}

    \item \textbf{CAS-based Synchronization vs. Traditional Locking}:
    \begin{itemize}
        \item \textbf{CAS-based (Optimistic approach)}:
        \begin{itemize}
            \item Optimistically update a variable first.
            \item Use collision detection to see if another thread updated concurrently.
            \item If conflict, retry operation until successful without conflict.
        \end{itemize}
        \item \textbf{Mutual-exclusion locking (Pessimistic strategy)}:
        \begin{itemize}
            \item Assume another thread is concurrently updating the variable.
            \item Pessimistically acquire the lock before making any updates.
        \end{itemize}
    \end{itemize}

    \item \textbf{Performance Guidelines (CAS vs. Traditional Synchronization)}:
    \begin{itemize}
        \item \textbf{Uncontended loads}: Both are generally fast; CAS protection is somewhat faster.
        \item \textbf{Moderate contention}: CAS protection is faster (possibly much faster).
        \begin{itemize}
            \item CAS operation succeeds most of the time.
            \item If it fails, it iterates only a few times before succeeding.
            \item Traditional locking: Any contended lock acquisition involves a more complex, time-intensive code path (suspends thread, places on wait queue, context switch).
        \end{itemize}
        \item \textbf{High contention}: Traditional synchronization is ultimately faster than CAS-based synchronization.
    \end{itemize}

    \item \textbf{Choosing Synchronization Mechanisms and Performance Impact}:
    \begin{itemize}
        \item Atomic integers are much lighter-weight than traditional locks.
        \item More appropriate than mutex locks or semaphores for single updates to shared variables (e.g., counters).
        \item Spinlocks are used on multiprocessor systems when locks are held for short durations.
        \item Mutex locks are simpler and have less overhead than semaphores.
        \item Mutex locks are preferable to binary semaphores for protecting critical section access.
        \item Counting semaphores are more appropriate than mutex locks for controlling access to a finite number of resources.
        \item Reader-writer locks may be preferred over mutex locks for higher concurrency (multiple readers allowed).
    \end{itemize}

    \item \textbf{Higher-Level Tools (Monitors, Condition Variables)}:
    \begin{itemize}
        \item \textbf{Appeal}: Simplicity and ease of use.
        \item \textbf{Drawbacks}: Significant overhead; may scale less effectively in highly contended situations depending on implementation.
    \end{itemize}

    \item \textbf{Ongoing Research}: Much research is focused on developing scalable, efficient tools for concurrent programming.
    \begin{itemize}
        \item Designing compilers for more efficient code.
        \item Developing languages with built-in concurrent programming support.
        \item Improving performance of existing libraries and APIs.
    \end{itemize}
    \item \textbf{Next Chapter}: Examines how various operating systems and APIs implement the synchronization tools discussed in this chapter.
\end{itemize}

\subsection*{Section glossary}
\rowcolors{2}{gray!10}{white}
\centering
\begin{tabular}{>{\raggedright}p{0.35\textwidth} >{\raggedright\arraybackslash}p{0.55\textwidth}}
\toprule
\textbf{Term} & \textbf{Definition} \\
\midrule
\textbf{lock-free} & An algorithmic strategy that provides protection from race conditions without requiring the overhead of locking. \\
\bottomrule
\end{tabular}
\vspace{\baselineskip}
