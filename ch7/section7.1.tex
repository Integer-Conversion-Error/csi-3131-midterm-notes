\section{Classic problems of synchronization}

\subsection{Introduction}
\begin{itemize}
    \item Previous chapter: Synchronization Tools.
    \item Focused on critical-section problem and race conditions with shared data.
    \item Examined tools to prevent race conditions:
    \begin{itemize}
        \item Low-level hardware: memory barriers, compare-and-swap.
        \item Higher-level: mutex locks, semaphores, monitors.
    \end{itemize}
    \item Discussed challenges: liveness hazards like deadlocks.
    \item This chapter:
    \begin{itemize}
        \item Applies synchronization tools to classic problems.
        \item Explores synchronization mechanisms in Linux, UNIX, Windows.
        \item Describes API details for Java and POSIX systems.
    \end{itemize}
\end{itemize}

\subsection{Chapter objectives}
\begin{itemize}
    \item Explain: bounded-buffer, readers-writers, dining-philosophers synchronization problems.
    \item Describe: specific tools used by Linux and Windows for process synchronization.
    \item Illustrate: how POSIX and Java solve process synchronization problems.
    \item Design and develop: solutions using POSIX and Java APIs.
\end{itemize}

\subsection{Classic problems of synchronization}
\begin{itemize}
    \item Examples of concurrency-control problems.
    \item Used for testing new synchronization schemes.
    \item Solutions traditionally use semaphores; mutex locks can be used for binary semaphores in actual implementations.
\end{itemize}

\subsection{The bounded-buffer problem}
\begin{itemize}
    \item Introduced in a previous chapter.
    \item Illustrates power of synchronization primitives.
    \item General structure presented; related programming project in exercises.
    \item Producer and consumer processes share data structures:
    \begin{verbatim}
int N;
semaphore mutex = 1;
semaphore empty = N;
semaphore full = 0
\end{verbatim}
    \item Pool: \texttt{N} buffers, each holding one item.
    \item \texttt{mutex}: binary semaphore, mutual exclusion for buffer pool access, initialized to 1.
    \item \texttt{empty}: counts empty buffers, initialized to \texttt{N}.
    \item \texttt{full}: counts full buffers, initialized to 0.
    \item The producer process structure is as follows:
    \begin{verbatim}
while (true) {
     . . .
  /* produce an item in next_produced */
     . . .
  wait(empty);
  wait(mutex);
     . . .
  /* add next_produced to the buffer */
     . . .
  signal(mutex);
  signal(full);
}
    \end{verbatim}
    \item The consumer process structure is as follows:
    \begin{verbatim}
while (true) {
  wait(full);
  wait(mutex);
    . . .
  /* remove an item from buffer to next_consumed */
    . . .
  signal(mutex);
  signal(empty);
    . . .
  /* consume the item in next_consumed */
    . . .
}
    \end{verbatim}
    \item Symmetry between producer and consumer.
    \item Interpretation: producer produces full buffers for consumer, or consumer produces empty buffers for producer.
\end{itemize}

\subsection{The readers-writers problem}
\begin{itemize}
    \item Shared database accessed by concurrent processes.
    \item \textbf{\textit{Readers}}: only read database.
    \item \textbf{\textit{Writers}}: update (read and write) database.
    \item Problem: If writer and another process (reader/writer) access simultaneously, chaos may ensue.
    \item Requirement: Writers must have exclusive access while writing.
    \item \textit{Readers-writers problem}: Synchronization problem to ensure this.
    \item Variations involve priorities:
    \begin{itemize}
        \item \textbf{\textit{First}} readers-writers problem: No reader waits unless a writer already has permission. Readers don't wait for other readers if a writer is waiting.
        \item \textbf{\textit{Second}} readers-writers problem: Writer performs write ASAP once ready. If writer is waiting, no new readers may start.
    \end{itemize}
    \item Starvation: Solutions may lead to starvation (writers in first case, readers in second).
    \item Solution to first readers-writers problem:
    \begin{itemize}
        \item Shared data structures for reader processes:
        \begin{verbatim}
semaphore rw_mutex = 1;
semaphore mutex = 1;
int read_count = 0;
\end{verbatim}
        \item \texttt{mutex} and \texttt{rw\_mutex}: binary semaphores, initialized to 1.
        \item \texttt{read\_count}: integer, number of active readers, initialized to 0.
        \item \texttt{rw\_mutex}: common to reader and writer processes, acts as mutual exclusion for writers, used by first/last reader entering/exiting critical section.
        \item \texttt{mutex}: ensures mutual exclusion when \texttt{read\_count} is updated.
        \item \texttt{read\_count}: tracks current readers.
        \item The writer process structure is as follows:
        \begin{verbatim}
while (true) {
  wait(rw_mutex);
    . . .
  /* writing is performed */
    . . .
  signal(rw_mutex);
}
        \end{verbatim}
        \item The reader process structure is as follows:
        \begin{verbatim}
while (true) {
  wait(mutex);
  read_count++;
  if (read_count == 1)
    wait(rw_mutex);
  signal(mutex);
    . . .
  /* reading is performed */
    . . .
  wait(mutex);
  read_count--;
  if (read_count == 0)
    signal(rw_mutex);
  signal(mutex);
}
        \end{verbatim}
        \item If writer in critical section and \textit{n} readers waiting: 1 reader queued on \texttt{rw\_mutex}, \textit{n} $-$ 1 readers queued on \texttt{mutex}.
        \item When writer executes \texttt{signal(rw\_mutex)}, scheduler selects waiting readers or a single waiting writer.
    \end{itemize}
    \item \textit{Reader-writer locks}: generalization of problem/solutions.
    \begin{itemize}
        \item Acquire lock by specifying mode: \textit{read} or \textit{write}.
        \item Read mode: multiple processes concurrently.
        \item Write mode: only one process (exclusive access).
        \item Most useful when:
        \begin{itemize}
            \item Easy to identify read-only vs. read-write processes.
            \item More readers than writers (increased concurrency compensates for overhead).
        \end{itemize}
    \end{itemize}
\end{itemize}

\subsection{The dining-philosophers problem}
\begin{itemize}
    \item Five philosophers, circular table, five chairs, bowl of rice, five single chopsticks.
    \item Philosopher thinks, then gets hungry.
    \item Tries to pick up two closest chopsticks (left and right neighbors).
    \item Picks up one chopstick at a time. Cannot pick up if neighbor holds it.
    \item Eats with both chopsticks, then puts them down and thinks.
    \item Classic synchronization problem: example of allocating several resources among several processes.
    \item Goal: deadlock-free and starvation-free allocation.

    \subsubsection{Semaphore solution}
    \begin{itemize}
        \item Each chopstick represented by a semaphore.
        \item Grab chopstick: \texttt{wait()} operation on semaphore.
        \item Release chopstick: \texttt{signal()} operation on semaphore.
        \item Shared data:
        \begin{verbatim}
semaphore chopstick[5];
\end{verbatim}
        \item All \texttt{chopstick} elements initialized to 1.
        \item The structure of philosopher $i$ is as follows:
        \begin{verbatim}
while (true) {
  wait(chopstick[i]);
  wait(chopstick[(i+1) % 5]);
    . . .
  /* eat for a while */
    . . .
  signal(chopstick[i]);
  signal(chopstick[(i+1) % 5]);
    . . .
  /* think for awhile */
    . . .
}
        \end{verbatim}
        \item Guarantees no two neighbors eat simultaneously.
        \item \textbf{Problem}: Could create deadlock.
        \begin{itemize}
            \item Example: All five philosophers hungry, each grabs left chopstick. All \texttt{chopstick} elements become 0.
            \item Each tries to grab right chopstick, delayed forever.
        \end{itemize}
        \item Remedies to deadlock:
        \begin{itemize}
            \item Allow at most four philosophers at table simultaneously.
            \item Philosopher picks up both chopsticks only if both available (in critical section).
            \item Asymmetric solution: odd-numbered philosopher picks left then right; even-numbered picks right then left.
        \end{itemize}
        \item A previous chapter presents a deadlock-free solution.
        \item Satisfactory solution must guard against starvation (deadlock-free $\neq$ starvation-free).
    \end{itemize}

    \subsubsection{Monitor solution}
    \begin{itemize}
        \item Deadlock-free solution using monitors.
        \item Restriction: Philosopher picks up chopsticks only if both available.
        \item Three states for a philosopher:
        \begin{verbatim}
enum {THINKING, HUNGRY, EATING} state[5];
\end{verbatim}
        \item \texttt{state[i] = EATING} only if neighbors not eating: (\texttt{state[(i+4) \% 5] != EATING}) and (\texttt{state[(i+1) \% 5] != EATING}).
        \item Condition variable:
        \begin{verbatim}
condition self[5];
\end{verbatim}
        \item Allows philosopher $i$ to delay if hungry but cannot get chopsticks.
        \item The distribution of chopsticks is controlled by the \texttt{DiningPhilosophers} monitor, which is defined below.
        \item Each philosopher $i$ must invoke the operations \texttt{pickup()} and \texttt{putdown()} in the following sequence:
        \begin{verbatim}
DiningPhilosophers.pickup(i);
                ...
                eat
                ...
DiningPhilosophers.putdown(i);
\end{verbatim}
        \item The monitor is defined as follows:
        \begin{verbatim}
monitor DiningPhilosophers
{
  enum {THINKING, HUNGRY, EATING} state[5];
  condition self[5];
 
  void pickup(int i) {
    state[i] = HUNGRY;
    test(i);
    if (state[i] != EATING)
      self[i].wait();
  }
 
  void putdown(int i) {
    state[i] = THINKING;
    test((i + 4) % 5);
    test((i + 1) % 5);
  }
 
  void test(int i) {
    if ((state[(i + 4) % 5] != EATING) &&
     (state[i] == HUNGRY) &&
     (state[(i + 1) % 5] != EATING)) {
       state[i] = EATING;
       self[i].signal();
    }
  }
 
  initialization_code() {
    for (int i = 0; i < 5; i++)
      state[i] = THINKING;
   }
}
        \end{verbatim}
        \item Ensures no two neighbors eat simultaneously and no deadlocks.
        \item Possible for a philosopher to starve (solution not presented, left as exercise).
    \end{itemize}
\end{itemize}

\subsection{Section glossary}
\rowcolors{2}{gray!10}{white}
\centering
\begin{tabular}{>{\raggedright}p{0.35\textwidth} >{\raggedright\arraybackslash}p{0.55\textwidth}}
\toprule
\textbf{Term} & \textbf{Definition} \\
\midrule
\textbf{readers-writers problem} & Synchronization problem where processes/threads either read or read/write shared data. \\
\textbf{reader-writer lock} & Lock for item access by read-only and read-write accessors. \\
\textbf{dining-philosophers problem} & Classic synchronization problem where multiple operators (philosophers) access multiple items (chopsticks) simultaneously. \\
\bottomrule
\end{tabular}
\vspace{\baselineskip}
