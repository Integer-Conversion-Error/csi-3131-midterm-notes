\section{Synchronization within the kernel}

\subsection{Synchronization in Windows}
\begin{itemize}
    \item Windows OS: multithreaded kernel, supports real-time applications and multiple processors.
    \item Single-processor systems:
    \begin{itemize}
        \item Kernel accesses global resource: temporarily masks interrupts for all interrupt handlers that may access the resource.
    \end{itemize}
    \item Multiprocessor systems:
    \begin{itemize}
        \item Kernel protects global resource access using spinlocks.
        \item Spinlocks used only for short code segments.
        \item Kernel ensures thread never preempted while holding a spinlock (for efficiency).
    \end{itemize}
    \item Thread synchronization outside kernel: \textit{dispatcher objects}.
    \begin{itemize}
        \item Threads synchronize using: mutex locks, semaphores, events, timers.
        \item Mutex locks: protect shared data; thread gains ownership to access, releases when finished.
        \item Semaphores: behave as described in Section \hyperref[sec:6.6]{6.6}.
        \item \textit{Events}: similar to condition variables; notify waiting thread when condition occurs.
        \item Timers: notify one (or more) threads when specified time expires.
    \end{itemize}
    \item Dispatcher objects states:
    \begin{itemize}
        \item \textit{Signaled state}: object available, thread will not block.
        \item \textit{Nonsignaled state}: object not available, thread will block.
        \item State transitions of mutex lock dispatcher object illustrated in Figure \ref{fig:7.2.1}.
    \end{itemize}
    \item Relationship between dispatcher object state and thread state:
    \begin{itemize}
        \item Thread blocks on nonsignaled object: state changes from ready to waiting, placed in waiting queue.
        \item Object moves to signaled state: kernel checks waiting threads.
        \item Kernel moves one (or more) threads from waiting to ready state.
        \item Number of threads selected depends on dispatcher object type:
        \begin{itemize}
            \item Mutex: only one thread (mutex owned by single thread).
            \item Event: all waiting threads.
        \end{itemize}
    \end{itemize}
    \item Mutex lock illustration:
    \begin{itemize}
        \item Thread tries to acquire nonsignaled mutex: suspended, placed in waiting queue.
        \item Mutex moves to signaled state (released by another thread): thread at front of queue moves from waiting to ready, acquires mutex.
    \end{itemize}
    \item \textit{Critical-section object}:
    \begin{itemize}
        \item User-mode mutex, often acquired/released without kernel intervention.
        \item Multiprocessor system: first uses spinlock while waiting.
        \item If spins too long: acquiring thread allocates kernel mutex and yields CPU.
        \item Efficient: kernel mutex allocated only when contention exists (rare in practice, significant savings).
    \end{itemize}
    \item Programming project at end of chapter uses mutex locks and semaphores in Windows API.
\end{itemize}

\subsection{Synchronization in Linux}
\begin{itemize}
    \item Prior to Version 2.6: nonpreemptive kernel (process in kernel mode could not be preempted).
    \item Now: Linux kernel is fully preemptive (task can be preempted while running in kernel).
    \item Mechanisms for synchronization in kernel:
    \begin{itemize}
        \item \textbf{Atomic integers}:
        \begin{itemize}
            \item Simplest synchronization technique.
            \item Opaque data type: \texttt{atomic\_t}.
            \item All math operations are atomic (performed without interruption).
            \item Example:
            \begin{verbatim}
atomic_t counter;
int value;
            \end{verbatim}
            \item Atomic operations:
            \begin{itemize}
                \item \texttt{atomic\_set(\&counter,5);}: \texttt{counter = 5}
                \item \texttt{atomic\_add(10,\&counter);}: \texttt{counter = counter + 10}
                \item \texttt{atomic\_sub(4,\&counter);}: \texttt{counter = counter - 4}
                \item \texttt{atomic\_inc(\&counter);}: \texttt{counter = counter + 1}
                \item \texttt{value = atomic\_read(\&counter);}: \texttt{value = 12} (example result)
            \end{itemize}
            \item Efficient for updating integer variables (e.g., counters); no locking overhead.
            \item Limited use: only for single integer variables. More sophisticated tools needed for multiple variables in race conditions.
        \end{itemize}
        \item \textbf{Mutex locks}:
        \begin{itemize}
            \item Protect critical sections within kernel.
            \item Task invokes \texttt{mutex\_lock()} before critical section, \texttt{mutex\_unlock()} after.
            \item If unavailable: task calling \texttt{mutex\_lock()} sleeps, awakened when owner invokes \texttt{mutex\_unlock()}.
        \end{itemize}
        \item \textbf{Spinlocks and Semaphores}:
        \begin{itemize}
            \item Linux also provides these (and reader-writer versions).
            \item SMP machines: spinlock is fundamental locking mechanism, held for short durations.
            \item Single-processor machines (e.g., embedded systems): spinlocks inappropriate. Replaced by enabling/disabling kernel preemption.
            \item Summary for single-processor:
            \begin{itemize}
                \item Instead of holding spinlock: kernel disables kernel preemption.
                \item Instead of releasing spinlock: kernel enables kernel preemption.
            \end{itemize}
        \end{itemize}
    \end{itemize}
    \item \textbf{\textit{Nonrecursive}} locks:
    \begin{itemize}
        \item Both spinlocks and mutex locks in Linux kernel are nonrecursive.
        \item If thread acquires lock, cannot acquire same lock again without releasing it first.
        \item Second attempt to acquire will block.
    \end{itemize}
    \item Disabling/Enabling kernel preemption:
    \begin{itemize}
        \item Linux approach: \texttt{preempt\_disable()} and \texttt{preempt\_enable()} system calls.
        \item Kernel not preemptible if task running in kernel holds a lock.
        \item Enforcement: each task has \texttt{thread-info} structure with \texttt{preempt\_count} counter.
        \item \texttt{preempt\_count}: indicates number of locks held by task.
        \item Lock acquired: \texttt{preempt\_count} incremented.
        \item Lock released: \texttt{preempt\_count} decremented.
        \item If \texttt{preempt\_count} $>$ 0: not safe to preempt kernel (task holds lock).
        \item If \texttt{preempt\_count} = 0: kernel can be safely interrupted (assuming no outstanding \texttt{preempt\_disable()} calls).
    \end{itemize}
    \item When to use which lock:
    \begin{itemize}
        \item Spinlocks (and kernel preemption disable/enable): only when lock held for short duration.
        \item Semaphores or mutex locks: when lock must be held for longer period.
    \end{itemize}
\end{itemize}

\subsection{Section glossary}
\begin{itemize}
    \item \textit{dispatcher objects}: Windows scheduler feature controlling dispatching and synchronization. Threads synchronize via mutex locks, semaphores, events, and timers.
    \item \textit{event}: Windows OS scheduling feature, similar to a condition variable.
    \item \textit{critical-section object}: User-mode mutex object in Windows OS, often acquired/released without kernel intervention.
\end{itemize}
