\section{POSIX synchronization}

\begin{itemize}
    \item Synchronization methods in previous section: kernel-level, for kernel developers.
    \item POSIX API: available for user-level programmers, not part of specific OS kernel.
    \item Implemented using host OS tools.
    \item This section covers: mutex locks, semaphores, condition variables in Pthreads and POSIX APIs.
    \item Widely used for thread creation and synchronization on UNIX, Linux, macOS.
\end{itemize}

\subsection{POSIX mutex locks}
\begin{itemize}
    \item Fundamental synchronization technique with Pthreads.
    \item Purpose: protect critical sections of code.
    \item Thread acquires lock before entering, releases upon exiting.
    \item Data type: \texttt{pthread\_mutex\_t}.
    \item Creation: \texttt{pthread\_mutex\_init()} function.
    \begin{itemize}
        \item First parameter: pointer to mutex.
        \item Second parameter: \texttt{NULL} for default attributes.
    \end{itemize}
    \item Example:
    \begin{verbatim}
#include <pthread.h>
 
pthread_mutex_t mutex;
 
/* create and initialize  the mutex lock */
pthread_mutex_init(&mutex,NULL);
    \end{verbatim}
    \item Acquisition and Release: \texttt{pthread\_mutex\_lock()} and \texttt{pthread\_mutex\_unlock()}.
    \begin{itemize}
        \item If \texttt{pthread\_mutex\_lock()} invoked and mutex unavailable: calling thread blocks until owner invokes \texttt{pthread\_mutex\_unlock()}.
    \end{itemize}
    \item Protecting critical section example:
    \begin{verbatim}
/* acquire the mutex lock */
pthread_mutex_lock(&mutex);
 
/* critical section */
 
/* release the mutex lock */
pthread_mutex_unlock(&mutex);
    \end{verbatim}
    \item Return values: 0 for correct operation, nonzero for error.
\end{itemize}

\subsection{POSIX semaphores}
\begin{itemize}
    \item Many systems implementing Pthreads provide semaphores.
    \item Not part of POSIX standard; belong to POSIX SEM extension.
    \item Two types: \textit{named} and \textit{unnamed}.
    \item Differences: how they are created and shared between processes.
    \item Linux systems (Version 2.6+ kernel) support both.

    \subsubsection{POSIX named semaphores}
    \begin{itemize}
        \item Creation and opening: \texttt{sem\_open()} function.
        \item Example:
        \begin{verbatim}
#include <semaphore.h>
sem_t *sem;
 
/* Create the semaphore and initialize it to 1 */
sem = sem_open("SEM", O_CREAT, 0666, 1);
        \end{verbatim}
        \item \texttt{"SEM"}: semaphore name.
        \item \texttt{O\_CREAT} flag: semaphore created if it doesn't exist.
        \item \texttt{0666}: read and write access for other processes.
        \item Initialized to 1.
        \item Advantage: multiple unrelated processes can easily use common semaphore by name.
        \item Subsequent \texttt{sem\_open()} calls (with same parameters) by other processes return descriptor to existing semaphore.
        \item Operations:
        \begin{itemize}
            \item \texttt{wait()} $\rightarrow$ \texttt{sem\_wait()}
            \item \texttt{signal()} $\rightarrow$ \texttt{sem\_post()}
        \end{itemize}
        \item Protecting critical section example:
        \begin{verbatim}
/* acquire the semaphore */
sem_wait(sem);
 
/* critical section */
 
/* release the semaphore */
sem_post(sem);
        \end{verbatim}
        \item Supported by Linux and macOS.
    \end{itemize}

    \subsubsection{POSIX unnamed semaphores}
    \begin{itemize}
        \item Creation and initialization: \texttt{sem\_init()} function.
        \item Parameters:
        \begin{enumerate}
            \item Pointer to the semaphore.
            \item Flag indicating level of sharing.
            \item Semaphore's initial value.
        \end{enumerate}
        \item Example:
        \begin{verbatim}
#include <semaphore.h>
sem_t sem;
 
/* Create the semaphore and initialize it to 1 */
sem_init(&sem, 0, 1);
        \end{verbatim}
        \item Flag 0: semaphore shared only by threads in creating process.
        \item Nonzero flag: allows sharing between separate processes (by placing in shared memory).
        \item Initialized to 1.
        \item Operations: uses same \texttt{sem\_wait()} and \texttt{sem\_post()} as named semaphores.
        \item Protecting critical section example:
        \begin{verbatim}
/* acquire the semaphore */
sem_wait(&sem);
 
/* critical section */
 
/* release the semaphore */
sem_post(&sem);
        \end{verbatim}
        \item Return values: 0 for success, nonzero for error.
    \end{itemize}
\end{itemize}

\subsection{POSIX condition variables}
\begin{itemize}
    \item Pthreads condition variables are similar to those described in a previous chapter.
    \item The difference is that the aforementioned chapter uses monitors for locking, whereas the C language, which is used with Pthreads, does not provide monitors.
    \item Locking accomplished by associating condition variable with a mutex lock.
    \item Data type: \texttt{pthread\_cond\_t}.
    \item Initialization: \texttt{pthread\_cond\_init()} function.
    \item Example of creating and initializing condition variable and associated mutex:
    \begin{verbatim}
pthread_mutex_t mutex;
pthread_cond_t cond_var;
 
pthread_mutex_init(&mutex,NULL);
pthread_cond_init(&cond_var,NULL);
    \end{verbatim}
    \item Waiting on condition variable: \texttt{pthread\_cond\_wait()} function.
    \item Example of waiting for \texttt{a == b}:
    \begin{verbatim}
pthread_mutex_lock(&mutex);
while (a != b)
     pthread_cond_wait(&cond_var, &mutex);
pthread_mutex_unlock(&mutex);
    \end{verbatim}
    \item Mutex lock must be acquired before \texttt{pthread\_cond\_wait()} call.
    \item Mutex protects data in conditional clause from race conditions.
    \item If condition not true: \texttt{pthread\_cond\_wait()} invoked, passing mutex and condition variable.
    \item \texttt{pthread\_cond\_wait()} releases mutex lock, allowing other threads to access/update shared data.
    \item Conditional clause within a loop: important to recheck condition after being signaled (protects against program errors).
    \item Signaling a condition variable: \texttt{pthread\_cond\_signal()} function.
    \item Thread modifying shared data invokes \texttt{pthread\_cond\_signal()} to signal one waiting thread.
    \item Example of signaling:
    \begin{verbatim}
pthread_mutex_lock(&mutex);
a = b;
pthread_cond_signal(&cond_var);
pthread_mutex_unlock(&mutex);
    \end{verbatim}
    \item \texttt{pthread\_cond\_signal()} does NOT release mutex lock.
    \item Subsequent \texttt{pthread\_mutex\_unlock()} releases mutex.
    \item Once mutex released, signaled thread becomes owner of mutex and returns from \texttt{pthread\_cond\_wait()}.
    \item Programming problems/projects at end of chapter use Pthreads mutex locks, condition variables, and POSIX semaphores.
\end{itemize}

\subsection{Section glossary}
\rowcolors{2}{gray!10}{white}
\centering
\begin{tabular}{>{\raggedright}p{0.35\textwidth} >{\raggedright\arraybackslash}p{0.55\textwidth}}
\toprule
\textbf{Term} & \textbf{Definition} \\
\midrule
\textbf{named semaphore} & POSIX scheduling construct, exists in file system, shareable by unrelated processes. \\
\textbf{unnamed semaphore} & POSIX scheduling construct, usable only by threads in the same process. \\
\bottomrule
\end{tabular}
\vspace{\baselineskip}
