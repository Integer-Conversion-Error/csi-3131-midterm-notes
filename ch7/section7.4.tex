\section{Synchronization in Java}

\begin{itemize}
    \item Java language and API: rich support for thread synchronization since its origins.
    \item This section covers:
    \begin{itemize}
        \item Java monitors (original mechanism).
        \item Reentrant locks, semaphores, condition variables (introduced in Release 1.5).
    \end{itemize}
    \item Focus on common locking and synchronization mechanisms.
    \item Java API has more features not covered (e.g., atomic variables, CAS instruction).
\end{itemize}

\subsection{Java monitors}
\begin{itemize}
    \item Java provides a monitor-like concurrency mechanism.
    \item Illustrated with \texttt{BoundedBuffer} class (Figure \ref{fig:7.4.1}).
    \item Producer/consumer invoke \texttt{insert()} and \texttt{remove()} methods.
    \item \texttt{BoundedBuffer} class structure:
    \begin{verbatim}
public class BoundedBuffer<E>
{
  private static final int BUFFER_SIZE = 5;
 
  private int count, in, out;
  private E[] buffer;
 
  public BoundedBuffer() {
    count = 0;
    in = 0;
    out = 0;
    buffer = (E[]) new Object[BUFFER_SIZE];
  }
 
  /* Producers call this method */
  public synchronized void insert(E item) {
    /* See Figure 7.4.3 */
  }
 
  /* Consumers call this method */
  public synchronized E remove() {
    /* See Figure 7.4.3 */
  }
}
    \end{verbatim}
    \item Every Java object has a single associated lock.
    \item \texttt{synchronized} method: entering requires owning the object's lock.
    \item Declared by placing \texttt{synchronized} keyword in method definition (e.g., \texttt{insert()}, \texttt{remove()}).
    \item Entering \texttt{synchronized} method:
    \begin{itemize}
        \item Requires owning lock on \texttt{BoundedBuffer} object instance.
        \item If lock owned by another thread: calling thread blocks, placed in object's \textit{entry set}.
        \item \textit{Entry set}: set of threads waiting for lock to become available.
        \item If lock available: calling thread becomes owner, enters method.
        \item Lock released when thread exits method.
        \item If entry set not empty on lock release: JVM arbitrarily selects thread from set to own lock (often FIFO in practice).
    \end{itemize}
    \item Entry set operation illustrated in Figure \ref{fig:7.4.2}.
    \item Every object also has a \textit{wait set} (initially empty).
    \item When thread enters \texttt{synchronized} method (owns lock):
    \begin{itemize}
        \item May be unable to continue if condition not met (e.g., producer calls \texttt{insert()} and buffer is full).
        \item Thread releases lock and waits until condition is met.
    \end{itemize}
\end{itemize}

\subsection{Block synchronization}
\begin{itemize}
    \item \textit{Scope} of lock: time between acquisition and release.
    \item \texttt{synchronized} method: large scope if only small part manipulates shared data.
    \item Better: synchronize only the block of code manipulating shared data (smaller lock scope).
    \item Java allows block synchronization:
    \begin{verbatim}
public void someMethod() {
   /* non-critical section */
 
   synchronized(this) {
      /* critical section */
   }
 
   /* remainder section */
}
    \end{verbatim}
    \item Only critical-section access requires ownership of \texttt{this} object lock.
    \item When thread calls \texttt{wait()} method:
    \begin{enumerate}
        \item Releases lock for the object.
        \item Thread state set to blocked.
        \item Thread placed in wait set for the object.
    \end{enumerate}
    \item Example (Figure \ref{fig:7.4.3}):
    \begin{itemize}
        \item Producer calls \texttt{insert()}, buffer full, calls \texttt{wait()}.
        \item \texttt{wait()} releases lock, blocks producer, puts producer in wait set.
        \item Consumer enters \texttt{remove()} (lock now available), frees buffer space.
    \end{itemize}
    \item Entry and wait sets illustrated in Figure \ref{fig:7.4.4}.
    \item \texttt{insert()} and \texttt{remove()} methods using \texttt{wait()} and \texttt{notify()} (Figure \ref{fig:7.4.3}):
    \begin{verbatim}
/* Producers call this method */
public synchronized void insert(E item) {
    while (count == BUFFER_SIZE) {
      try {
        wait();
      }
      catch (InterruptedException ie) { }
    }
 
    buffer[in] = item;
    in = (in + 1) % BUFFER_SIZE;
    count++;
 
    notify();
}

/* Consumers call this method */
public synchronized E remove() {
    E item;
 
    while (count == 0) {
      try {
        wait();
      }
      catch (InterruptedException ie) { }
    }
 
    item = buffer[out];
    out = (out + 1) % BUFFER_SIZE;
    count--;
 
    notify();
 
    return item;
}
    \end{verbatim}
    \item \texttt{notify()} method:
    \begin{itemize}
        \item Picks arbitrary thread \texttt{T} from wait set.
        \item Moves \texttt{T} from wait set to entry set.
        \item Sets state of \texttt{T} from blocked to runnable.
    \end{itemize}
    \item \texttt{T} now eligible to compete for lock.
    \item Once \texttt{T} regains lock, returns from \texttt{wait()}, rechecks \texttt{count}.
    \item \texttt{notify()} ignored if no thread in wait set.
    \item Sequence of events with \texttt{wait()} and \texttt{notify()}:
    \begin{itemize}
        \item Buffer full, lock available.
        \item Producer calls \texttt{insert()}, enters, finds buffer full, calls \texttt{wait()}.
        \item \texttt{wait()} releases lock, blocks producer, puts producer in wait set.
        \item Consumer calls \texttt{remove()}, enters (lock available), removes item, calls \texttt{notify()}. Consumer still owns lock.
        \item \texttt{notify()} moves producer from wait set to entry set, sets state to runnable.
        \item Consumer exits \texttt{remove()}, releases lock.
        \item Producer reacquires lock, resumes from \texttt{wait()}.
        \item Producer tests \texttt{while} loop, finds room, proceeds with \texttt{insert()}.
        \item Producer exits \texttt{insert()}, releases lock.
    \end{itemize}
    \item \texttt{synchronized}, \texttt{wait()}, \texttt{notify()} are original Java mechanisms.
    \item Later Java API revisions introduced more flexible/robust locking.
\end{itemize}

\subsection{Reentrant locks}
\begin{itemize}
    \item Simplest locking mechanism in API: \texttt{ReentrantLock}.
    \item Similar to \texttt{synchronized} statement: owned by single thread, provides mutual exclusive access to shared resource.
    \item Additional features: e.g., \textit{fairness} parameter (favors longest-waiting thread).
    \item Acquisition: invoke \texttt{lock()} method.
    \begin{itemize}
        \item If lock available OR invoking thread already owns it (\textbf{\textit{reentrant}}): \texttt{lock()} assigns ownership, returns control.
        \item If lock unavailable: invoking thread blocks until owner invokes \texttt{unlock()}.
    \end{itemize}
    \item Implements \texttt{Lock} interface.
    \item Usage example:
    \begin{verbatim}
Lock key = new ReentrantLock();
 
key.lock();
try {
  /* critical section */
}
finally {
  key.unlock();
}
    \end{verbatim}
    \item \texttt{try} and \texttt{finally} idiom:
    \begin{itemize}
        \item Ensures lock is released (via \texttt{unlock()}) after critical section completes or if exception occurs in \texttt{try} block.
        \item \texttt{lock()} not placed in \texttt{try} clause because it doesn't throw checked exceptions.
        \item Avoids \texttt{IllegalMonitorStateException} if unchecked exception occurs during \texttt{lock()} invocation (e.g., \texttt{OutofMemoryError}), which would obscure original failure reason.
    \end{itemize}
    \item \texttt{ReentrantLock} provides mutual exclusion.
    \item \texttt{ReentrantReadWriteLock}: Java API also provides this for scenarios with more readers than writers.
    \begin{itemize}
        \item Allows multiple concurrent readers but only one writer.
    \end{itemize}
\end{itemize}

\subsection{Semaphores}
\begin{itemize}
    \item Java API provides counting semaphore (as in Section \hyperref[sec:6.6]{6.6}).
    \item Constructor: \texttt{Semaphore(int value)}.
    \item \texttt{value}: initial value of semaphore (negative allowed).
    \item \texttt{acquire()} method: throws \texttt{InterruptedException} if acquiring thread interrupted.
    \item Example using semaphore for mutual exclusion:
    \begin{verbatim}
Semaphore sem = new Semaphore(1);
 
try {
  sem.acquire();
  /* critical section */
}
catch (InterruptedException ie) { }
finally {
  sem.release();
}
    \end{verbatim}
    \item \texttt{release()} placed in \texttt{finally} clause to ensure semaphore is released.
\end{itemize}

\subsection{Condition variables}
\begin{itemize}
    \item Java API utility: condition variable.
    \item Functionality similar to \texttt{wait()} and \texttt{notify()} methods.
    \item Must be associated with a reentrant lock for mutual exclusion.
    \item Creation:
    \begin{enumerate}
        \item Create a \texttt{ReentrantLock}.
        \item Invoke its \texttt{newCondition()} method.
    \end{enumerate}
    \item Returns a \texttt{Condition} object (representing condition variable for associated \texttt{ReentrantLock}).
    \item Example:
    \begin{verbatim}
Lock key = new ReentrantLock();
Condition condVar = key.newCondition();
    \end{verbatim}
    \item Operations: \texttt{await()} and \texttt{signal()} methods.
    \item Function same as \texttt{wait()} and \texttt{signal()} in Section \hyperref[sec:6.7]{6.7}.
    \item Named vs. unnamed condition variables:
    \begin{itemize}
        \item Monitors (Section \hyperref[sec:6.7]{6.7}): \texttt{wait()} and \texttt{signal()} apply to \textbf{\textit{named}} condition variables.
        \item Java (language level): does not provide named condition variables.
        \item Each Java monitor: associated with one unnamed condition variable.
        \item \texttt{wait()} and \texttt{notify()} (Section Java monitors): apply only to this single unnamed condition variable.
        \item When Java thread awakened via \texttt{notify()}: receives no info on why; reactivated thread must check condition itself.
        \item Condition variables (this section): remedy this by allowing specific thread to be notified.
    \end{itemize}
    \item Example: Five threads (0-4), shared variable \texttt{turn}.
    \item \texttt{doWork(int threadNumber)} method (Figure \ref{fig:7.4.5}):
    \begin{itemize}
        \item Only thread with \texttt{threadNumber} matching \texttt{turn} proceeds; others wait.
        \item \texttt{doWork()} method:
        \begin{verbatim}
/* threadNumber is the thread that wishes to do some work */
public void doWork(int threadNumber)
{
  lock.lock();
 
  try {
    /**
     * If it's not my turn, then wait
     * until I'm signaled.
     */
    if (threadNumber != turn)
      condVars[threadNumber].await();
 
    /**
     * Do some work for awhile ...
     */
 
    /**
     * Now signal to the next thread.
     */
    turn = (turn + 1) % 5;
    condVars[turn].signal();
  }
  catch (InterruptedException ie) { }
  finally {
    lock.unlock();
  }
}
        \end{verbatim}
    \end{itemize}
    \item Requires \texttt{ReentrantLock} and five condition variables (\texttt{condVars}).
    \item Initialization:
    \begin{verbatim}
Lock lock = new ReentrantLock();
Condition[] condVars = new Condition[5];
 
for (int i = 0; i < 5; i++)
  condVars[i] = lock.newCondition();
    \end{verbatim}
    \item When thread enters \texttt{doWork()}:
    \begin{itemize}
        \item If \texttt{threadNumber != turn}: invokes \texttt{await()} on its associated condition variable.
        \item Resumes only when signaled by another thread.
        \item After work: signals condition variable for next thread's turn.
    \end{itemize}
    \item \texttt{doWork()} does not need to be \texttt{synchronized}.
    \begin{itemize}
        \item \texttt{ReentrantLock} provides mutual exclusion.
        \item \texttt{await()} on condition variable releases associated \texttt{ReentrantLock}.
        \item \texttt{signal()} only signals condition variable; lock released by \texttt{unlock()}.
    \end{itemize}
    \item Programming problems/projects at end of chapter use Pthreads mutex locks, condition variables, and POSIX semaphores.
\end{itemize}

\subsection{Section glossary}
\begin{itemize}
    \item \textit{entry set}: In Java, the set of threads waiting to enter a monitor.
    \item \textit{wait set}: In Java, a set of threads, each waiting for a condition that will allow it to continue.
    \item \textit{scope}: The time between when a lock is acquired and when it is released.
\end{itemize}
