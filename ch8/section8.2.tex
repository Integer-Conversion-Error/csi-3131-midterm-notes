\section{Deadlock in multithreaded applications}

\begin{itemize}
    \item \textbf{Pthreads Example}: Illustrates deadlock with POSIX mutex locks.
    \begin{itemize}
        \item \texttt{pthread\_mutex\_init()}: Initializes an unlocked mutex.
        \item \texttt{pthread\_mutex\_lock()}: Acquires a lock; blocks if the lock is held.
        \item \texttt{pthread\_mutex\_unlock()}: Releases a lock.
    \end{itemize}
    \item \textbf{Deadlock Scenario}:
    \begin{itemize}
        \item Two mutex locks created: \texttt{first\_mutex}, \texttt{second\_mutex}.
        \item Two threads, \texttt{thread\_one} and \texttt{thread\_two}, access both locks.
        \item \texttt{thread\_one} locks in order: (1) \texttt{first\_mutex}, (2) \texttt{second\_mutex}.
        \item \texttt{thread\_two} locks in order: (1) \texttt{second\_mutex}, (2) \texttt{first\_mutex}.
        \item \textbf{Deadlock possible}: If \texttt{thread\_one} acquires \texttt{first\_mutex} and \texttt{thread\_two} acquires \texttt{second\_mutex}, both threads will block waiting for the other's lock.
    \end{itemize}
\end{itemize}

\subsection{Deadlock Example Code}
\begin{verbatim}
/* thread_one runs in this function */
void *do_work_one(void *param)
{
   pthread_mutex_lock(&first_mutex);
   pthread_mutex_lock(&second_mutex);
   /**
     * Do some work
     */
   pthread_mutex_unlock(&second_mutex);
   pthread_mutex_unlock(&first_mutex);
 
   pthread_exit(0);
}
 
/* thread_two runs in this function */
void *do_work_two(void *param)
{
   pthread_mutex_lock(&second_mutex);
   pthread_mutex_lock(&first_mutex);
   /**
    * Do some work
    */
   pthread_mutex_unlock(&first_mutex);
   pthread_mutex_unlock(&second_mutex);
 
   pthread_exit(0);
}
\end{verbatim}

\begin{itemize}
    \item \textbf{Intermittent Nature}:
    \begin{itemize}
        \item Deadlock might not occur if one thread acquires and releases both locks before the other thread starts.
        \item Occurrence depends on the CPU scheduler.
        \item Makes identifying and testing for deadlocks difficult.
    \end{itemize}
\end{itemize}

\subsection{Livelock}
\begin{itemize}
    \item \textbf{Liveness Failure}: Similar to deadlock, but threads are not blocked.
    \item \textbf{Condition}: A thread continuously attempts an action that fails.
    \item \textbf{Analogy}: Two people trying to pass in a hallway, repeatedly moving into each other's way. They are active but make no progress.
    \item \textbf{Pthreads Example with \texttt{pthread\_mutex\_trylock()}}:
    \begin{itemize}
        \item \texttt{pthread\_mutex\_trylock()} attempts to acquire a lock without blocking.
        \item \textbf{Livelock Scenario}: \texttt{thread\_one} acquires \texttt{first\_mutex}, \texttt{thread\_two} acquires \texttt{second\_mutex}. Both then call \texttt{trylock} on the other mutex, which fails. They release their locks and repeat indefinitely.
    \end{itemize}
\end{itemize}

\subsection{Livelock Example Code}
\begin{verbatim}
/* thread_one runs in this function */
void *do_work_one(void *param)
{
   int done = 0;
 
   while (!done) {
      pthread_mutex_lock(&first_mutex);
      if (pthread_mutex_trylock(&second_mutex)) {
         /**
          * Do some work
          */
         pthread_mutex_unlock(&second_mutex);
         pthread_mutex_unlock(&first_mutex);
         done = 1;
     }
     else
         pthread_mutex_unlock(&first_mutex);
   }
   pthread_exit(0);
}
 
/* thread_two runs in this function */
void *do_work_two(void *param)
{
   int done = 0;
 
   while (!done) {
     pthread_mutex_lock(&second_mutex);
     if (pthread_mutex_trylock(&first_mutex)) {
         /**
          * Do some work
          */
         pthread_mutex_unlock(&first_mutex);
         pthread_mutex_unlock(&second_mutex);
         done = 1;
     }
     else
         pthread_mutex_unlock(&second_mutex);
   }
 
   pthread_exit(0);
}
\end{verbatim}

\begin{itemize}
    \item \textbf{Avoidance}:
    \begin{itemize}
        \item Livelock often occurs when threads retry failing operations at the same time.
        \item Can be avoided by having threads retry at random times.
        \item \textbf{Ethernet Example}: Hosts involved in a network collision \textbf{backoff} for a random period before retransmitting.
    \end{itemize}
    \item \textbf{Rarity}: Less common than deadlock, but still a challenge in concurrent application design.
\end{itemize}

\subsection{Section glossary}
\begin{tabular}{p{0.2\textwidth}p{0.8\textwidth}}
    \toprule
    \textbf{Term} & \textbf{Definition} \\
    \midrule
    livelock & A condition in which a thread continuously attempts an action that fails. \\
    \bottomrule
\end{tabular}
