\section{Deadlock characterization}

\subsection{Necessary conditions}
A deadlock situation can arise if the following four conditions hold simultaneously in a system:
\begin{itemize}
    \item \textbf{Mutual exclusion:} At least one resource must be held in a nonsharable mode.
    \item \textbf{Hold and wait:} A thread must be holding at least one resource and waiting to acquire additional resources held by other threads.
    \item \textbf{No preemption:} A resource can be released only voluntarily by the thread holding it.
    \item \textbf{Circular wait:} A set of waiting threads $\{T_0, T_1, \dots, T_n\}$ must exist such that $T_0$ is waiting for a resource held by $T_1$, $T_1$ is waiting for a resource held by $T_2$, \dots, $T_{n-1}$ is waiting for a resource held by $T_n$, and $T_n$ is waiting for a resource held by $T_0$.
\end{itemize}
All four conditions must hold for a deadlock to occur. The circular-wait condition implies the hold-and-wait condition.

\subsection{Resource-allocation graph}
Deadlocks can be described using a directed graph called a \textbf{system resource-allocation graph}.
\begin{itemize}
    \item The graph consists of a set of vertices V and a set of edges E.
    \item Vertices V are partitioned into two types:
    \begin{itemize}
        \item T = $\{T_1, T_2, \dots, T_n\}$, the set of all active threads.
        \item R = $\{R_1, R_2, \dots, R_m\}$, the set of all resource types.
    \end{itemize}
    \item A directed edge from thread $T_i$ to resource type $R_j$ ($T_i \rightarrow R_j$) is a \textbf{request edge}; it signifies that $T_i$ has requested an instance of $R_j$.
    \item A directed edge from resource type $R_j$ to thread $T_i$ ($R_j \rightarrow T_i$) is an \textbf{assignment edge}; it signifies that an instance of $R_j$ has been allocated to $T_i$.
    \item If the graph contains no cycles, no thread is deadlocked.
    \item If the graph contains a cycle, a deadlock may exist.
    \begin{itemize}
        \item If each resource type has exactly one instance, a cycle implies a deadlock has occurred.
        \item If each resource type has several instances, a cycle is a necessary but not a sufficient condition for deadlock.
    \end{itemize}
\end{itemize}

\subsection*{Section glossary}
\begin{tabular}{p{0.3\textwidth}p{0.7\textwidth}}
\toprule
\textbf{Term} & \textbf{Definition} \\
\midrule
\rowcolor{gray!10}
system resource-allocation graph & A directed graph for precise description of deadlocks. \\
request edge & In a system resource-allocation graph, an edge (arrow) indicating a resource request. \\
\rowcolor{gray!10}
assignment edge & In a system resource-allocation graph, an edge (arrow) indicating a resource assignment. \\
\bottomrule
\end{tabular}
