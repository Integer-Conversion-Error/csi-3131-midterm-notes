\section{Deadlock avoidance}

\begin{itemize}[noitemsep, topsep=0pt]
    \item Deadlock-prevention limits how requests are made, ensuring no necessary condition occurs.
    \item Side effects: low device utilization, reduced system throughput.
    \item Alternative: require additional info on resource requests.
    \item System needs to know max resources each thread may need (a priori information).
    \item Deadlock-avoidance algorithm dynamically examines resource-allocation state to prevent circular-wait.
    \item Resource-allocation \textbf{state}: defined by available/allocated resources and max demands.
\end{itemize}

\subsection*{Safe state}
\begin{itemize}[noitemsep, topsep=0pt]
    \item A state is \textbf{safe} if system can allocate resources to each thread (up to max) in some order and avoid deadlock.
    \item System is safe if a \textbf{safe sequence} exists.
    \item Safe sequence $<T_1, T_2, \dots, T_n>$: for each $T_i$, its resource requests can be met by currently available resources + resources held by all $T_j$ ($j < i$).
    \item Safe state is not deadlocked; deadlocked state is unsafe.
    \item Not all unsafe states are deadlocks, but unsafe states \textbf{may} lead to deadlock.
    \item OS avoids unsafe states as long as state is safe.
    \item In unsafe state, OS cannot prevent deadlocks; thread behavior controls unsafe states.
\end{itemize}

\subsection*{Resource-allocation-graph algorithm}
\begin{itemize}[noitemsep, topsep=0pt]
    \item For systems with only one instance of each resource type.
    \item Introduces \textbf{claim edge} ($T_i \to R_j$ dashed line): indicates $T_i$ may request $R_j$ in future.
    \item Claim edge $T_i \to R_j$ converted to request edge when $T_i$ requests $R_j$.
    \item Request edge $T_i \to R_j$ converted to assignment edge when $R_j$ allocated to $T_i$.
    \item Assignment edge $R_j \to T_i$ converted to claim edge when $R_j$ released by $T_i$.
    \item Resources must be claimed a priori.
    \item Request granted only if no cycle formed in graph (checked by cycle-detection algorithm).
    \item Cycle indicates unsafe state.
    \item Algorithm complexity: $O(n^2)$ operations, where $n$ is number of threads.
\end{itemize}

\subsection*{Banker's algorithm}
\begin{itemize}[noitemsep, topsep=0pt]
    \item Applicable to systems with multiple instances of each resource type.
    \item Less efficient than resource-allocation graph scheme.
    \item When new thread enters, declares max instances of each resource type needed (cannot exceed total system resources).
    \item Request granted only if allocation leaves system in safe state.
    \item Data structures:
    \begin{itemize}[noitemsep, topsep=0pt]
        \item \textbf{Available}: vector of length $m$, number of available resources of each type.
        \item \textbf{Max}: $n \times m$ matrix, max demand of each thread.
        \item \textbf{Allocation}: $n \times m$ matrix, resources currently allocated to each thread.
        \item \textbf{Need}: $n \times m$ matrix, remaining resource need of each thread ($Need[i][j] = Max[i][j] - Allocation[i][j]$).
    \end{itemize}
\end{itemize}

\subsubsection*{Safety algorithm}
\begin{enumerate}[noitemsep, topsep=0pt]
    \item Initialize $Work = Available$, $Finish[i] = false$ for $i = 0, 1, \dots, n-1$.
    \item Find index $i$ such that $Finish[i] == false$ and $Need_i \le Work$. If no such $i$, go to step 4.
    \item $Work = Work + Allocation_i$; $Finish[i] = true$. Go to step 2.
    \item If $Finish[i] == true$ for all $i$, system is in safe state.
\end{enumerate}
\begin{itemize}[noitemsep, topsep=0pt]
    \item Algorithm complexity: $O(m \times n^2)$ operations.
\end{itemize}

\subsubsection*{Resource-request algorithm}
\begin{itemize}[noitemsep, topsep=0pt]
    \item Let $Request_i$ be request vector for thread $T_i$.
    \begin{enumerate}[noitemsep, topsep=0pt]
        \item If $Request_i \le Need_i$, go to step 2. Else, error (thread exceeded max claim).
        \item If $Request_i \le Available$, go to step 3. Else, $T_i$ must wait (resources unavailable).
        \item Pretend allocation:
        \begin{itemize}[noitemsep, topsep=0pt]
            \item $Available = Available - Request_i$
            \item $Allocation_i = Allocation_i + Request_i$
            \item $Need_i = Need_i - Request_i$
        \end{itemize}
        \item If resulting state is safe (using Safety Algorithm), grant request. Else, $T_i$ waits, restore old state.
    \end{enumerate}
\end{itemize}

\subsection*{Section glossary}
\begin{tabular}{p{0.3\textwidth}p{0.7\textwidth}}
    \toprule
    \textbf{Term} & \textbf{Definition} \\
    \midrule
    \textbf{safe sequence} & Sequence of processes $<P_1, P_2, \dots, P_n>$ where each $P_i$'s requests can be met by available resources + resources held by $P_j$ ($j < i$). \\
    \textbf{claim edge} & In resource-allocation graph, an edge indicating a process might claim a resource in the future. \\
    \textbf{banker's algorithm} & Deadlock avoidance algorithm for multiple resource instances, less efficient than resource-allocation graph. \\
    \bottomrule
\end{tabular}
