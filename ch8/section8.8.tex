\section{Recovery from deadlock}

\begin{itemize}
    \item When deadlock detected, options:
    \begin{itemize}
        \item Inform operator (manual recovery).
        \item System recovers automatically.
    \end{itemize}
    \item Two options for breaking deadlock:
    \begin{itemize}
        \item Abort one or more threads (break circular wait).
        \item Preempt resources from deadlocked threads.
    \end{itemize}
\end{itemize}

\subsection{Process and thread termination}
\begin{itemize}
    \item Eliminate deadlocks by aborting process/thread.
    \item System reclaims all resources.
    \item Methods:
    \begin{itemize}
        \item \textbf{Abort all deadlocked processes}: Breaks cycle, but expensive (discarded computations, recomputation needed).
        \item \textbf{Abort one process at a time until deadlock eliminated}: High overhead (deadlock-detection after each abort).
    \end{itemize}
    \item Aborting process issues:
    \begin{itemize}
        \item File in incorrect state if updating.
        \item Shared data integrity issues if updating while holding mutex lock (must restore lock status).
    \end{itemize}
    \item If partial termination, determine which process to terminate (policy decision, economic).
    \item Factors for choosing victim (minimum cost):
    \begin{itemize}
        \item Process priority.
        \item Computation time (how long computed, how much longer).
        \item Resources used (types, ease of preemption).
        \item Resources needed to complete.
        \item Number of processes to terminate.
    \end{itemize}
\end{itemize}

\subsection{Resource preemption}
\begin{itemize}
    \item Successively preempt resources from processes, give to others until deadlock broken.
    \item Three issues:
    \begin{itemize}
        \item \textbf{Selecting a victim}: Which resources/processes to preempt? Minimize cost (e.g., resources held, time consumed).
        \item \textbf{Rollback}: What to do with preempted process?
        \begin{itemize}
            \item Cannot continue normal execution (missing resource).
            \item Roll back to safe state, restart.
            \item Simplest: total rollback (abort, restart).
            \item More effective: roll back only as necessary (requires more state info).
        \end{itemize}
        \item \textbf{Starvation}: How to ensure resources not always preempted from same process?
        \begin{itemize}
            \item Process never completes.
            \item Ensure process picked as victim finite number of times.
            \item Common solution: include number of rollbacks in cost factor.
        \end{itemize}
    \end{itemize}
\end{itemize}

\subsection*{Section glossary}
\begin{tabular}{p{0.3\textwidth}p{0.7\textwidth}}
    \toprule
    \textbf{Term} & \textbf{Definition} \\
    \midrule
    \textbf{recovery mode} & A system boot state providing limited services and designed to enable the system admin to repair system problems and debug system startup. \\
    \bottomrule
\end{tabular}
