\subsection{9.2 Contiguous memory allocation}

\begin{itemize}
    \item Main memory accommodates OS and user processes.
    \item Need efficient memory allocation.
    \item Memory usually divided into two partitions: one for OS, one for user processes.
    \item OS can be in low or high memory (many OS, including Linux/Windows, use high memory).
    \item Several user processes reside in memory concurrently.
    \item \textbf{Contiguous memory allocation}: each process in single contiguous memory section.
\end{itemize}

\subsubsection*{Memory protection}
\begin{itemize}
    \item Prevent process from accessing unowned memory.
    \item Combine \textbf{relocation register} (smallest physical address) and \textbf{limit register} (range of logical addresses).
    \item Each logical address must fall within limit register range.
    \item MMU maps logical address dynamically by adding relocation register value.
    \item Mapped address sent to memory.
    \item CPU scheduler loads relocation and limit registers during context switch.
    \item Every CPU-generated address checked against these registers.
    \item Protects OS and other user programs/data from modification by running process.
    \item Relocation-register scheme allows dynamic OS size changes.
    \item Desirable for device drivers: load only when needed, remove when no longer needed.
\end{itemize}

\subsubsection*{Memory allocation}
\begin{itemize}
    \item Simplest method: assign processes to variably sized partitions.
    \item Each partition contains exactly one process (\textbf{variable-partition} scheme).
    \item OS keeps table of available/occupied memory parts.
    \item Initially: all memory available for user processes, one large block (\textbf{hole}).
    \item Eventually: memory contains set of holes of various sizes.
    \item When process arrives: OS considers memory requirements and available space.
    \item Allocated space: process loaded into memory, competes for CPU time.
    \item Process terminates: releases memory, OS provides to another process.
    \item Insufficient memory for arriving process: reject or place in wait queue.
    \item Memory blocks: set of holes scattered throughout memory.
    \item Process needs memory: system searches for large enough hole.
    \item If hole too large: split into two parts (one allocated, other returned to holes).
    \item Process terminates: releases block, placed back in holes.
    \item New hole adjacent to others: merged to form larger hole.
    \item This procedure: instance of \textbf{dynamic storage-allocation problem}.
    \item Common strategies to select free hole:
    \begin{itemize}
        \item \textbf{First-fit}: Allocate first hole big enough. Search from beginning or last search end. Stop when large enough hole found.
        \item \textbf{Best-fit}: Allocate smallest hole big enough. Must search entire list (unless ordered by size). Produces smallest leftover hole.
        \item \textbf{Worst-fit}: Allocate largest hole. Must search entire list (unless sorted by size). Produces largest leftover hole (may be more useful).
    \end{itemize}
    \item Simulations: first-fit and best-fit better than worst-fit (decreasing time, storage utilization).
    \item Neither first-fit nor best-fit clearly better for storage utilization, but first-fit generally faster.
\end{itemize}

\subsubsection*{Fragmentation}
\begin{itemize}
    \item First-fit and best-fit suffer from \textbf{external fragmentation}.
    \item Processes loaded/removed: free memory broken into small pieces.
    \item External fragmentation: enough total memory, but spaces not contiguous (storage fragmented).
    \item Problem can be severe: wasted memory between processes.
    \item If small pieces were one big block: could run more processes.
    \item Strategy (first-fit/best-fit) affects fragmentation amount.
    \item Which end of free block allocated also a factor.
    \item External fragmentation always a problem.
    \item Statistical analysis (first fit): 0.5 N blocks lost to fragmentation for N allocated blocks.
    \item \textbf{50-percent rule}: one-third of memory unusable.
    \item Memory fragmentation: internal and external.
    \item \textbf{Internal fragmentation}: unused memory internal to a partition.
    \item Occurs when allocated memory slightly larger than requested (e.g., fixed-sized blocks).
    \item Solution to external fragmentation: \textbf{compaction}.
    \item Goal: shuffle memory contents, place all free memory together in one large block.
    \item Compaction not always possible:
    \begin{itemize}
        \item If relocation static (assembly/load time): cannot compact.
        \item Possible only if relocation dynamic (execution time).
    \end{itemize}
    \item If dynamic relocation: move program/data, change base register.
    \item Compaction can be expensive (e.g., move all processes to one end).
    \item Another solution to external fragmentation: permit noncontiguous logical address space.
    \item Allows process to be allocated physical memory wherever available.
    \item Strategy used in \textbf{paging} (most common memory-management technique).
    \item Fragmentation: general problem in computing (storage management chapters).
\end{itemize}

\subsubsection*{Section glossary}
\begin{tabular}{p{0.3\textwidth}p{0.7\textwidth}}
    \toprule
    \textbf{Term} & \textbf{Definition} \\
    \midrule
    \textbf{contiguous memory allocation} & Memory allocation method where each process is in a single contiguous memory section. \\
    \textbf{variable-partition} & Memory-allocation scheme where each memory partition contains exactly one process. \\
    \textbf{hole} & In variable partition memory allocation, a contiguous section of unused memory. \\
    \textbf{dynamic storage-allocation problem} & Problem of satisfying a memory request of size n from a list of free holes. \\
    \textbf{first-fit} & In memory allocation, selecting the first hole large enough for a request. \\
    \textbf{best-fit} & In memory allocation, selecting the smallest hole large enough for a request. \\
    \textbf{worst-fit} & In memory allocation, selecting the largest hole available. \\
    \textbf{external fragmentation} & Fragmentation where available memory has holes that together are enough, but no single hole is large enough. \\
    \textbf{50-percent rule} & Statistical finding that fragmentation may result in 50 percent space loss. \\
    \textbf{internal fragmentation} & Unused memory internal to a partition. \\
    \textbf{compaction} & Shuffling storage to consolidate used space and create one or more large holes. \\
    \bottomrule
\end{tabular}
