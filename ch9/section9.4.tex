\section{Structure of the page table}

\subsection{Hierarchical paging}
\begin{itemize}
    \item Explores common techniques for structuring the page table: hierarchical paging, hashed page tables, and inverted page tables.
    \item Modern computer systems support large logical address spaces ($2^{32}$ to $2^{64}$).
    \item Page table itself becomes excessively large.
    \item \textbf{Example}: 32-bit logical address space, 4 KB page size ($2^{12}$).
    \begin{itemize}
        \item Page table: over 1 million entries ($2^{20}$).
        \item Each entry: 4 bytes.
        \item Each process: up to 4 MB physical address space for page table alone.
        \item Solution: divide page table into smaller pieces.
    \end{itemize}
    \item \textbf{Two-level paging algorithm}: page table itself is paged.
    \begin{itemize}
        \item 32-bit logical address space, 4 KB page size.
        \item Logical address divided into:
        \begin{itemize}
            \item Page number: 20 bits.
            \item Page offset: 12 bits.
        \end{itemize}
        \item Page number further divided:
        \begin{itemize}
            \item $p_1$: 10-bit outer page number (index into outer page table).
            \item $p_2$: 10-bit inner page offset (displacement within inner page table).
        \end{itemize}
        \item Address translation: from outer page table inward.
        \item Also known as a \textbf{forward-mapped} page table.
    \end{itemize}
    \item \textbf{64-bit logical address space}: two-level paging scheme inappropriate.
    \begin{itemize}
        \item \textbf{Example}: 4 KB page size ($2^{12}$).
        \item Page table: up to $2^{52}$ entries.
        \item Inner page tables: one page long, $2^{10}$ 4-byte entries.
        \item Outer page table: still $2^{42}$ entries, or $2^{44}$ bytes (16 GB).
    \end{itemize}
    \item \textbf{Three-level paging scheme}: paging the outer page table.
    \begin{itemize}
        \item Outer page table made up of standard-size pages ($2^{10}$ entries, or $2^{12}$ bytes).
        \item 64-bit address space still daunting.
        \item 64-bit UltraSPARC: would require seven levels of paging (prohibitive memory accesses).
        \item Hierarchical page tables generally inappropriate for 64-bit architectures.
    \end{itemize}
\end{itemize}

\subsection{Hashed page tables}
\begin{itemize}
    \item Approach for handling address spaces larger than 32 bits.
    \item Hash value: virtual page number.
    \item Each entry in hash table: linked list of elements (to handle collisions).
    \item Each element consists of three fields:
    \begin{enumerate}
        \item Virtual page number.
        \item Value of mapped page frame.
        \item Pointer to next element in linked list.
    \end{enumerate}
    \item \textbf{Algorithm}:
    \begin{itemize}
        \item Virtual page number in virtual address hashed into hash table.
        \item Virtual page number compared with field 1 in first element of linked list.
        \item Match: corresponding page frame (field 2) used to form physical address.
        \item No match: subsequent entries in linked list searched.
    \end{itemize}
    \item \textbf{Clustered page tables}: variation for 64-bit address spaces.
    \begin{itemize}
        \item Similar to hashed page tables.
        \item Each entry refers to several pages (e.g., 16) instead of a single page.
        \item Single page-table entry stores mappings for multiple physical-page frames.
        \item Useful for \textbf{sparse} address spaces (memory references noncontiguous, scattered).
    \end{itemize}
\end{itemize}

\subsection{Inverted page tables}
\begin{itemize}
    \item Usually, each process has an associated page table.
    \item Standard page table: one entry for each page process is using (or each virtual address).
    \item Drawback: each page table may consist of millions of entries, consuming large amounts of physical memory.
    \item \textbf{Inverted page table}: one entry for each real page (frame) of memory.
    \item Each entry: virtual address of page stored in that real memory location, plus process information.
    \item Only one page table in system, one entry per physical memory page.
    \item Often requires an address-space identifier (process-id) in each entry.
    \item Ensures logical page for particular process maps to corresponding physical page frame.
    \item \textbf{Examples}: 64-bit UltraSPARC and PowerPC.
    \item \textbf{IBM RT simplified version}:
    \begin{itemize}
        \item Virtual address: \texttt{<process-id, page-number, offset>}.
        \item Inverted page-table entry: \texttt{<process-id, page-number>}.
        \item Memory reference: \texttt{<process-id, page-number>} presented to memory subsystem.
        \item Inverted page table searched for match.
        \item Match at entry $i$: physical address \texttt{<i, offset>} generated.
        \item No match: illegal address access.
    \end{itemize}
    \item \textbf{Drawback}: increases time to search table (sorted by physical address, lookups by virtual address).
    \item \textbf{Alleviation}: use a hash table to limit search.
    \item Each access to hash table adds memory reference: one virtual memory reference requires at least two real memory reads (hash-table entry + page table).
    \item TLB searched first for performance improvement.
    \item \textbf{Shared memory issue}:
    \begin{itemize}
        \item Standard paging: multiple virtual addresses map to same physical address.
        \item Inverted page tables: only one virtual page entry for every physical page.
        \item One physical page cannot have two (or more) shared virtual addresses.
        \item Reference by another process sharing memory: page fault, replaces mapping.
    \end{itemize}
\end{itemize}

\subsection{Oracle SPARC Solaris}
\begin{itemize}
    \item Modern 64-bit CPU and OS tightly integrated for low-overhead virtual memory.
    \item \textbf{Solaris} running on \textbf{SPARC} CPU: fully 64-bit OS.
    \item Solves virtual memory problem efficiently using hashed page tables.
    \item Two hash tables: one for kernel, one for all user processes.
    \item Each maps virtual to physical memory.
    \item Each hash-table entry: contiguous area of mapped virtual memory (more efficient than per-page entry).
    \item Entry has base address and span (number of pages represented).
    \item \textbf{TLB (Translation Lookaside Buffer)}: holds translation table entries (TTEs) for fast hardware lookups.
    \item Cache of TTEs: \textbf{translation storage buffer (TSB)}.
    \item TSB includes entry per recently accessed page.
    \item \textbf{Virtual address reference process}:
    \begin{enumerate}
        \item Hardware searches TLB for translation.
        \item None found: hardware walks through in-memory TSB for TTE. (\textbf{TLB walk})
        \item Match in TSB: CPU copies TSB entry into TLB, memory translation completes.
        \item No match in TSB: kernel interrupted to search hash table.
        \item Kernel creates TTE from hash table, stores in TSB for automatic loading into TLB by MMU.
        \item Interrupt handler returns control to MMU, completes address translation, retrieves data.
    \end{enumerate}
\end{itemize}

\vspace{1em}
\subsection*{Section glossary}
\begin{tabular}{p{0.3\textwidth}p{0.7\textwidth}}
    \toprule
    \textbf{Term} & \textbf{Definition} \\
    \midrule
    \textbf{forward-mapped} & Scheme for hierarchical page tables where address translation starts at the outer page table and moves inward. \\
    \textbf{hashed page table} & A page table that is hashed for faster access; the hash value is the virtual page number. \\
    \textbf{clustered page table} & Similar to a hashed page table, but an entry refers to a cluster of several pages. \\
    \textbf{sparse} & In memory management, describes a page table with noncontiguous, scattered entries; an address space with many holes. \\
    \textbf{inverted page table} & A page-table scheme with one entry for each real physical page frame in memory, mapping to a logical page (virtual address) value. \\
    \textbf{Solaris} & A UNIX derivative, main operating system of Sun Microsystems (now Oracle); active open source version called Illumos. \\
    \textbf{SPARC} & A proprietary RISC CPU created by Sun Microsystems (now Oracle); active open source version called OpenSPARC. \\
    \textbf{TLB walk} & Steps involved in traversing page-table structures to locate a needed translation and copying the result into the TLB. \\
    \bottomrule
\end{tabular}
