\section{Swapping}

\begin{itemize}
    \item Process instructions and data must be in memory for execution.
    \item Process or portion can be \textbf{swapped} temporarily out of memory to a \textbf{backing store}.
    \item Then brought back into memory for continued execution (Figure 9.5.1).
    \item Swapping allows total physical address space of all processes to exceed real physical memory.
    \item Increases degree of multiprogramming.
\end{itemize}

\subsection{Standard swapping}
\begin{itemize}
    \item Involves moving entire processes between main memory and backing store.
    \item Backing store: fast secondary storage, large enough for process parts, direct access to memory images.
    \item When process/part swapped to backing store, associated data structures (including per-thread data for multithreaded processes) must be written.
    \item OS maintains metadata for swapped-out processes for restoration.
    \item Advantage: allows physical memory to be oversubscribed, accommodates more processes than physical memory.
    \item Idle/mostly idle processes good candidates for swapping.
    \item Memory allocated to inactive processes can be dedicated to active processes.
    \item If inactive swapped-out process becomes active, must be swapped back in (Figure 9.5.1).
\end{itemize}

\subsection{Swapping with paging}
\begin{itemize}
    \item Standard swapping generally no longer used in contemporary OS (exception: Solaris under dire circumstances).
    \item Reason: time to move entire processes prohibitive.
    \item Most systems (Linux, Windows) use variation: pages of a process swapped, not entire process.
    \item Still allows physical memory oversubscription.
    \item Does not incur cost of swapping entire processes (only small number of pages involved).
    \item Term \textbf{swapping} now generally refers to standard swapping.
    \item \textbf{Paging} refers to swapping with paging.
    \item \textbf{Page out}: moves page from memory to backing store.
    \item \textbf{Page in}: reverse process.
    \item Illustrated in Figure 9.5.2 (subset of pages for processes A and B paged-out/paged-in).
    \item Works well with virtual memory (Chapter Virtual Memory).
\end{itemize}

\subsection{Swapping on mobile systems}
\begin{itemize}
    \item Mobile systems typically do not support swapping.
    \item Reasons:
    \begin{itemize}
        \item Flash memory (not hard disks) for nonvolatile storage $\rightarrow$ space constraint.
        \item Limited number of writes flash memory tolerates before unreliability.
        \item Poor throughput between main memory and flash memory.
    \end{itemize}
    \item Instead of swapping:
    \begin{itemize}
        \item Apple's iOS: asks applications to voluntarily relinquish allocated memory when free memory low.
        \begin{itemize}
            \item Read-only data (code) removed from main memory, reloaded from flash if needed.
            \item Modified data (stack) never removed.
            \item Applications failing to free memory may be terminated by OS.
        \end{itemize}
        \item Android: similar strategy, may terminate process if insufficient free memory.
        \begin{itemize}
            \item Before termination, writes \textbf{application state} to flash memory for quick restart.
        \end{itemize}
    \end{itemize}
    \item Developers for mobile systems must carefully allocate/release memory to avoid excessive use or leaks.
\end{itemize}

\subsection*{System performance under swapping}
\begin{itemize}
    \item Swapping (any form) often sign of more active processes than available physical memory.
    \item Two approaches:
    \begin{itemize}
        \item Terminate some processes.
        \item Get more physical memory.
    \end{itemize}
\end{itemize}

\subsection*{Section glossary}
\begin{tabular}{p{0.3\textwidth}p{0.7\textwidth}}
    \toprule
    \textbf{Term} & \textbf{Definition} \\
    \midrule
    \textbf{swapped} & Moved between main memory and a backing store. Process swapped out to free main memory, then swapped back in to continue execution. \\
    \textbf{backing store} & Secondary storage area used for process swapping. \\
    \textbf{application state} & Software construct for data storage. \\
    \bottomrule
\end{tabular}
