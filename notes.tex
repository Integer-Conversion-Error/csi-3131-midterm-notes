\documentclass{article}
\usepackage[utf8]{inputenc}
\usepackage{amsmath}
\usepackage{amssymb}
\usepackage{booktabs}
\usepackage{enumitem}
\usepackage{geometry}
\usepackage{array}
\usepackage{ragged2e} % For better justification in p-columns
\usepackage{hyperref} % For clickable ToC and links
\usepackage{parskip} % For compact paragraphs
\linespread{0.9} % For more compact line spacing

\geometry{a4paper, margin=0.2in}

\hypersetup{
    colorlinks=true,
    linkcolor=blue,
    filecolor=magenta,      
    urlcolor=cyan,
    pdftitle={Operating Systems Notes},
    pdfpagemode=FullScreen,
}

\title{Operating Systems Notes}
\author{} % Author intentionally left blank
\date{\today}

\begin{document}
\maketitle
\tableofcontents
\newpage

\section*{Chapter 1 – Foundations of Operating Systems}
\subsection*{1 Purpose and Core Responsibilities}
\begin{itemize}
    \item \textbf{Intermediary:} OS sits between user \& applications and hardware, hiding details while exposing services.
    \item \textbf{Goals:}
    \begin{itemize}
        \item Run programs conveniently (consistent UI).
        \item Run programs efficiently (performance, safe sharing).
        \item Manage hardware resources (CPU, memory, I/O, storage).
    \end{itemize}
\end{itemize}

\subsection*{2 Abstraction Layers (high $\rightarrow$ low)}
\begin{enumerate}[label=\arabic*.]
    \item \textbf{User} – GUI, CLI, touch, voice.
    \item \textbf{Application programs} – browsers, compilers, games.
    \item \textbf{Operating system} – kernel, system programs, middleware.
    \item \textbf{Computer hardware} – CPU, memory, devices.
\end{enumerate}

\subsection*{3 User View vs. System View}
\begin{itemize}
    \item \textbf{User view:} Focus on ease of use; resource details hidden.
    \item \textbf{System view:}
    \begin{itemize}
        \item \textbf{Resource allocator} – arbitrates CPU, memory, I/O, etc.
        \item \textbf{Control program} – governs execution, manages devices.
    \end{itemize}
\end{itemize}

\subsection*{4 What Forms an Operating System?}
\centering
\begin{tabular}{>{\RaggedRight}p{0.28\textwidth} >{\RaggedRight}p{0.28\textwidth} >{\RaggedRight\arraybackslash}p{0.34\textwidth}}
\toprule
\textbf{Always} & \textbf{Usually} & \textbf{Optional / Varies} \\
\midrule
\textbf{Kernel} – always runs after boot. & \textbf{Device drivers}, loadable modules. & \textbf{Middleware} (graphics, DB), extra utilities. \\
& \textbf{System programs} (shells, daemons). & \\
\bottomrule
\end{tabular}
\vspace{\baselineskip} % Add some space after the table

\subsection*{5 Why Study Operating Systems?}
\begin{itemize}
    \item All code runs on an OS; knowing its policies \& APIs yields safer, faster, portable software.
    \item OS concepts (processes, memory, I/O, security) recur in servers, clouds, IoT.
\end{itemize}

\subsection*{6 Glossary Highlights (1.1)}
\begin{itemize}
    \item \textbf{Operating system, kernel, system program, middleware, resource allocator, control program, ease of use, resource utilization}
\end{itemize}

\section*{1.2 Computer-System Organization}
\subsection*{1 System-Level Hardware Layout}
\begin{itemize}
    \item Shared \textbf{system bus} connects CPU cores, main memory, device controllers.
    \item \textbf{Device controller:} logic per device class (disk, GPU, USB) with registers \& buffer.
    \item \textbf{Device driver:} kernel API shielding hardware details.
    \item \textbf{Parallelism:} CPU \& controllers run concurrently; \textbf{memory controller} arbitrates.
\end{itemize}

\subsection*{2 Interrupts – Hardware-Driven Events}
\subsubsection*{2.1 Lifecycle}
\begin{enumerate}[label=\arabic*.]
    \item Controller raises signal on \textbf{interrupt-request line}.
    \item CPU catches signal, saves state, jumps to handler via \textbf{interrupt vector}.
    \item Handler services device, restores state, executes \texttt{return\_from\_interrupt}.
\end{enumerate}

\subsubsection*{2.2 Efficiency Features}
\begin{itemize}
    \item \textbf{O(1) dispatch} via vector table.
    \item \textbf{Nonmaskable} vs. \textbf{maskable} lines (maskable can be disabled).
    \item \textbf{Priority \& interrupt chaining:} high-priority pre-empts low; vector entry may head list of handlers.
\end{itemize}

\subsubsection*{2.3 End-to-End I/O Timeline}
\texttt{User code $\rightarrow$ I/O request $\rightarrow$ controller/DMA busy $\rightarrow$ finish $\rightarrow$ interrupt $\rightarrow$ handler $\rightarrow$ resume user code}

\subsection*{3 Storage Structure}
\subsubsection*{3.1 Memory Hierarchy}
\centering
\begin{tabular}{lllll}
\toprule
\textbf{Layer} & \textbf{Volatile?} & \textbf{Size} & \textbf{Access} & \textbf{Notes} \\
\midrule
Registers & Yes & bytes & sub-ns & in CPU \\
Cache & Yes & KB–MB & ns & SRAM \\
Main memory & Yes & GBs & $\sim$10 ns & DRAM \\
NVM/SSD & No & 10 GB–TB & $\mu$s & electrical \\
HDD & No & 100 GB–TB & ms & mechanical \\
Optical/tape & No & TB–PB & s–min & archival \\
\bottomrule
\end{tabular}
\vspace{\baselineskip}

\subsubsection*{3.2 Key Units \& Terms}
\begin{itemize}
    \item \textbf{Bit $\rightarrow$ byte (8 bits) $\rightarrow$ word} (CPU width).
    \item \textbf{KiB / MiB / GiB / TiB / PiB} = powers of 1024.
    \item \textbf{Volatile} memory loses data on power-off; \textbf{non-volatile storage (NVS)} persists.
    \item \textbf{Firmware / EEPROM} stores bootstrap program.
\end{itemize}

\subsubsection*{3.3 Why Secondary Storage?}
\begin{itemize}
    \item Main memory is finite and volatile; programs \& data live on slower persistent media until loaded.
\end{itemize}

\subsection*{4 I/O Structure}
\subsubsection*{4.1 Direct Memory Access (DMA)}
\begin{itemize}
    \item Driver configures DMA engine; controller moves block without CPU, raises one interrupt on completion.
\end{itemize}

\subsubsection*{4.2 Bus vs. Switched Fabrics}
\begin{itemize}
    \item \textbf{Shared bus:} one transfer at a time; common on PCs.
    \item \textbf{Switch fabric:} concurrent links; used in servers \& SoCs.
\end{itemize}

\subsubsection*{4.3 Full I/O Cycle}
\begin{enumerate}[label=\arabic*.]
    \item Driver starts I/O.
    \item Controller activates device/DMA.
    \item CPU handles other tasks, checks interrupts.
    \item Device finishes, raises interrupt.
    \item Handler validates data, wakes waiting process.
\end{enumerate}

\subsection*{5 Glossary Highlights (1.2)}
\centering
\begin{tabular}{>{\RaggedRight}p{0.35\textwidth} >{\RaggedRight\arraybackslash}p{0.55\textwidth}}
\toprule
\textbf{Term} & \textbf{Definition} \\
\midrule
\textbf{Bus} & Shared path linking CPU, memory, controllers. \\
\textbf{Device driver} & Kernel interface to a controller. \\
\textbf{Interrupt / vector / request line} & Hardware signal and its dispatch mechanism. \\
\textbf{Maskable / nonmaskable interrupt} & Can or cannot be disabled. \\
\textbf{DMA} & Controller-driven block transfer to/from memory. \\
\textbf{Volatile / non-volatile memory} & Data lost or retained on power loss. \\
\bottomrule
\end{tabular}
\vspace{\baselineskip}

\end{document}
